\chapter{ASPECTOS SOCIAIS DO MOVIMENTO DO SOFTWARE LIVRE} \label{2}

Este capítulo tem como propósito apresentar os elementos políticos e sociais do movimento SL/CA presente na literatura universitária desenvolvida a seu respeito, e propor uma tipologia deles. Depois de apresentar os termos usados para designar os atuantes do movimento do software livre dentro das comunidades e nos sentidos comuns e universitários (\ref{2.1.1}) abordaremos o movimento através do conceito de ética hacker (\ref{2.1.3}), entendido num contexto teórico de construção social da ética (\ref{2.1.2}). Uma crítica construtiva desse conceito permite reinterpretar as significações culturais do hacking (\ref{2.2}) pela noção de "agnosticismo político" (\ref{2.2.1}). Isola-se então a paixão para a informação como matriz mínima das interações de natureza política dos usuários-desenvolvedores, permitindo então construir uma tipologia dela a partir dos trabalhos da antropóloga Gabriella Coleman (\ref{2.2.2}).

A partir disso, construiremos um paradigma de interpretação da atividade do usuário-desenvolvedor de SL/CA através da idéia de pragmáticas da programação (\ref{2.3}).  Após ter apresentado o ato de programar na sua dimensão artística e reguladora (\ref{2.3.1}) poderemos usar das pragmáticas de programação como a interpretação sociopolítica de um conceito lingüístico, permitindo interpretar as relações dos usuários-desenvolvedores com suas políticas informais e a informação (\ref{2.3.2}).

\section{A Ética Hacker} \label{2.1}

\subsection{Hackers, usuários-desenvolvedores, geeks e nerds} \label{2.1.1}

Numerosos termos designam as pessoas nos arredores do software livre.  Porém, significações suplementares vêm apontar categorias mais específicas ou gerais de indivíduos. Por essa razão, apresentaremos, com detalhes, quatro designações que foram encontrados com muita freqüência ao longo das pesquisas bibliográficas e de campo.

\subsubsection{Hacker} \label{2.1.1.a}

"Hacker" é provavelmente o termo o mais romanceado a respeito da cultura informática. A indústria cinematográfica de Hollywood já se aproveitou dos recursos cognitivos que se encontram por volta deste domínio: um "ego ilimitado" exprimido em um mundo de 0 e 1 enfrenta uma realidade manipulada, truncada, decepcionante, limitada. A trilogia dos irmãos Wachowski\footnote{\emph{The Matrix} (1999), \emph{The Matrix Reloaded} (2003), \emph{The Matrix Revolutions} (2003), por Larry and Andy Wachowsky.}, \emph{Cypher}\footnote{Vicenzo Natali, \emph{Cypher} (2002)}, \emph{Pi}\footnote{Darren Aronofsky, \emph{Pi} (1998)}, desenvolvem essa mesma idéia, sem contar com as inumeráveis produções de segundo plano, cuja lista não pode ser exaustiva\footnote{\emph{Wargames}, de John Badham (1983); \emph{Hackers}, de Iain Softley (1995); \emph{Pay Check}, de John Woo (2003); \emph{Live Free or die hard}, de Len Wiseman (2007), entre outros.}.

Este termo, hacker, é quase sempre apresentado como derivado do termo inglês \emph{hijacker}, pirata. Com menos frequência, sublinha-se a sua relação etimológica com o verbo cortar nas línguas germânicas, e também em inglês. Enquanto a etimologia inglesa faz referência ao sentido muito midiático de bandido explorando sistemas informacionais e redes de computadores (\emph{Cracker}), a etimologia germânica nos abre a um horizonte mais amplo de significados. De fato, trata-se então de "cortar" os caminhos já conhecidos para achar dicas, soluções, hipóteses para um problema que sejam originais, não óbvios e elegantes. É nesse sentido que o termo \emph{hack} foi usado pela primeira vez, dentro dos laboratórios do clube do \emph{Tech Model RailRoad} (TMRC) do MIT na década de 50. Assim, chamava-se de "\emph{hack}" as modificações inteligentes que faziam nos relês eletrônicos. Também, na década de 60, fazia-se referência ao ato de hackear (\emph{hacking}) dentro do laboratório de inteligência artificial da mesma universidade, onde foi fundado o projeto GNU. Apontava então o fato de ir procurar o código fonte para melhorar e adaptar um software para seus próprios fins. Nesse sentido, hackear relaciona-se particularmente com o movimento do software livre. As empresas brasileiras de informática como Google Brasil, Microsoft Brasil, IBM Brasil, sempre fizeram referência ao \emph{Decifrador}, em respeito à língua portuguesa. 

\subsubsection{Geek e Nerd} \label{2.1.1.b}

Geek é uma "gíria" inglês derivada da palavra \emph{geck}, idiota (\emph{fool}) e estranho/marginal (\emph{freak}). Também o termo geek tem origens com a palavra \emph{geck} do germânico antigo, significando louco, e em dialetos franceses, \emph{Gicque}, significando "louco de carnaval"\footnote{Fonte : Oxford American Dictionary}. Trata-se de “uma pessoa esquisita, ou pelo menos estranha, particularmente uma sendo percebida por ser demais obsessiva com uma ou várias coisas dentro das quais a intelectualidade, a eletrônica”\footnote{Dictionary.com}.  Podemos qualificar a cultura \emph{geek} como movimento cultural popular típico da sociedade de informação contemporânea. Além disso, os geeks são qualificados como tendo interesses exaustivos e criativos sobre assuntos relacionados como novas tecnologias (KONZACH, 2006). Por isso o software livre pode ser considerado como uma área da cultura \emph{geek}, e as pessoas participando por volta dele como \emph{geeks}. Assim, o ato de programar encaixa-se perfeitamente nas exigências de exaustividade, de criatividade e de mudança presentes na cultura \emph{Geek}. 

O antropólogo Christopher Kelty, no seu estudo sobre as significações culturais do \emph{free software}, faz constantemente referência aos \emph{geeks} como sendo os atuantes principais do movimento (KELTY, 2008), e para ele, trata-se de uma convergência filosófica pelo lado do transumanismo. Este transumanismo é uma modalidade de crença na linha do tempo do progresso técnico, a qual permitiria ir além do corpo humano. Para os transumanistas, a intervenção técnica tem um papel específico que a diferencia da intervenção política, legal, cultural ou social. Nesse âmbito, não tem retórica, mas uma hipotética “verdade pura” do “funciona!”. No extremo e para usar de uma metáfora própria ao mundo da programação: “as idéias ruins não se compilarão\footnote{Um programa é  “compilado” quando uma serie de comandos e instruções se tornam um aplicativo executável.}”.

Embora tivesse uma conotação de “bizarro” por volta do termo \emph{geek}, parece que a conotação pejorativa fica mais na palavra \emph{Nerd}. De fato \emph{Nerd} designa uma pessoa de alta capacidade intelectual e tendo interesses fora dos padrões sociais comuns. Assim apontam-se mais os significados de desvio, estranho, fora do padrão. No sentido comum, parece que a figura do \emph{geek} constrói-se no âmbito de um especialista autodidata, enquanto a do \emph{nerd} desenha a de um adolescente sarapintado.  Na verdade, essas diferenciações aparecem quando se procura definir categorias, pois a guerra \emph{Geek-Nerd} não tem propósito por seus atores, e cada um pode chamar o outro sem ter uma identidade muito diferente. O termo pode ainda ser usado da mesma forma que \emph{geek} para designar atores do software livre, como o \emph{Nerd} é o ideal-tipo do hacker usado por Paul Graham em \emph{Hackers and Painters} (GRAHAM, 2004), por exemplo.

\subsubsection{Usuários-desenvolvedores} \label{2.1.1.c}

Por fim o termo de usuário-desenvolvedor é o único dos termos revistos aqui que é específico às comunidades de software livre. Designa as pessoas que por volta dessas comunidades usam softwares livres, e por esse uso chegam a ajudar ao desenvolvimento do próprio software. Essa ajuda pode ser mínima, ou quase simbólica, aceitando mandar “\emph{crash report}”\footnote{Relatórios de mau-funcionamento} ou de responder a questionários. Ademais, pode tratar-se de tarefas muito maiores como o estabelecimento de documentação, programação de módulos adicionais, desenvolvimento do próprio software.

A figura do usuário-desenvolvedor é muito presente nos projetos livres, eles constituem a “comunidade” que define esse modelo de desenvolvimento como colaborativo. E é provavelmente na reprodução dessa figura em outros domínios de produções que se percebe a influência que pode ter o movimento do software livre fora de suas fronteiras. Isto é, os padrões de “consumidor-produtor” da economia colaborativa (Wikinomia) e de “leitor-redator” da produção de conteúdo na Web 2.0 que são intimamente ligadas ao imaginário do software livre e a figura do usuário-desenvolvedor (TAPSCOTT e WILLIAMS, 2006).

\subsection{A construção social da ética} \label{2.1.2}

Uma das análises bastante desenvolvidas a respeito dos aspectos culturais do movimento do software livre e da cultura hacker é a da ética. Assim, a “ética hacker” é um paradigma suficientemente flexível para integrar a multidão de casos que formam as comunidades por volta de cada software. Uma demonstração disso é que se hacker e usuário-desenvolvedor não se alinham exatamente nas mesmas realidades, a cultura hacker é um dos pontos comuns a todas as comunidades de usuários-desenvolvedores.

Para trazer ao conceito de ética a flexibilidade desejada, deve-se afastar a concepção kantiana da ética, segundo a qual a lei moral vem de nenhuma forma da experiência empírica, mas sim de imperativos categóricos, pré-estabelecidos: “Age de tal modo que a máxima da tua ação se possa tornar princípio de uma legislação universal”. Trata-se para nós de tomar conta da “vida social da ética” no sentido dos teóricos sociais como Bakhtin (BAKHTIN, 1984, 1993). Assim, as regras de vida e de convivência constroem-se interativamente e dinamicamente a partir das práticas sociais e das experiências vividas no quadro comunitário. Desta maneira, a ética pode se manter somente observada a uma relativa estabilidade das práticas sociais e uma comunidade de sentido por volta das experiências vividas. A análise se foca, então, sobre eventos que geram transformações, e desse modo, aponta-se a importância da auto-modelação comunitária, ou seja, do papel decisivo do contexto social, histórico e econômico. 

Nessa perspectiva, a idéia é simples e desenvolvida por numerosos teóricos (GALISON, 1997; GOOD, 1994; GUTERSON, 1996), pois se trata de excluir os excessos de particularismos como os de universalismos e ficar focado na observação dos acontecimentos concretos. Assim, os indivíduos adotam valores e realizam escolhas morais através de ações que são influenciadas por instituições ou tecnologias. É isso que podemos aprender das palavras de Coleman e Hill: “As experiências sociais que vêm a favor ou contra uma prática ética, modelam a natureza das relações individuais e determinam as orientações éticas de uma comunidade” (COLEMAN e HILL, 2005). Em outras palavras, o uso e a organização dessa ética interagem com o meio da comunidade ou do grupo em escadas econômicas, políticas e sociais. Em nível econômico, pode se tratar da produção de um bem, de um modelo de desenvolvimento ou de organização. Já no político, pode ser sua agenda política e suas manifestações públicas. Por fim, no âmbito social, uma rede e nós de indivíduos e grupos estendendo-se e traduzindo suas normas em contato com novos domínios e grupos. Por nos permitirmos fazer o laço entre o modelo de desenvolvimento participativo, alguns crimes eletrônicos, as conferências internacionais sobre o Open-Source, Internet, etc, a "ética hacker" é uma noção heurística ao entendimento do fenômeno do software livre.

\subsection{O conceito de ética hacker} \label{2.1.3}

Contam em literaturas que a expressão "ética hacker" foi usada pela primeira vez pelo jornalista Steven Levy, em uma das primeiras obras reconhecida sobre as cybers-comunidades: \emph{Hackers, Heroes of the Computer Revolution} (LEVY, 1984). Em contraste com as definições usadas logo após, Levy não hesita em dar ênfase a seu conteúdo. Segundo ele, a ética hacker é um tipo de predecessor moral ao software livre e às comunidades open-source. De acordo com essa interpretação da ética do \emph{hack}, o acesso a informação é total. Qualquer acesso a um conhecimento faz descobrir uma parte do mundo, e por isso deveria ser absoluto, sem limites. Por isso, a informação e sua circulação devem ser livres. Para livrar-se dos constrangimentos institucionais, precisaria desafiar as autoridades estabelecidas e ficar desconfiado na frente de legitimidades exteriores às comunidades. Nesse sentido precisa privilegiar formas descentralizadas de organização e, nesse ambiente, os hackers devem ser apreciados em função de suas qualidades técnicas e criativas e não por qualquer critério social (raça, idade, sexo). Por fim, tem-se a idéia que o fortalecimento técnico e prático das tecnologias informáticas contribuirá para um "mundo melhor", uma melhor organização da sociedade, mais repartida, justa, democrática.

À propósito dos "verdadeiros hackers" que Steven Levy queria identificar com essa definição, achamos: John McCarthy, Bill Cosper, Richard Greenblatt, Richard Stallman. Depois o autor descreve uma "segunda geração’" de hackers, os \emph{hardware hackers}.  Tratam-se dos grandes nomes da computação pessoal cuja maioria está envolvido no início do Vale do Silicone: Steve Wozniack, Steve Jobs, Bill Gates, Bob Marsh, Fred Moore. Por fim, uma terceira geração aparece na época dos primeiros jogos interativos, os "games hackers" (John Harris, Ken Williams, entre outros).

Todavia, concentrando-se sobre as produções ganhadoras das comunidades de informáticas (Software, Hardware, Jogos), o autor de \emph{Hackers} parece passar ao lado do fenômeno social dos agrupamentos de hackers como um todo. De fato, mesmo em 1984 - data da publicação do livro - as práticas hackers ultrapassavam já muito os quadros industriais e contribuíam à construção e à publicação de uma quantidade imensa de informações que ia permitir a qualquer um, a seu nível, de participar/aproveitar desse progresso técnico, da era da informação e das redes mundiais. No caso do software livre, atribuir a Richard Stallman ou a Linus Torvalds a paternidade do sistema GNU/Linux não é um erro somente político de esquecimento das massas de engenheiros que contribuíram ao projeto, mas também afasta o observador ou o leitor dos interesses, laços e valores que motivou milhares de hackers do mundo \emph{online} a trabalhar juntos. Isso representa de aceitar como material de análise e de testemunha dessa época os \emph{mailing-lists}, fóruns, \emph{wikis}, manifestos, debates, fluxos de notícias que fizeram a cultura hacker e estruturaram sua ética.

Desse modo, o filósofo finlandês Pekka Himanem desenvolveu uma noção de ética hacker diferente. Ao fazer referência aos trabalhos de Weber sobre a ética protestante e o capitalismo moderno, ele opõe os valores hackers aos valores protestantes. Segundo ele, a ética hacker aproxima-se mais das definições da "atividade" como se pode achá-la a obra de Platão ou Aristóteles: um trabalho de excelência por volta de uma procura de paixão, de felicidade e de criatividade (HIMANEM, 2001).

As an\'alises de Himanem como as do Levy, todavia, parecem se cristalizar por volta de um contrapeso aos desenvolvimentos pejorativos da imagem do pirata informático, e nesse sentido, achamos trabalhos ainda mais explícitos tentando "resgatar" uma fama ridiculizada (BEST, 2003; HANEMYR, 1999). De modo geral, as concepções do hacker e da hacking as quais esses trabalhos tentam se opor são, por um lado, adolescentes obcecados pela internet e pela conquista de saberes perigosos e/ou proibidos (BORSOOK, 2001; SLATALLA e QUINTER, 1996), e por outro, torneios audaciosos de intrusão em sistemas privados (SCHWARTAU, 2000). De fato, parece que os estudos até então citados se entregam a interagir com os preconceitos existentes a respeito da figura do hacker e de seu papel na sociedade da informação, sem nunca se perguntar o que faz a particularidade em si desse grupo. Nesse sentido, a antropóloga Gabriella Coleman, cujos trabalhos vão fazer referência ao longo do nosso estudo, chegou a afirmar: “A literatura sobre Hackers, pois, tem tendência a encerrar o ato de hackear num binário moral no qual hackers são ou louvados, ou desprezados. Esse tendência ameaça isolar mas do que esclarecer a significação cultural do computer hacking.”\footnote{“The literature  on hackers, thus, tens to collapse hacking into a moral binary in which hackers are either lauded or denounced. This tendency threatens to obscure more than it reveals about cultural significance of computer hacking.”} (COLEMAN, 2008, p.256).

\section{As significações culturais do hacking} \label{2.2}

\begin{quote}
“Voltada a sua expressão a mais simples – e abstração feita de seu conteúdo – a ética hacker tem seu equivalente na formula l’art pour l’art (a arte pela arte). O importante é de entender que ao contrario do ativismo político, o objeto da atividade do hacker, o conhecimento e o exercício da curiosidade, é interior ao sujeito.”\footnote{“Ramenée à sa plus simple expression, - et abstraction faite de son contenu – l'éthique hacker a son equivalent dans la formule 'l'art pour l'art'. L'important içi est de saisir que, contrairement à l'activisme politique, l'objet de l'activité hacker, la connaissance et l'exercice de la curiosité, est intérieure à son sujet”}
\begin{flushright}
REIMENS, 2002
\end{flushright}
\end{quote}

\subsection{O agnosticismo político do Movimento do Software Livre e da figura do Hacker} \label{2.2.1}

É indubitável que numerosos projetos foram inspirados pela filosofia do “livre”. Na área do jornalismo, disponibilizaram-se arquivos inteiros (ex: BBC) e desenvolveu-se a figura de um leitor participando – leitor-redator – à construção e discussão dos fluxos informativos, especificamente nas tendências da web 2.0. Na área da lei, acham-se formas contratuais novas para regimentar as produções intelectuais e artísticas inspiradas pelas licenças livres (ex: \emph{Creative Commmons}). Na educação, cursos são disponibilizados \emph{online} como é o caso no MIT com o OpenCourseWare, e isso por cima de tecnologias livres. Esses projetos são todos ligados a um contexto de rede e de interconexão oferecido pela internet e a sociedade de informação. Mas como o afirma Christopher Kelty, as significações culturais do software livre vão além do simples diagnóstico da “sociedade da informação”, pois provém de uma reorientação mais específica, portanto tem a ver com práticas técnicas e legais detalhadas e precisas. Também se trata de uma reorientação mais geral porque cultural e não somente econômica ou legal. Uma prova disso é que com a Internet, a governança e o controle da criação e da disseminação do saber mudou consideravelmente, operando assim uma “reorientação do poder e do conhecimento” que questionam profundamente a relevância e a legitimidade do sistema de propriedade intelectual (KELTY, 2008).

Nesse contexto, embora seja claro o fato de que a vida política do SL/CA, por suas atuações e influencias, já é avançada, a figura do hacker e dos atuantes do movimento continua recusar qualquer afiliação política. É isso que nota Gabriella Coleman ao dizer: “Enquanto é perfeitamente aceitável e incentivado de ter uma mesa sobre software livre numa conferência antiglobalização, desenvolvedores de SL/CA recomendam que seja inaceitável de reivindicar que o SL/CA tem como propósito a antiglobalização, ou a respeito disso, qualquer programa político.”\footnote{“While it is perfectly acceptable and encouraged to have a panel on free software at an anti-globalization conference, FOSS developers would suggest that it is unacceptable to claim that FOSS has as one of its goals anti-globalization, or for that matter any political program.”}(COLEMAN, 2004, p.507).

Cabe aqui pensar que esse “agnosticismo político” pode exprimir-se através de varias defesas contra as tentativas de dar um sentido político a um ato, uma produção ou uma situação. Na área de segurança informática, o critério principal é a perfeição do sistema de proteção, e por isso as considerações de ordem ideológicas são afastadas com a observação de que a tecnologia a ser usada pode ser livre ou não, enquanto for aberta a atualizações rápidas e com desempenho estável. Cabe aqui acrescentar o que um participante do canal \emph{\#hack} do servidor IRC Freenode comentou durante um debate sobre o \emph{free software}: “não é bem 'tanto faz', assim... se as ferramentas forem livres, melhor, que a gente trabalhe com sistemas abertos [freeBSD, Linux], mas o importante é ter a tecnologia mais segura, mais configurável, mais controlável. É isso que faz de uma tecnologia que ela seja boa para um trabalho de segurança, e não o fato dela ser livre ou não.”\footnote{\#hack@irc.freenode.net, 10 novembro 2008}. Assim o site \emph{insecure.org} que mantém uma famosa lista de 10 melhores ferramentas de segurança informática\footnote{http://sectools.org/}, sequer comenta o tipo de licença que acompanha os softwares. Ademais, os critérios privilegiados são o preço (gratuito ou não), a compatibilidade com diversos sistemas (Linux, Windows, OSX e BSDs) e o acesso ao código fonte. O software que chega em primeiro dessa lista desde anos, Nessus, uma ferramenta para identificar falhas de segurança, é de código fechado e pago.

Em atuações mais diversificadas da computação, incluindo administração de redes, análise de sistemas de informações, programação web, o recuso da ideologia manifesta-se muito ao encontro da figura do responsável de projeto, cujas capacidades técnicas são frequentemente reduzidas, e aos efeitos de moda pelos quais eles passam. Ao longo do nosso trabalho, encontramos numerosos testemunhos nesse sentido, particularmente na área de desenvolvimento web, onde os programadores convivem com superiores voltados ao marketing e à comunicação publicitária. A ideologia aparece, então, como um meio para o ignorante da técnica de se familiarizar com as limitações e do executante da tarefa. Assim, ele constrói um discurso permitindo justificar as opções tecnológicas, de orçamentos, de prazos, entre outros. Chega-se, desta forma, numa situação onde o responsável de projetos, muitas vezes numa situação hierárquica ascendente, vai realizar escolhas a partir de informações truncadas que os técnicos vão seguir sem poderem se preocupar em saber quais são as melhores opções a seguir para garantir o melhor serviço. Podemos identificar isso na testemunha de Marcelo, administrador de rede em uma universidade publica brasileira, ilustra:

\begin{quote}
Devíamos programar o troço para a SOFTEX\footnote{Associação para Promoção da Excelência do Software Brasileiro - SOFTEX}... Ai reunião no escritório com o coordenador do projeto acompanhado de seu estudante que “manja de informática”. A idéia é clara, simples, não tem problema até o cara falar para gente que deve programar o negócio em Java porque Java é a super linguagem que vai ganhar de todos os outros, porque tem a 'máquina virtual' super mega adaptável, blábláblá... Todos acreditam na moda, dá a eles ares de experto... Então... Eu respondo que Java não é a melhor solução para esse projeto... E ai começa a máquina a propaganda TI, e para dar um suporte, o estudantezinho “que manja de informática” apoia todas, Java funciona em 100\% dos casos... Ai, desafio o estudante de me dizer se é possível de programar tal coisa em Java... Eles pensam, tentam mudar de assunto, etc, etc e ai conclusão brilhante do gordo: então se não for 100\%, é 99\%...
\end{quote}

Ao contar esse evento, o informante dá um exemplo de situação aonde a ideologia (“a propaganda TI”) vem prejudicar o trabalho dele e obrigá-lo a realizar uma tarefa com opções tecnológicas que não são as melhores, isso por razões que parecem “absurdas”, ou seja, “crenças” de alguém com posição hierárquica superior e com um conhecimento técnico incompleto e influenciado por efeitos de moda. Aqui “o cara da SOFTEX” deve provavelmente basear sua escolha da linguagem de programação JAVA sobre tendências internas a sua instituição de promoção do Software Brasileiro (“o JAVA é entendido como o ‘melhor’, então um software feito com JAVA ‘é’ melhor”) desligando-se do raciocínio lógico voltado ao desempenho da tecnologia.

Embora esse exemplo ilustre uma polêmica técnica que não pode ser chamada diretamente de “política”, parece que é a mesma lógica que determina a consciência política desses atuantes. Assim é que a figura do responsável permanece nas lógicas da sua instituição favorecendo seus interesses próprios ou pelo menos interesses que não são os dos profissionais. Outra questão colocada é que a idéia de exigência de excelência técnica cria um raciocínio que protege os atores do político, nesse sentido, o movimento SL/CA não pretende ganhar uma decisão política no seu favor ou conquistar políticas públicas, mas continuar evoluindo com uma autonomia própria permitindo o melhor desenvolvimento técnico.

Em nosso entendimento, essas observações permitem conhecer melhor o fato que ambas as literaturas, anarquistas e liberais, fazem referência ao movimento do software livre como exemplar do seu modo de produção e de organização. Por um lado, as análises liberais acham aqui um exemplo concreto de “Mão invisível”, de ausência de intervenção do estado ou de qualquer outra autoridade externa nos processos de decisão internos. Os atores ainda desenvolvem uma consciência própria do ambiente deles por um processo livre e conseguem tomar decisões, para fazer escolhas voltadas a seus interesses próprios que, juntos, criam e mantém a comunidade do software livre. Somado a isso, temos pensadores que acham realizações concretas de formas anarquistas de organizações no movimento SL/CA e suas comunidades (MOGLEN, 1999, GROSS, 2007). Sublinham-se, então, a alta produtividade real da autogestão de trabalhadores, e as raízes lógicas do desafio às autoridades. Assim, mantendo uma possível alternativa ao neoliberalismo e ao anarquismo, como uma reformulação deles, o agnosticismo político do movimento SL/CA e da figura do hacker protege sua exigência de excelência técnica de direções políticas predeterminadas.

\subsection{A informação como paixão e a tipologia das políticas informais} \label{2.2.2}

Tentando achar a essência do hacking, ou seja, um ponto fixo de análise razoavelmente independente dos preconceitos dos sentidos comuns, a antropóloga Gabriella Coleman vai focar-se sobre a relação que mantém os hackers com a informação. Assim, uma semelhança de todos os usuários-desenvolvedores, hackers e \emph{nerds} é o amor dado à liberdade de informação e sua livre circulação (COLEMAN, 2003). Trata-se de um sentimento muito forte, maníaco e estético próximo ao \emph{amour fou} (amor louco) descrito nas obras do anarquismo ontológico, como o meio de realizar-se inteiramente como indivíduo e desafiar as estruturas das sociedades (BEY, 1985 e 1991). Segundo as palavras de Bruce Sterling, os hackers são “possuídos não meramente por curiosidade, mas por uma verdadeira luxúria do saber.”\footnote{Citado em COLEMAN, 2003.}. Tentando achar as origens de tal paixão, Gabriella Coleman compartilha sua experiência:

\begin{quote}
Minha experiência com o software livre defende esse principio. O espírito de exploração que forma as bases do hacking deve começar em desmontar a mistura familiar, pelo horror da mãe: depois traz a aprender a programar com cinco anos, pela alegria da unidade parental, depois se transforme em se trancar no quarto para ler todos os manuais de computação, os quais os pais confundem com inquietude pré-adolescente; depois é aprender cada entrada e saída desse sistema operacional simples chamado de Unix, descobrindo cada um dos traços tipográficos e temporais da Internet, pelo espanto do antropólogo; e finalmente passar seu tempo contribuindo, escrevendo código para projetos de códigos abertos, muitas vezes, de novo, pela consternação de seus pais.\footnote{“My experience with free software support this fundamental tenet. The spirit of exploration that forms the basis of hacking might start by taking apart a household blender, much to a mother’s horror: then lead to learning how to programme at the age of 5, much to the delight of the parental unit; then transform into locking oneself in the bedroom to read every computer manual, wich parents duly confuse with pré-teen angst ; then learning every last topographical and temporal feature of the Net much to the amazment of the anthropologist ; and finally to volunteering their time to code on fre software projects, often to the dismay, again, of their parents.”}(COLEMAN, 2003, p.298)
\end{quote}

Depois de toda esta discussão, vale fazer referência a nossa experiência dentro de um departamento de administração de redes e sistemas de uma universidade pública brasileira, como o contato com numerosos profissionais e usuários desenvolvedores de vários domínios da informática. Isso nos permite afirmar que o elemento de paixão pela informação é determinante na diferenciação do "exército de reserva" dos profissionais das TI, interessados pela aquisição de um \emph{know-how} e de seu valor direto no mercado do trabalho, e os que, em cima dessas preocupações materiais, desenvolvem um interesse constante para a apropriação, a produção e a discussão da informação. Dentro desse departamento de informática, a divisão das tarefas parecia ser mais determinada pela capacidade de cada um “brincar” na frente de um problema que por formação ou cargo oficial. O termo de “brincar”, usado pelos próprios atuantes, refere-se principalmente às aptidões dos funcionários a responsabilizarem-se, ou seja, a enxergar os ramos do problema enfrentado com outros possíveis erros e tentar resolvê-los do jeito o mais sistemático, e de surpreender-se, ou seja, jogar com os paradigmas utilizados naquele processo e cruzá-los com considerações técnicas e teóricas para melhorar sua concepção geral do sistema. Desta forma, a curiosidade e o interesse pessoal para o funcionamento das tecnologias utilizadas parecem ser o critério determinante para julgar da qualidade de um funcionário. Um informante ilustra:

\begin{quote}
Cada vez que um novo estagiário entra [no departamento de informática] eu falo o seguinte: a Universidade não vai te aprender um trabalho mas te dar uma formação geral... Você vai aprender as coisas colocando as mãos na massa, com problemas técnicos, concretos, procurando soluções, jeitinhos simples e eficazes. Mais você fica procurando as razões atrás do problema, ou seja, as estruturas do sistema, como funciona uma rede, um protocolo, como dialogam dois computadores, mais você vai criando concertos expertos, inteligentes, que vão ficar, em cima dos quais a gente vai trabalhar e desenvolver novas coisas. Isso, poucos entendem, nhé? A última chamada para estagiário teve mais de 15 no meu escritório e nenhum deles soube me dizer o que é um banco de dado... Acredita?
\end{quote}

Ao entender essa característica do hacker, seja técnico de laboratório ou usuário-desenvolvedor num projeto livre, permite enxergar o que define mais essa figura é sua procura “absurda” para uma informação livre, abundante e de precisão, e é a partir disso que se constrói a interação do atuante com seu contexto social. O usuário-desenvolvedor quer uma informação livre capaz de circular e de ser modificada, mas isso acontece em um ambiente de micro-restrições que limitam sua atividade. Portanto, falamos no trato de restrições técnicas, legais, culturais, políticas, econômicas, que perturbam uma atividade subjetiva de auto-realização. A respeito do software livre, as restrições mais óbvias são as legais, ou seja, dentro do código, aquele que se pode abrir e modificar, e aquele que somente pode executar. Dentre outras opções, a engenharia reversa permite obter parte de um código fechado, porém ultrapassam-se os limites dos direitos autorais, como o caso de usos de inovações livres para melhorar tecnologias fechadas, por exemplo, isso ilustra o fato que a fronteira entre software livre e pirataria não é tão nítida que ambos dos atores – livre e proprietário – o deixam entender. Além de ter numerosos protocolos e ferramentas de pirataria com licença livres (Bittorrent, \emph{Peer-to-peer}) a própria raiz do movimento SL/CA foi de certa forma o \emph{hacking} do sistema UNIX da AT\&T (cf. \ref{1.2.2}). Essa confusão pode até ser explorada pelas próprias companhias de software proprietários, e com essa preocupação, Túlio Vianna observou que os relatórios americanos sobre a pirataria mundial, principalmente da \emph{United States Trade Representative} (USTR), juntam os \emph{marketshares} do software livre com os da pirataria para denunciar os crimes contra os direitos autorais (VIANNA, 2006). Mas o contexto de restrições que encontra o \emph{geek} na sua atividade de busca vai bem além das considerações legais que ficam as mais mediáticas, porque há a criação de vários genéricos concretos de interação com a sociedade e suas estruturas. Nesse sentido, Coleman identificou vários tipos ideais de lógicas especificas às éticas da informação diferenciadas, é o que pontuaremos a seguir:

\subsubsection{As políticas de transgressões: o \emph{underground hacking}} \label{2.2.2.a}

A parte transgressora do mundo do \emph{hacking} é seu rosto fantasiado e mediático, e também desprezado. Em todos os casos, ela fica quantitativamente, em termos de produção de código, de participantes e de instituição, minoritária. O \emph{hacking underground} ilustrou-se particularmente na formação dos conceitos de engenharia social e de \emph{human data} (dados humanos) onde a sistematização do comportamento humano e sua piratagem expressam-se através de um cinismo explícito (MITNICK e SIMON, 2002). Isso significa uma crítica forte do liberalismo associada à noção Nietzschiana do poder e do prazer. Mas “como a tentativa de Nietzsche de elevar o poder criativo do individuo nunca conseguiu verdadeiramente escapar-se das noções liberais do iluminismo, a prática do \emph{hacking underground} representa mais uma radicalização dos fundamentos do liberalismo do que um verdadeiro cisma” (COLEMAN e GOLUB, 2008, p.263). Isso acontece devido a uma ética da transgressão como crítica política que se manifesta pelo culto “prazer de ser vigiado” e da “interface entre o vigiamento e escapamento ao vigiamento” (HEBDIGE, 1997). A informação é “boa”, “agradável” se ela é proibida e que sua aquisição necessita transgressão.

\subsubsection{As políticas de tecnologias: a criptoliberdade} \label{2.2.2.b}

A criptografia é a cifragem de dados por um algoritmo reversível por meio de uma chave de dados secundária (decifragem). Essa tecnologia é usada com fins de confidencialidade, de autenticação e de controle de integralidade. Na área da informática, a primeira chave pública foi publicada em 1975 por Whitefiels Diffie e Martin Hellman (MIT), que revolucionaram então o domínio da criptografia. O uso da criptografia estende-se, então, às instituições, às empresas, por exemplo, mas não existem ainda soluções para os computadores pessoais.

Em 1991, enquanto o senado americano ia votar uma lei para proibir o uso privado da criptografia, Phil Zimmerman publica a primeira chave publica utilizável num computador pessoal (PGP – \emph{Pretty Good Privacy}), tornando-se então culpado de um ato de desobediência civil e arriscando acusações por traição. Ainda, o autor do programa desenvolve assim todo um discurso durante sua defesa na cena mediática, vejamos:

\begin{quote}
Se a privacidade é fora-da-lei, somente os criminosos terão direito a privacidade: […] PGP permite às pessoas de encarregar-se da sua privacidade. Há uma carência social crescente para isso, é por isso que eu o escrevi.
\begin{flushright}
Phil Zimmerman
\end{flushright}
\end{quote}

Como observamos anteriormente, isso é o início da ética criptolibertária, que ficou depois fortalecida pela criação em 1992, dos Cypherpunks, associação de hackers militantes para os direitos civis. Eles trabalham em cima das tecnologias de privacidade e militam contra as leis limitando a privacidade individual, o que quer dizer que valores liberais, no plano político, sustentam esse movimento, notavelmente as de autonomia do indivíduo e de sua liberdade a respeito do governo, pois se cria a expectativa que as tecnologias podem resolver problemas sociais, porque reformulam na nova linguagem tecnológica a repulsão liberal a intervenção do estado. Concretamente, esse movimento junta anarcho-capitalista radicais, democratas, republicanos, e, para alguns deles, nada há de novo nesse movimento, isto é, somente a continuação de uma luta para o direito constitucional.

Ademais, a área da criptografia é um bom exemplo de influência das comunidades livres nas estruturas da sociedade, tornando difícil a interdição do uso da tecnologia disponibilizando-a. Assim, alguns países como a França, onde a criptografia foi proibida de principio, chegaram a autorizá-la (janeiro 1999) por necessidades provocadas pelas realizações concretas das comunidades (uso livre, necessidade social).

\subsubsection{As políticas de inversão: o movimento do software livre} \label{2.2.2.c}

Paralelamente ao movimento criptolibertário, outros hackers desenvolvem outra visão da segurança. Para Richard Stallman, fundador da FSF, o conhecimento de maneira geral, não deve ser objeto de qualquer orientação porque o benefício tirado de uma informação é sempre feito no detrimento da comunidade. Stallman militava dentro dos escritórios do MIT, deixando sua máquina sem nenhuma senha para os arquivos a serem disponibilizados para todos com uma mensagem de boas vindas explicando sua filosofia da informação. Quando a FSF foi criada em 1984, desenvolveu-se uma pedagogia que abriu o mundo do hacker para fora. Suas ações respeitavam a lei e se servia dela para proteger-se. A criação da licença pública (GPL) delimitou uma zona legal de segurança, de publicidade onde códigos ficariam abertos. \'E um conceito de liberdade positiva que é colocado na frente, de liberdade pela abertura, e não uma liberdade negativa, pelo fechamento ou o segredo como é o caso com a criptografia.

Essa idéia de inversão acha-se também no artigo \emph{Beating Them at their Own Game} (ganhar ao jogo deles) (BEST, 2003) onde Kirsty Best demonstra que o SL/CA representa para seus usuários mais um investimento do que um rejeito das estruturas sociais existentes nos arredores das novas tecnologias. Não se trata de promover uma mudança radical, mas de “entrar no jogo” das estruturas capitalistas para redefinir seus termos a respeito das tecnologias da informação e dos novos meios de comunicação.

\subsubsection{As políticas de colaboração: o movimento \emph{open-source}} \label{2.2.2.d}

Esse tipo ideal foi adicionado em nosso trabalho à tipologia da Gabriella Coleman, com o objetivo de marcar a diferença entre a ética \emph{open-source} e a “livre”, e para sublinhar a convergência de várias éticas dentro dessa.

Desse modo, associado por “coincidência” (TORVALDS e DIAMONDS, 2001) ao projeto GNU, Linux e seu projeto tecnológico vão favorecer a variação “open-source” do movimento do “livre” chamado a se radicalizar. O Software de código aberto, promovido antes de tudo como modelo de desenvolvimento não é mais somente “bom”, mas principalmente eficiente. A liberdade de informação libera um \emph{entertainment}, um mérito que são motivações muito mais eficazes que um simples salário para incitar à participação a um espaço colaborativo produtivo. Essa ética hacker é muito desenvolvida na obra de Himanem, prefaciada pelo criador de Linux, onde o foco é feito sobre a flexibilização do trabalho, e considerado como \emph{hobby} e paixão antes de tudo, sem remuneração direta sistemática (HIMANEM, 2001). A ideologia transmitida aqui é mais econômica do que política, e, além disso, a liberdade de informação é necessária porque ela é estratégica. Para tanto, precisa-se favorecer a criação de valores e a liberdade de fazer um benefício dela. Nessa nova economia colaborativa, “inteligente”, a “wikinomia”\footnote{Wikinomia, nome dado à economia colaborativa a partir do nome da plataforma colaborativa wiki que por exemplo serve de base a enciclopedia online Wikipedia.}, “a capacidade de juntar os talentos de indivíduos e empresas dispersadas está se tornando a competência chave do dirigente e da empresa” (TAPSCOTT e WILLIAMS, 2007) e, por assim afirmado, quase o objeto do próprio hack. Na área do software, esse grupo ilustrou-se na construção da web 2.0 pelas ferramentas participativos onde os usuários doam conteúdo a plataforma, como é o caso da enciclopédia Wikipédia baseado na tecnologia Wiki, software livre desenvolvido por Ward Cunningham em 1995. Ainda, a ética com dominante colaborativa ilustra-se na esfera tecnológica com a codificação de algoritmos performáticos, visando criar estatísticas dedicadas à análise dos comportamentos dos usuários da internet. Isto é “programar a inteligência coletiva”\footnote{Titulo de um livro recente tendo feito referencia na area da programaçao do web collborativo : SEGERAN, Toby, \emph{Programming Collective Intelligence: Bulding Smart Web 2.0 applications}, O'Reilly, 2007.} e tratar as informações acumuladas pela observação dos usuários, para, por exemplo, sugerir “produtos assimilados”, “par perfeito”, etc. Se a programação da colaboração desenvolve sua própria ética da informação, suas ferramentas tecnológicos inspiram-se muito dos movimentos vistos precedentemente: Associa a engenharia social do \emph{hacking underground} e é visceralmente ligada ao livre acesso a informação, a fim de tratá-las, e a sua estrita confidencialidade, para garantir a privacidade dos usuários e impedir as contestações judiciárias de prejuízo a vida privada.

\begin{center}
*
\end{center}

Ao observar essa tipologia, há uma \emph{heteroglossia} (BAKHTIN, 1981, 1986), ou seja, uma variedade no mesmo código lingüístico, no qual se acha uma discussão incessante sobre a liberdade. O que constitui o discurso moral dos hackers, e o que diferencia as éticas deles, é a elaboração de um sentido por volta do que é a liberdade e o que significa ser livre. Essa diversidade dá um dinamismo às comunidades hackers, que passam então de uma variedade de discurso a outra, e mudam assim de registro de referências sem se preocupar muito com contradições de conteúdo, de estilo ou de efeito político. Ademais, a trajetória discursiva que cerca a colaboração nas comunidades está em negociação constante, Porém, ao longo dessas negociações, pontos fixos ficam. À defesa da liberdade de informação, que se junta a uma tradição liberal que encontra então uma nova visibilidade e um novo discurso heteróclito em harmonia com a ‘era digital’. Entretanto, afirma-se que, o que se junta com esses discursos é uma prática comum da programação. Ou seja, qualquer área tecnológica ou aura política desses tipos ideais designam comunidades que têm com semelhança, como atividade comum, o fato de escrever código. Nesse sentido, Coleman afirma que “a liberdade do software livre, enquanto é influenciada por sensibilidades liberais maiores, é fundamentalmente modelado pelas pragmáticas da programação e o contexto social do uso da internet.”\footnote{“The freedom of free software, while influenced by wider liberal sensibilities, is fundamentally shaped by pragmatics of programming and the social context of internet use.”} (COLEMAN, 2004, p.509).

\section{As pragmáticas da programação como abordagem para seguir a interação dos usuários desenvolvedores com a informação} \label{2.3}

Num esforço de definição do ato de programar, Andrew Goffey usa da definição de 'dizer' feita por Michel Foucault em \emph{A arqueologia do saber}: “\st{Dizer} [programar] é fazer algo – algo diferente de expressar o que alguém pensa, traduzir o que alguém sabe, e algo diferente de brincar com as estruturas da linguagem.”\footnote{“To \st{speak} [program] is to do something – something other than to express what one thinks, to translate what one knows, and something other than to play with the stucture of language.”} (GOFFEY, in FULLER, 2008, p.14). Ao observar os usuários desenvolvedores como programadores, uma dualidade parece dirigir a tarefa deles: por um lado trata-se de um ato muito preciso, um tratamento racional de uma situação racional num contexto racional. Porém, por outro, o discurso entre programadores sobre o objeto deles, a ontologia que o autor desenvolve sobre sua obra, traz muitas referências subjetivas, intersubjetivas, e próxima a expressão de uma natura artística e estética do código (\ref{2.3.1}). E observando e analisando essas naturezas artísticas e reguladoras do ato de programar, poderemos, então, expor nosso paradigma de análise, isto é, as pragmáticas da programação como vetor do agnosticismo do movimento SL/CA e com relação à informação (\ref{2.3.2}).

\subsection{A programação como arte e regulamentação} \label{2.3.1}

\subsubsection{A programação como arte e a estética do código} \label{2.3.1.a}

{\footnotesize
\begin{verse}
\begin{flushright}
"Poeta para poeta. Eu imagino você\\
na margem da linguagem, no inicio do verão\\
em Wolfeboro, New Hampshire, escrevendo código.\\
Você não tem a noção do tempo. Nem a noção dos minutos.\\
Eles não podem alcançar seu mundo,\\
seu computador cinza\\
com quando já agora jamais e uma vez.\\
Você tinha perdido os outros sete.\\
Este é o oitavo dia da criação.\\
\ldots \\
Estou escrevendo numa tela azul\\
como qualquer morro, como qualquer lago, compondo isto\\
para te mostrar como o mundo recomeça:\\
Uma palavra de cada vez.\\
De uma mulher para a outra"\footnote{Homenagem da poeta Eavan Boland à Sra. Grace Murray Hopper (1906-1988), programadora de dum compilador para a linguagem COBOL, durante os anos 40.“Poet to poet. I imagine you / at the edge of language, at the start of summer / in Wolfeboro, New Hampshire, Writing code. / You have no sense of time. No sense of minutes, even. / They cannot reach inside your world, / your grey workstation / with when yet now never and once. / You have missed the other seven. / This is the eighth day of creation.[…] I am writing at a screen as blue /as any hill, as any lake, composing this / to show you how the world begins again: / One word at a time. / One woman to another.”. BOLAND, Eavan, Code, 2001.} \\
\emph{Código}, BOLAND, 2001.
\end{flushright}
\end{verse}
}

\emph{A Arte da Programação} é o título de um conjunto de quatros volumes reconhecidos por ser uma das maiores obras didáticas da área de computação, e foi nomeada com uma das doze maiores monografias científicas pelo jornal \emph{American Scientist}. A redação desse livro iniciou-se em 1962 quando a editora Addison Wesley propus a Donald E. Knuth, então doutor em matemática de 24 anos e candidato a uma vaga de professor, de escrever um livro sobre compiladores. Até hoje o livro é atualizado com frequência pelo autor, e um quinto volume esta em preparação para 2015. Segundo Maurice J. Black, essa obra segue uma corrente de ensino e comentário dos programas e da programação comparável a um esforço de crítica literária. Antes de \emph{A Arte da Programação}, os \emph{Comentários de Lions sobre a sexta versão de Unix} de John Lions já acompanhavam os programadores no estudo do código fonte do sistema da AT\&T, então disponibilizado. Mais do que um estudo, tornava-se a lida de uma entidade estética cujos comentários revelariam os sentidos, particularidades e lógicas (BLACK, 2002). 

Além do fato que os escritos sobre código possam ser considerados como crítica literária, o código ele mesmo aparece como obra em si. Esse ponto de vista é perfeitamente encarnado pelas teorias da programação literária (\emph{Literate programming}) desenvolvida por Donald Knuth:

\begin{quote}
O praticante da programação literária pode ser olhado como um ensaísta, cuja principal preocupação é expor com uma excelência de estilo. Tal autor, com tesauro nas mãos, escolha os nomes das variáveis com cuidado e explica o que cada umas delas significa. Ele ou ela empenha-se para um programa que é compreensível porque seus conceitos foram introduzidos numa ordem que é a melhor para o entendimento humano, usando uma mistura de métodos formais e informais que gentilmente reforçam-se cada uns os outros.\footnote{“The practitioner of literate programming can be regarded as an essayist, whose main concern is with exposition and excellence of style. Such an author, with thesaurus in hand, chooses the names of variables carefully and explains what each variable means. He or she strives for a program that is comprehensible because its concepts have been introduced in an order that is best for human understanding, using a mixture of formal and informal methods that nicely reinforce each other”}
\begin{flushright}
KNUTH, 1983, p.1.
\end{flushright}
\end{quote}

Assim a analogia feita da programação com a literatura reformula uma metodologia do ato de escrever código. “Segundo Knuth, o melhor código não é escrito desde a perspectiva pragmática de um engenheiro, mas sim da perspectiva artística de um autor”\footnote{“For Knuth, the best code is written not from the pragmatic perspective of an engineer, but from the artistic perspective of an author. Economy of style, clarity of expression, and formal elegance are as essential to good programming as they are to good writing”} (BLACK, 2002). Mais radical ainda no sentido de considerar o ato de codificar como uma prática poética, o movimento de Poesia Perl (\emph{Perl Poetry}) usa a linguagem de programação Perl, caracterizada por o uso de funções designada em termos ingleses “naturais”, para traduzir e escrever poemas compiláveis (Figura \ref{fig2.1}, p.\pageref{fig2.1}).

\begin{figure}[htb]
\caption{The Coming of Wisdom with Time (1910, 2000)} \label{fig2.1}
\begin{multicols}{2}

\begin{verse}
Though leaves are many, the root is one;\\
Through all the lying days of my youth\\
I swayed my leaves and flowers in the sun;\\
Now I may wither into the truth\\
\end{verse}
\begin{flushright}
William Butler Yeats, 1910
\end{flushright}

\columnbreak
{\scriptsize
\begin{verbatim}
while ($leaves > 1) {
        $root = 1;
}
foreach($lyingdays{'myyouth'}) {
        sway($leaves, $flowers);
}
while ($i > $truth) {
        $i--;
}
sub sway {
        my ($leaves, $flowers) = @_;
        die unless $^O =~ /sun/i;
}
\end{verbatim}
}\\
\begin{flushright}
Wayne Meyers, 2000.\\
\end{flushright}
{\scriptsize
\begin{verbatim}
perl The\ Coming\ of\ Wisdom\ with\ Time 
Died at The Coming of Wisdom with Time line 12.
\end{verbatim}
}
\end{multicols}
\end{figure}

A metáfora da arte não é estrangeira aos fatos de linguagens comuns dentro do meio da informática. Como notam Samir Chopra e Scott D. Dexter, a “beleza” do código é um assunto confortável para os engenheiros da computação e a linguagem para descrevê-lo tem muitos adjetivos emotivos. Um conjunto de instruções para o computador é bonito ou feio, limpo ou sujo, leve ou pesado, fantástico, impressionante, horrível, ilegível, etc (CHOPRA e DEXTER, 2007). Nesse sentido, um pedaço do código do kernel Linux de 1990 ilustra (Figura \ref{fig2.2}, p.\pageref{fig2.2}):

\begin{figure}[h]
\caption{Pedaço do Kernel Linux 0.11 (1990)} \label{fig2.2}
{\footnotesize
\begin{verbatim}
/*
* Yeah, yeah, it's ugly, but I cannot find how to do this correctly
* and this seems to work. If anybody has more info on the real-time
* clock I'd be interested. Most of this was trial and error, and some
* bios-listing reading. Urghh
*/
#define CMOS_READ(addr) ({ \
outb_p(0x80|addr,0x70); \
inb_p(0x71); \
})
(Linux Kernel 0.11 main.c )
[...]
#define outb(value,port) \
__asm__(“outb %%al,%%dx”::”a” (value),”d” (port))
#define inb(port) ({ \
unsigned char_v; \
__asm__ volatile (“inb %%dx,%%al”:” = a” (_v):”d” (pot)); \
_v; \
}]
(Linux Kernel 0.11 /include/asm/io.h )
\end{verbatim}
}
\end{figure}

Embora esteja funcionando corretamente, esse código é descrito como “feio” (\emph{ugly}) por seu próprio autor, por não ser a melhor solução ao problema encontrado, nesse caso sincronizar o horário do sistema operacional com o relógio físico do hardware. Poderia acreditar-se que se um código faz funcionar uma ferramenta e que seu resultado é o qual o usuário esta esperando, o código é válido, correto. Porém a relação do programador com seu código, com os outros programadores e com o objeto programado com um tudo – aqui o sistema operacional – exige uma elegância lógica que deve fazer sentido para seus atuantes. Esse código deve regular a interação do software com o hardware, do usuário com ambos desses, e dos desenvolvedores a vir com ele mesmo. Essas interações formam um objeto abstrato sujeito a muitas interpretações e construções diferentes, e como o artista extrai uma obra única do seu corpo ou de uma representação, pode-se considerar “o código como um tentativa da capturar a beleza de um objeto abstrato, o algoritmo.” (CHOPRA e DEXTER, 2007, p.77). Comparando com a citação que segue dum crítico literário tentando definir a poesia: “Então! A poesia seria o momento exato, a montagem exata que é obtida para que enfim o real que conseguiu ser delimitado, é restituído graças ao material.”\footnote{“Donc! La poésie serait le moment où le mot exact, l'agencement exact est obtenu pour que enfin le réel qui a pu être cerner, est restituer grâce au materiau !”, Fabrice Luchini em dedicace no Virgin Megastore em Paris, 17/11/2008. http://www.dailymotion.com/relevance/search/luchini/video/x7yrj5\_fabrice-luchini-en-rencontre-dedica\_creation}, um algoritmo por ser uma montagem única que delimitiu interações e por ser restituída graças a um programa, é, então, como a poesia, uma arte.

O ponto em entender o código como arte é de poder delimitar características do software livre própria a sua natura artística. Sendo que a estética diga algo sobre o objeto software e sua criação, sobre a relação entre o programador e seu artefato, tratar de software livre é delimitar as especificidades de um código aberto nessas interações. O principal argumento das comunidades FOSS nesse sentido é o fato de que, como o artista é influenciado por seu contexto de produção, pelas obras com as quais ele interage e se inspira, o modelo de desenvolvimento colaborativo característico do software livre permite a uma estética muito forte de se estruturar. Como um bom autor, deve-se ler bons escritos, um bom programador deve inspirar-se em bons códigos para realizar as exigências comunitárias de estética. Nesse sentido, e não sem ironia, Bill Gates afirma:

\begin{quote}
O melhor jeito de treinar [para ser um bom programador] é de escrever programas, e de estudar excelentes programas que outras pessoas têm escritos... Você deve querer ler o código das outras pessoas, e depois escrever o seu, e depois chamar os outros para revisar seu código. Você deve querer ficar nesse círculo incredível de retorno, onde você encontra pessoas ao nível mundial para lhe dizer o que você esta fazendo errado... Se um dia você fala com um bom programador, você descobrirá que ele conhece suas ferramentas como um artista conhece seus pinceis. E impressionante de ver até que ponto bons programadores tem em comum no jeito deles de desenvolver... Quando você traz essas pessoas a olhar um pedaço de código excelente, você observa uma reação muito, muito comun.\footnote{“The best way to prepare [to be a good programmer] is to write programs, and to study great programs that other people have written\ldots You've got to be willing to read other people's code, then write your own, then have other people review your code.  You've got to want to be in this incredible feedback loop where you get the world-class people to tell you what you're doing wrong\ldots If you ever talk to a great programmer you will find he knows his tools like an artist knows his paintbrushes. It's amazing to see how much great programmers have in common in they way they develepoed\ldots When you get those people to look at great piece of code, you get a very, very common reaction.”} (Bill Gates, in LAMMERS, 1989, p.83.)
\begin{flushright}
\end{flushright}
\end{quote}

Assim, o software livre facilita uma interação sem restrições com códigos diversificados. A inspiração por outros códigos, além de ser possibilitada, é incentivada por uma tradição de forte crítica e comentários dentro das comunidades FOSS. Há uma criatividade de grupo que caracteriza o produto final como sendo um “melhor” código porque são escritos por melhores programadores, porque eles são formados em contato com esse mesmo código “melhor”, formado paulatinamente dentro de uma comunidade com altas exigências estéticas. Ademais, dentro das comunidades FOSS há a idéia de que a exigência tradicional de estética e elegância técnica faz "sobreviver" o que a literatura clássica perdeu. Nesse sentido o Maurice Black comenta:

\begin{quote}
[A crítica literária] demonstrou seus compromissos políticos quase abandonando a literatura e a estética como seu tema [\ldots] e, mais ainda, demonstrando sua falta de fé na estética através sua reificação da feiúra ao nível do estilo da critica e de escolha de tema – com o objetivo de transgredir e destabilizar os cânones literários a fins políticos, os teóricos culturais agora estão com alacridade pronta para assuntos como loucura, tortura, amolação, monstruosidade, pornografia, e doença. A cultura informática, por outro, adotou um modelo tradicional de estética literária, como meio de mudança efetiva, achando uma utilidade política e um valor social a um produto bem acabado, que é no mesmo tempo inteiramente operacional e lindo a contemplar como um todo.\footnote{“The first has demonstrated its political commitments by all but abandoning literature and aesthetics as its subject matter [\ldots] and even by demonstrating its loss of faith in aesthetics through its reification of ugliness at the level of critical style and choice of subject matter – with the goal of transgressing and destabilizing the literary canon for political effect, cultural theorists now turn with ready alacrity to subjects such as madness, torture, pain, monstrosity, pornography, and disease. Computing culture, on the other hand, has adopted a traditionel model of literary aesthetics as a means of effecting change, finding political utility and social value in the well-crafted product that is at once entirely usable and wholly beautiful to contemplate.”}
\begin{flushright}
BLACK, 2002, p.20.
\end{flushright}
\end{quote}

\subsubsection{A programação como arquitetura e a regulação pelo código} \label{2.3.1.b}

Quando num estabelecimento privado de alimentação, alguém volta da sala de banho para a sala principal, como a praça de alimentação de um shopping por exemplo, encontra-se quase sistematicamente uma porta dando acesso às cozinhas, aos armazéns, ou a um lugar de descanso para os funcionários. Colocado na porta, acha-se uma mensagem proibindo o acesso: “acesso restrito”, “entrada proibida”, “reservado aos funcionários”. Assim, a pessoa dirige-se em função de seu estatuto (freguês, funcionário, segurança) pelo caminho que ela pode, para alcançar o lugar que ela quer. Quando alguém se conecta a sua conta Gmail, seu computador lê dados de um disco rigido de um servidor de um centro Google. O navegador dele lê então dados que são fisicamente vizinhos de numerosas outras contas Gmail. Porém, uma regra o impede de direcionar a cabeça de leitura do disco rigido para outra conta, como o tinha feito para a conta dele. \'E uma regra codificada numa linguagem para máquina que permite essa operação. Tem nesse servidor um sistema operacional que administra contas como espaços e privilégios distintos (usuários diferenciados, administrador de sistema, etc.). Então, como no shopping, uma arquitetura física vem delimitar salas e corredores, cujos acessos são restritos por normas expressas por símbolos, para controlar a atividade humana, o software de um servidor de email vem regular a atividade de seus usuários por uma arquitetura e leis.

\begin{figure}[htb]
\caption{Comparação entre um plano de arquitetura e um código delimitando espaços discos para usuários} \label{fig2.3}

\begin{multicols}{2}

Um plano de arquitetura 
\\
\includegraphics[width=60mm]{plan.png}

\columnbreak

Um código Linux padrão delimitando os espaços de vários usuários

{\scriptsize
\begin{verbatim}
root:x:0:0:root:/root:/bin/bash
fulano:x:1000:1000:fulano,,,:/home/fulano:/bin/bash
daemon:x:1:1:daemon:/usr/sbin:/bin/sh
nobody:x:65534:65534:nobody:/nonexistent:/bin/sh
sys:x:3:3:sys:/dev:/bin/sh
sync:x:4:65534:sync:/bin:/bin/sync
games:x:5:60:games:/usr/games:/bin/sh
man:x:6:12:man:/var/cache/man:/bin/sh
lp:x:7:7:lp:/var/spool/lpd:/bin/sh
mail:x:8:8:mail:/var/mail:/bin/sh
news:x:9:9:news:/var/spool/news:/bin/sh
uucp:x:10:10:uucp:/var/spool/uucp:/bin/sh
proxy:x:13:13:proxy:/bin:/bin/sh
\end{verbatim}
}
\end{multicols}
\end{figure}

Desde a publicação do livro de Lawrence Lessig, \emph{Code and others laws of cyberspace} (LESSIG, 1999), afirmar que o código é uma lei que regula os comportamentos por coerções e regras tornou-se uma idéia comum dentro das análises da sociedade da informação. Com metáforas detalhadas da lei, da arquitetura física, das normas sociais e dos mercados financeiros, o autor mostrou que o software regula os comportamentos, que “o código é lei”.  A partir dessa observação, a reflexão de Lessig vai se articular segundo o silogismo seguinte: se as instituições podem regular o software e o software pode regular os indivíduos, então as instituições podem regular os indivíduos pelo software. Assim sublinha-se a importância de entender as implicações políticas e sociais das decisões técnicas, por que elas encarnam uma nova forma de lei. De maneira mais geral também, discute-se as mudanças que os dirigentes devem promover para a Internet.

James Grimmelmann faz uma crítica do trabalho de Lessig e dos debates que nasceram das analogias entre software, lei e arquitetura. Embora essas metáforas sejam consideradas como heurísticas à compreensão dos papéis sociais e políticos do código, Grimmelmann aponta a falta de discussão das diferenças qualitativas entre software e lei, e entre software e arquitetura. Assim, se essas analogias fazem muito sentido para entender o objeto software no seu contexto social e político, várias diferenças de naturezas, de modo de operação diferenciam caracteristicas desses objetos. Para ele, a comparação com a arquitetura permite sublinhar a natura automatizada e imediata do software, enquanto a da lei permite apontar a sua natureza plástica. O software é \emph{automatizado} porque uma vez que ele foi programado, ele age sem precisar de nenhuma intervenção humana. Além da própria norma, que uma vez estabilizada se reproduz com autonomia, algoritmos performáticos, comunemente chamado de 'inteligentes' podem adaptar-se a evolução dos comportamentos que eles regulam. O software é \emph{imediato} pois, em vez de apoiar-se nas bases da sanção, ele simplesmente impede a ação de acontecer. Por um lado a coerção não tem graus diferentes dependendo da sua apreensão pelos comportamentos, ou seja do medo da sanção, pois a ato \emph{não pode} acontecer. Enfim, o software é \emph{plástico} porque é concebível qualquer sistema que um programador possa imaginar e descrever com precisão. Isso revela também a natureza frágil ou quebrável do código, isto é, que qualquer falha que possa ser imaginada pode ser explorada. Nesse contexto, uma característica suplementar do software diferencia o código aberto do fechado. Um software cujo código é aberto é transparente, seus modos de regulaçoes são acessíveis ao entendimento dos sujeitos dos quais ele determina o campo dos possíveis. Por outro, um software fechado mantém seu algoritmo regulador escondido, com fato único e autônomo, independente do entendimento que seus sujeitos têm dele.

\subsection{As pragmáticas da programação como paradigma de análise} \label{2.3.2}

\subsubsection{As pragmáticas da programação: a interpretação sócio-política de um conceito lingüístico} \label{2.3.2.a}

Umas das definições mais antigas da pragmática é feita por Charles W. Morris: “A pragmática é essa parte da semiótica que trata da relação entre signos e os usuários dos signos” (MORRIS, 1938). De um ponto de vista estritamente linguístico, a pragmática trata antes de tudo do sentido, como a semântica, mas para ela, é o uso feito das formas linguísticas que determina o sentido que elas possam ter. Na sua asa mais formalista, notavelmente na obra de Yehosua Bar-Hillel (BAR-HILLEL, 1970), usa-se de paradigmas como o estudo dos símbolos lexicais, do sentido literal e do sentido comunicado, e dos atos de linguagens. 

De maneira geral, a pragmática permite questionar vários fatos da linguagem: quando falamos, o que fazemos? O que dizemos exatamente? Quem comunica com quem? Quem sou eu para meu interlocutor? Qual sentido é necessário à coerência de minhas palavras? Uma palavra tem um sentido literal? Quais são os usos da linguagem? (ARMENGAUD, 2007). Assim falar sobre “pragmáticas da programação” seria questionar do mesmo jeito o ato escrever código: o que fazemos quando programamos? O que programamos? Como o usuário percebe o computador? Como o computador recebe as instruções do usuário? Como se formam os erros de interpretação e de sintaxe? Quais são os usos das linguagens, as suas evoluções? E, principalmente, em que medida a realidade do programador e sua capacidade a programar interagem, determinam-se uma com a outra?

Numa abordagem filosófica da pragmática e das interpretações sociais e políticas que ela permite, a pragmática reorientou o olhar da ciência da linguagem sobre os interlocutores. A complexidade deles, mostrada por sua análise, traz o analista a questionar os conceitos de sujeito e indivíduo, notavelmente se focando sobre a interlocução. Assim, se há uma Razão, ela somente pode ser real com a validação intersubjetiva do saber. Aplicada à estética, a abordagem da pragmática faz das obras artísticas um conjunto de formas simbólicas cujos sentidos e referências dependem das condições de vida e de ação dela, de seu contexto. A arte é então uma experiência que tem por consequência comunicar com um contexto que vem dar a ela seu sentido: “Eu não digo que comunicação é o propósito de um artista. Mas é uma consequência de seu trabalho.”\footnote{“I do not say that communication is the intent of an artist. But it is a consequence of his work.”}  (DEWEY, 1934). 

Assim sendo, a analogia entre a programação e a arte pelo uso da idéia de estética do código (CHOPRA e DEXTER, 2007) demonstra que, vários paralelos estão aceitos entre escrever código e escrever literatura, entre comentar códigos e fazer crítica literária, entre escrever scripts como se escreve poesia (BLACK, 2002). Por isso, aceitando a idéia de que a programação possa ser considerada como uma das “belas artes” (LEVY, 1992), o conceito de pragmática ajuda a afastar as noções de sujeito da análise do movimento do software livre e focar-se sobre a constituição do corpus de código que ele constitui. As diferentes comunidades e suas interações por volta de projetos de software constituem uma \emph{práxis} que se revela ao observador pela análise das relações desenvolvidas com a informação.

\subsubsection{A pragmática da programação como relação à informação e suas políticas informais} \label{2.3.2.b}

A hipótese que esta por traz do uso do paradigma da pragmática como ferramenta de análise do movimento do software livre é que a programação é um discurso sobre a informação. Um programa trata, gera, cria, modela informações, e as pessoas que os criam geram um discurso sobre essas informações, escolhem o jeito que a máquina vai tratá-las. Cada protocolo de rede trata os fluxos de informação de um jeito diferente, e o esforço de configuração feito pelo administrador de uma rede determinada, realiza escolhas específicas que também vêem estruturar a informação, sua difusão e sua recepção. De essas escolhas e regulações que constituam a relação à informação, os programadores constróem uma ontologia que o programa vem automatizar. Nesse sentido, a noção de pragmática permite isolar esta tarefa do programador e desenvolver uma visão ampla das interações que acontecem em torno do ato de programar, considerado como discurso. Além das interações entre os próprios usuários desenvolvedores, podem ser consideradas as interações deles com o objeto computador, suas lógicas, seus limites. Ela mantém uma relação privilegiada, por ser muito normativa, decisiva, com a máquina que trata a informação, então entendido como \emph{dispositivo}, ou seja, “conjunto heterogêneo que resulta do cruzamento das relações de poder e de saber” (AGAMBEN, 2007). Essa noção de "dispositivo" é central ao esforço das comunidades FOSS dentro do movimento tecnológico. A máquina isolada, particular, define-se, nas suas funções e possibilidades, pelos softwares que ela usa para tratar as informações que o usuário joga nela. As normas e estruturas de tais softwares resultem das relações de poder e de saber presentes no processo de codificação do programa. Então a pragmática do programador é a etapa a mais fundamental (após a criação da própria linguagem de programação), onde essas relações realizam-se.

Desta forma, a informação é estruturada pelo objeto software, e as comunidades F/LOSS geram um discurso em cima dos fluxos informativos que geram um conjunto automatizado e plástico que dá seu conteúdo ao SL/CA como movimento político e regulador do espaço digital. Porém, como argumentamos, a pragmática da programação é o vetor do agnosticismo político do movimento SL/CA pelo fato que é a lógica da programação que determina, e permite, a ausência de discurso político aparente, ou reverenciável de maneira comum. Portanto, as pragmáticas da programação formam um discurso próprio sobre a sociedade de informação que constrói uma base de referência própria por seus atuantes.  Assim, o usuário-desenvolvedor cria um discurso que é o ato. Com essa idéia, entendemos que o código é uma teoria, uma abstração e uma pragmática, ele é a escritura de um texto, uma prova, e a realização de uma experiência concreta. Por isso, idéias políticas, como a liberdade de expressão, são implementadas pelas comunidades no processo de produção e no objeto produzido. Um exemplo é a tecnologia Wiki, que foi construída em comunidades 'livres' com dinâmicas participativas, para produzir um software livre, permitindo a criação de conteúdo coletivo aberto. A criatividade de grupo cria um processo cujas condições se reproduzem no produto (SAWYER, 2003).

Em suma, o esforço do observador do movimento deve ser de emitir hipóteses ao nível mais fundamental desse discurso – o código – para entender a construção das entidades políticas do movimento SL/CA dentro da tipologia de suas políticas informais (transgressão, civismo, inversão, colaboração). 
