\chapter*{Conclus\~ao}

\begin{quote}
\textit{Contaram-me uma piada que o Richard Stallman tem costume de fazer durante suas numerosas aparições públicas em congresso e seminários para promover o Software Livre. Ele vai até o quadro ou pega uma folha de papel e começa desenhar:
\begin{center}
\includegraphics[width=60mm]{gnu.png}
\end{center}
“O movimento do software livre é tão experto que até o nome dele é recursivo!”, exclame-se ele no meio das risadas dos engenheiros, simpatizantes e amadores do SL/CA.}
\end{quote}

Há um pensamento contemporâneo que procura conceber a técnica como uma simples ferramenta. Assim, o contabilista, o maquinista, de maneira geral, o \emph{técnico} possui uma prática hermética que serve uma a outra, ou seja, a do \emph{pensamento} político, filosófico, social e econômico. Afirmamos as origens de tais ideais na filosofia antiga grega e sua concepção clássica do saber como um \emph{dentro} que deve ficar independente de seu \emph{fora}, a technê. A técnica, por sua vez, não participa da procura da felicidade, da beleza, do justo ou da verdade, mas a serve.

Urge então a imagem platônica do sofista, que usa o pensamento como uma técnica para tornar qualquer questão indiscutível, frequentemente a fins imorais.  Numerosas historiografias desta época, porém, mostraram que a corrente sofista era desacreditada dos padrões da história das idéias e que se tratava na verdade de uma filosofia própria, pragmática e racional, privilegiando a análise das situações, dos lugares, dos eventos e das linguagens de maneira concreta e não como um fim em si. Contudo, acham-se tais desenvolvimentos pejorativos em certas obras contemporâneas, como a do Bernard Stiegler, antigo aluna de Jacques Derrida e diretor do instituto de pesquisa e inovação (IRI) do centro Georges Pompidou, na França. Segundo ele, todo pensamento da técnica necessariamente ultrapassa os limites da filosofia. Para ele, a técnica não é nada.

Por tanto, se não é uma decisão política que conduz o movimento técnico científico atual e que recusamos o preconceito de imoralidade ou de inutilidade de uma technê autônoma, o que conduz então o movimento das redes e dos materiais que constrói nosso mundo já dependente dele? Acham-se elementos de resposta na obra do antropólogo Christopher Kelty e sua antropologia do geek através do estudo das significações culturais do movimento SL/CA. Segundo ele, a participação colaborativa e aberta de numerosos indivíduos permitiu a um público \emph{recursivo} legítimo de se constituir. Como a função recursiva fatorial, função matemática (figura 3.2) que multiplica um número pelos inteiros que o precedem, e assim produz um novo numero maior; o público identificado por Kelty realiza seu presente invocando seus elementos já existentes.

%\begin{center}
\setcounter{equation}{1}
%\begin{figure}[h]
%\caption{A função Fatorial} 
\begin{equation}
n!=\prod_{i=1}^n i= 1 \times 2 \times 3 \times \cdots \times (n - 1) \times n
\end{equation} \label{fig3.2}
%\end{figure}
%\end{center}
O software é o conceito que as comunidades livres vêem invocar recursivamente. O software invoca o software, que invoca o software, que invoca o software. E assim produz programas, formatos, protocolos novos que participem do movimento tecnológico contemporâneo.

Poderíamos descrever tal movimento como uma experiência vivida do pensamento pragmático segundo o qual, “o esforço não é de praticar a inteligência, mas de intelectualizar a prática” (ELDRIDGE, 1998, p.5). Trata-se de uma recusa da procura para a certidão com categorias pré-conceituais para o contexto de análise presente. Num nível mais epistemológico, isto é, uma posição anti-retificação para a qual o conceito e a teoria não são objetos independentes, mas abstrações, cujo produto volta a experiência.

Nesse sentido, as primeiras linhas do manifesto Hacker dizem: “todas as classes temem essa abstração implacável do mundo, em cima da qual as fortunas ainda dependem”. Entretanto, a “classe hacker" posiciona diferente, como notamos em seu discurso: “Somos os hackers da abstração. Nós produzimos novos conceitos,  percepções e sensações hackeadas dos dados brutos”.  Aqui, numa metáfora marxista de um mundo dividido por classes, a “classe Hacker” diferencia-se por uma relação inversa à abstração, como se ela não subisse sua cognição, mas a produziria a partir de um tratamento consciente das coisas num estado bruto.

Como observamos no capítulo primeiro desse estudo, as alternativas oferecidas pelo software livre sempre se constituíram pelo tratamento recursivo de uma matéria prima, Unix primeiro, depois os projetos GNU, BSD, Mozilla, para constituir um conjunto heterogêneo em termos de tecnologia, de regulamentos, como da esfera política e social. 

O foco dado à figura do usuário-desenvolvedor no segundo capítulo nos permitiu achar os mesmo termos dentro da ética e as significações culturais do hacking. Nesse sentido, achamos exemplos nos trabalhos da antropóloga Gabriella Coleman, que nos mostraram como o agnosticismo político das comunidades SL/CA permitiu aos indivíduos de reinterpretar sua liberdade técnica e ética num contexto digital. A evolução, o ato, o movimento não se operam segundo idéias, preceitos, propósitos políticos preestabelecidos, mas pela prática de um presente. Este presente realiza-se num ato, o de programar, cujas pragmáticas pertencem tanto a uma responsabilidade legal, de regulamentação do espaço digital, como a uma responsabilidade artística, de manter e promover uma estética forte e elegante. 

A relação à informação que nasce dessa pragmática da programação estrutura as especificidades das comunidades que tentamos apontar no capítulo três. Assim, as pragmáticas do ato de programar e as lógicas culturais da cultura do programador interagem para sublinhar pontos característicos das comunidades gNewSense, Samba e BSD, nas prioridades que dêem no processo de tratamento da informação. Que sejam as preocupações de liberdade, abertura ou segurança que confirmem o ato permanece recursivo, por tratar sempre as problemáticas comunitárias a partir do desenvolvimento das tecnologias e informações presentes e passadas.

Assim, constitui-se um público cujas modalidades dêem seus conteúdos políticos aos seus movimentos que se diferenciam e, às vezes, se opõem. O que permanece comum a esse público é seu tratamento recursivo das tecnologias para produzir novas tecnologias enriquecidas e modificadas. É esse movimento tecnológico que baseia o imaginário político de seus atuantes. A técnica é entendida como uma infra-estrutura, um conjunto de meios socialmente identificados e diferenciados para constituir sistemas operacionais, protocolos ou formato de arquivos que mantém um ambiente tecnológico, então dependente de várias éticas cujos atuantes se sentem responsável. 

Portanto, por mais diversificada que seja, a ética do SL/CA permanece uma alternativa a outro modelo, proprietário, por expor os laços entre moral e técnica, infra-estrutura e superestrutura, sistema operacional e sistema social, como tantos objetos devendo ser construído por um público que ganha sua legitimidade sendo aberto e livre.

\vspace{1cm}

\begin{center}
***
\end{center}
