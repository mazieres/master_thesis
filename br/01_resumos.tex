\begin{center} \textbf{RESUMO} \end{center}

\vspace*{.5cm}

Esta dissertação procura apresentar as significações políticas e culturais de um movimento de Software Livre e de Código Aberto (SL/CA) entendido como conjunto muito heterogêneo de comunidades e projetos. Ademais, a partir de um histórico do objeto “software” desde a sua origem, mostramos como ele foi diferenciado do hardware e depois encerrado como um objeto fechado pela companhias de software nascentes. Nesse contexto, o movimento SL/CA aparece tanto uma reação ao fenômeno de blackboxing, como uma continuação da tradição de compartilhamento de informações dentro da engenharia da computação. Por isso, estrutura-se ao redor de vários ramos da ética hacker e de seu agnosticismo político para constituir uma alternativa tecnológica concreta. Isto nos permite afirmar que as características sociopolíticas das comunidades do Software Livre devem ser procuradas no próprio ato de programar, na pragmática, como arte ou regulação. Dessa forma, estudamos os casos específicos de varias comunidades (gNewSense, Samba, BSD) para tentar sistematizar os seus posicionamentos tecnológicos e sociopolíticos a respeito do movimento tecnológico contemporâneo.

\textbf{Palavras-Chaves:} Software livre, Hacker, Programação (Computadores) – Aspectos políticos, Programação (Computadores) - Pragmática.

\newpage

\begin{center} \textbf{ABSTRACT} \end{center}

\vspace*{.5cm}

This dissertation presents some political and cultural significations of a Free Software Movement, understood as a heterogeneous aggregation of projects and communities. Then, the historical analysis of the "software object" shows how it's become, in the first place, differentiated from the hardware and, then secondly, closed as an end-product by the rising software companies. In this context, the Free Software Movement presents itself as a reaction to \emph{blackboxing} phenomena, as well as a continuation of the computer engineering tradition of sharing knowledge freely. Therefore, the FS movement has become structured through diverse blends of Hacker Ethic and its own political agnosticism, in order to build a concrete technological alternative. This leads to the argument that sociopolitical characteristics of Free Software communities should be found in the very act of programming, and in its pragmatics as an art or a regulation. Finally, specific cases of several communities (gNewSense, Samba, BSD) are examined in an attempt to systematize their sociopolitical and technological positions of the contemporary technological movement.

\textbf{Tags:} Free software, Hacker, Computer programming management –   Political aspects, Computer programming management –Pragmatics.

\newpage

\begin{center} \textbf{RESUMÉ} \end{center}

\vspace*{.5cm}

Ce mémoire cherche à présenter les significations politiques et culturelles d’un Mouvement du Logiciel Libre considéré comme un ensemble hétérogène de communautés et de projets. De plus, a partir d’un historique de l’objet « Logiciel » depuis son origine, nous montrons qu’il a été différencié du matériel et, ainsi, fermé comme un produit fini par les entreprises de softwares naissantes. Dans ce contexte, le Mouvement du Logiciel Libre se présente autant comme une réaction au phénomène de « blackboxing », que comme une continuation de la tradition de libre-échange d’informations au sein de l'ingénierie informatique. Pour cela, Il se structure à travers plusieurs aspects de l'éthique Hacker et de son agnosticisme politique pour construire une alternative technologique concrète. Ainsi, nous pouvons affirmer que les caractéristiques socio-politiques des communautés du logiciel libre doivent être recherchées dans l’acte même de la programmation, dans sa pragmatique, en tant qu’art ou régulation. De cette manière, nous étudions les cas spécifiques de plusieurs communautés (gNewSense, Samba, BSD) pour tenter de systématiser leurs positionnements techniques et socio-politiques envers le mouvement technologique contemporain.

\textbf{Mots-clés :} Logiciel Libre, Hacker, Programmation (ordinateur) - Aspects Politiques, Programmation (ordinateur) - Pragmatique.

\newpage




