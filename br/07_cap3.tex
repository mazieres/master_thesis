\chapter{As pragm\'aticas da Programaç\~ao e suas l\'ogicas pol\'iticas informais} \label{3}

Nós observamos no capítulo anterior as significações culturais do movimento do software livre por meio de: a) da noção de ética hacker, o que permitiu apontar quatros idéias-tipos de lógicas de suas políticas informais (transgressão, civismo, inversão e colaboração); e, b) construímos a idéia de pragmáticas da programação como sendo o vetor da interação do usuário-desenvolvedor FOSS com a informação, e por isso, nos permite emitir hipótese ao nível o mais fundamental da atividade do programador – isto é, o código – para entender a construção da entidade política do movimento SL/CA.

Para esse propósito, propomos aqui uma tipologia de três pragmáticas que se encontram pautadas no discurso das comunidades livres. Em primeiro lugar, consideramos a pragmática de \emph{liberdade} entendida como a ideologia atrás do movimento GNU e os primeiros rastros da cultura do software livre. Por exemplo, pode-se escolher uma ferramenta pelo fato dela estar sob licença GPL. Assim, ela parece como imperativo superior a qualquer outro, enquanto procura-se a construção de uma solução antes de tudo homogeneamente “livre”. Ainda, a pragmática da liberdade é o discurso e as interações do movimento SL/CA que privilegia as tecnologias “100\% livre”, no sentido que o entendem os atuantes.

Além disso, encontramos a idéia de \emph{abertura} nas discussões sobre a compatibilidade das tecnologias, principalmente os protocolos e os formatos de arquivos. A pragmática de abertura seria, então, o discurso dos atuantes do SL/CA sobre essas problemáticas, e o que privilegia essas características sobre as outras.

Por fim, a preocupação com a \emph{segurança} acha-se particularmente nas problemáticas de estabilidade dos sistemas de informação e suas defesas contra eventuais ataques na rede. Assim, a pragmática da segurança seria refletida  nos esforços das comunidades em codificar programas antes de tudo seguros, mesmo se as tecnologias produzidas sofrerem a falta de desempenho, de compatibilidade ou se as licenças híbridas protegem parte do código.

Ademais, neste capítulo, o nosso propósito é fazer interagir a tipologia de lógicas das políticas informais próprias à ética hacker com as pragmáticas da programação para que se revele as propriedades políticas do ato que consideramos como fundamental à atividade social de programação. Para seguir esse caminho, trataremos cada uma das pragmáticas isoladas com a apresentação de um estudo de caso, analisado a partir das quatro lógicas políticas. A saber, a pragmática de liberdade será examinada a partir do estudo da comunidade gNewSense (\ref{3.2}), já a de abertura com o apoio do projeto SAMBA (\ref{3.3}), e a de segurança por meio dos sistemas BSD (\ref{3.4}); e finalmente, apresentaremos um quadro de comparação das interações conceituais desejadas (\ref{3.5}).

\section{Pesquisa de campo e metodologia} \label{3.1}

A analise a seguir é o resultado de um trabalho de campo realizado ao longo do meu processo de mestrado, iniciado em julho 2006. O primeiro ato que quero considerar relevante a respeito da minha pesquisa é o uso progressivo de todas as tecnologias livres disponíveis no meu trabalho universitário. Por exemplo, em destaque, a migração do sistema Windows para o sistema Linux (Ubuntu). De fato, tal ação necessita bastante procura de informação na Internet para se familiarizar com um sistema ainda não "intuitivo" como pretendem ser seus concorrentes proprietários. Boa parte dessa procura de informação realiza-se interagindo com outros usuários nas plataformas virtuais de "ajuda" de todas as comunidades. Construiu-se, então, um vocabulário específico permitindo expressar os problemas encontrados ao longo do nosso caminho. Esse próprio vocabulário é aquele mesmo que permite interagir com as comunidades para assuntos mais \emph{off-topic} (fora do assunto) e oferece então a entender o ambiente político-ideologico que se encontra ao redor das comunidades livres para justificar as suas escolhas de usar ou não software livre.

Tendo adquirido um bom nível técnico, suficiente para ser “autônomo” e somente precisar de um acesso ao Google e/ou ao Wikipédia para conseguir solucionar a maioria dos problemas encontrados, aproveitei um contato pessoal para entrar em um centro de computação de um departamento de ciências exata\footnote{Parece-me relevante precisar a área desse departamento (ciências exatas) porque esses departamentos fazem, de maneira geral, um uso muito mais intensivo das tecnologias da informação do que os departamentos afiliados à apelação “ciências humanas”.} de uma grande universidade paulista. Assumi então a posição de voluntário com carga de técnico de apoio, isto durante cinco meses (de maio até setembro 2008). Rapidamente foram-me atribuídos projetos precisos: atualização das estações Linux-Fedora da versão 7.0 para a versão 9.0, preparação da distribuição Ubuntu para a padronização e implementação dentro do departamento (substituição a Fedora), alguns testes para a configuração de um servidor de virtualizaç\~ao de sistemas para ajudar o atendimento (Xen, VirtualBox). Além dessas tarefas predeterminadas, eu tinha que assumir vários serviços de atendimento ao usuário, isto é resolver problemas diários que alguns pesquisadores e alunos do departamento encontravam ao usar ferramentas livres.

Embora o centro usasse de tecnologias livres e tecnologias proprietárias, podemos observar que interagi principalmente ao redor de tecnologias de código aberto. Isso aconteceu pelo fato de eu me apresentar aos administradores do centro de computação como interessado especificamente nessas tecnologias, por causa da minha pesquisa, e que isso justificava meu estatuto de “voluntário”\footnote{concretamente, isto quer dizer que recebi nenhuma verba por esses serviços, e que a responsabilidade jurídica da minha presença aqui dependia do meu departamento (DCP/IFCH, UNICAMP) e não de esse centro de computação.}. Porém, embora o fato de ser volunt\'ario me deu mais flexibilidade na organizaç\~ao de meu trabalho, assumi uma carga horaria normal, ou seja, 8 horas por dia, 5 dias por semana.

O meu processo de aceitaç\~ao dentro do centro de computaç\~ao foi facilitado pelo fato que minha imagem de aluno francês ultrapassava muito aquela de cientista social. Em relaç\~ao aos funcion\'arios, a intergraç\~ao foi mais r\'apida ainda, sendo que eles se interessavam mais na qualidade do meu trabalho e de minhas observaç\~oes do que em qualquer outra coisa. Em realç\~ao aos estagiarios (mais jovens), como disse, eles eram mais interessado em detalhar seus preconceitos e fofocas sobre à França (Moulin Rouge, Zidane, Carla Bruni, Sarkozy, etc.) do que qualquer outra coisa. N\~ao encontrei nenhum problema para fazer as perguntas que eu queria, e assim, encontrar todas as informaç\~oes que procurava. Atribuo esta facilidade com minha entrada em campo a dois fatores 1) tentei ser amig\'avel com todos e 2) competente no meu trabalho. 

Entre as perguntas que eu fazia, bastante eram de ordem hist\'oricas, ou seja, eu pedia tanto do que poss\'ivel como tal ou tal tecnologia nasceu, evoluiu, superou seus equivalentes, etc. Também eu tomava cuidado em entender os conceitos modernos de computaç\~ao (protocolos, termos organizacionais, etc) através de um olhar mais "antigo" para compreender suas raizes\footnote{exemplos ilustrativos: \emph{Internet} é uma rede feita de protocolos, um \emph{Podcast} é um v\'ideo}. Nesse sentido, as perguntas que me serviram mais foram "como era antes?", "como chegou a ser assim?" e "para onde vai?"

Basicamente, esse trabalho me permitiu sistematizar o contato que eu tinha com as comunidades livre e sair do “romantismo” que pode sofrer uma interação sob propósitos pessoais. Em vez de encontrar um problema especifico para minha situação especifica, eu aprendi procurar de maneira sistemática soluções para uso amplo, isto é, para as centenas de computadores em uso no departamento. As relações profissionalizam-se dentro dos fóruns de ajuda e os diálogos ficam estritamente ao redor das problemáticas técnicas. Desse jeito trazem-se os discursos encontrados dentro do próprio centro de computação e os debates internos vêm realizar as escolhas técnicas ainda em duvida na plataforma virtual.
Esta observa\c c\~ao me levou a formar a hip\'otese que acabaria sendo central na minha disserta\c c\~ao sobre a pragm\'atica da programa\c c\~ao. De fato, observei que os critérios selecionados pelos atuantes durante esses debates eram técnicos, justificava-se tal ou tal opção por argumentos técnicos, porém era nesse momento que se formulava a aura ideológico-política dos projetos mencionados. Ilustramos com um debate sobre as distribuições do Linux:

\begin{quote}
\footnotesize
{\bfseries Eu}: Então olhei mais ou menos todas as ferramentas disponíveis no Ubuntu e acho que é possível  começar fazer os testes \\
{\bfseries 1}: bom, acabamos o a padronização do Fedora, e depois passamos nos testes do Ubuntu. Acho que vai ser mais simples para a parte multimídia. Fedora da um trabalho para os plug-ins... \\
{\bfseries Eu}: Pois é... Com Ubuntu fica muito simples, permite instalá-los com um comando só \\
{\bfseries 1}: Tem idéia de porque eles não fazem a mesma coisa do que o Ubuntu? O Fedora esta perdendo usuários com essa atitude... \\
{\bfseries Eu}: o Ubuntu quer facilitar o acesso para os usuários, o Fedora é mais fiel a respeitar as licenças dos softwares disponibilizados. \\
{\bfseries 2} (\emph{ouvindo o argumento}): Seria você, trabalharia em cima do Kurumi, ele já faz tudo isto, DVD, flash, Java, tudo t\'a disponho. \\
{\bfseries 1}: mas ai vai dar problema para os aplicativos científicos, ele não é feito para isso. Não, acho que Ubuntu já ta bom. \\
{\bfseries 2}: instalei ArchLinux ontem na minha maquina para experimentar... Muito legal esses sistema, bom não da para colocar aqui, as atualizações são demais rápidas. Mas eles tentam organizar o sistema que nem o BSD, bem logicamente, para os desenvolvedores mesmo. \\
\ldots \\
{\bfseries 1}: pois BSD é um saco viu, estou trabalhando com o servidor de impressão e para fazer dialogar ele com o Linux dá um trabalho. Já no outro no [departamento vizinho] eles homogeneizaram todas as máquinas para facilitar o serviço... Mas aqui o povo quer guardar o BSD. Ta é mais segura, estável, etc. 
}
\end{quote}

Esses debates freqüentes aconteciam também com os usuários universitários. Por\'em percebia-se uma certa confus\~ao entre os argumentos ideologico-politicos a favor do SL/CA e, ao mesmo tempo, uma cr\'itica aos funcionairos t\'ecnicos por não disponibilizar as ferramentas fechadas. 

Não realizei durante esse campo entrevistas formais, mas usei das interações informais que eu podia analisar depois com os rastos que deixava delas no meu caderno de campo. O dialogo transcrito a cima é um exemplo de rastros que podia-se encontrar no meu caderno. Também, anotei quadros comparativos de tecnologias equivalentes em debates, com os argumentos sublinhados pelos actantes (dentro ou fora do centro de computa\c c\~ao), palavras com duplo sentido, piadas especificas às personas que encontrei, etc.

Durante o período de escrita (6 meses), passava v\'arias horas por dia nos canais IRC, olhando e participando de debates técnicos que, de vez em quando, acabavam em debate sobre os propósitos e compromissos sociais e políticos do SL/CA. Entre outros, os canais IRC nos quais participei foram: Samba, gNewSense, FreeBSD, LaTeX, Ubuntu, Fedora, OpenOffice, Hack, Kubuntu, GNU, FSF\footnote{servidor: irc.freenode.net:6667}. Essa interação em tempo real permitiu reproduzir alguns debates presentes no departamento no qual trabalhei, mas numa praça pública maior, pois centenas de usuários conversam aqui na mesma interface. Essas observaçoes ineragiam diretamente com meu processo de escrita, permitindo verificar informaç\~oes na hora que eu as escrevia, jogar debates dentro dos canais quando me faltava opini\~oes formalizados. Isto permitiu tornar previz\'ivel a maioria dos pontos de vista sobre os assuntos que eu tratava, o que, eu acho, é um dos passos anterior e necess\'ario à sistematizaç\~ao. Tentativas de an\'alises quantitativas de dados qualitativos foram realisadas em cima dos arquivos IRC das comunidades Debian e Ubuntu, principalemente a respeito do uso das palavras "livre" e "aberto". Porém, nao encontrei resultado satisfat\'orios sendo que n\~ao tinha ainda achado tipologias interessantes para explorar.

No IRC, obteve-se uma interação completa sobre o assunto desejado com duas pessoas, conscientes do meu propósito universitário. Neste caso podemos falar de entrevistas formais que guardei  e analisei na sua integralidade. Aconteceram dentro da comunidade gNewSense, e uma parte é transcrita no ponto \ref{3.5}.

De maneira geral, o que é importante sublinhar é a grande abundância de informaç\~ao dentro das comunidades livres. Como nota o antrop\'ologo Christopher Kelty, "\emph{Geeks talk a lot}" (KELTY, 2008) e eles como eu sendo interessados no assunto, n\~ao tem retenç\~ao de informaç\~ao nenhuma. Que seja dentro do centro de computaç\~ao onde eu fiz campo, ou nas inumer\'aveis plataformas virtuais nas quais interagi, as minhas perguntas formulavam-se seguindo os fluxos presentes de interesses e n\~ao respondiam a um roteiro pré-estabelecido, do tipo questoni\'ario ou linhas diretivas. Por um lado isto me impede de reformular na presente seç\~ao o caminho exato e detalhado da minha pesquisa de campo e da minha percepç\~ao te\'orica. Por outro lado, quero pensar que tal m\'etodo permite, no prazo curto de um mestrado, mergulhar mais profundentemente nas representaç\~oes dos atores que foram encontrados. Iniciei meu processo de observaç\~ao sendo chamado pela grande versatilidade desse movimento, dos discursos e posicionamentos pol\'itico-econ\^omicos que ele permite, e tentei sistematizar essa diversidade juntando os fluxos discursivos que encontrei.


\section{As pragmáticas de liberdade: A comunidade gNewSense} \label{3.2}

\begin{quote}
Software não-livre nunca é uma solução, então, por favor, não racionalize, justifique, ou minimize as consequências de propor software não-livre como uma solução.\footnote{“Non-Free Software is never a solution so please do not rationalize, justify, and minimize the consequences of proposing non-free software as a solution”}
\begin{flushright}
5a diretriz comunitária de gNS
\end{flushright}
\end{quote}

A pragmática de liberdade é a primeira que aparece em contato com as comunidades SL/CA. Por isso, os responsáveis quererem defender por meio da tecnologia o direito a participar da evolução livremente, pois o código aberto é a condição de uma idéia de liberdade. O modelo GNU e o domínio de código sob licença GPL assumem de maneira radical esse propósito do movimento SL/CA. Tais pretensões foram abraçadas emblematicamente pela comunidade gNewSense, que tenta construir uma versão 100\% livre de Linux, pois o mesmo no é "livre" no sentido que quer promover a FSF e as comunidades perto dela. As distribuições de Linux apontadas sempre permitiram a instalação de software com restrições que, no final, constituíam um sistema sob direitos heterogêneos. Essa banalização de códigos restritos no ambiente Linux fez um passo pela frente com a adoção em 2006 de pedaços (\emph{blobs}) de códigos fechados no próprio kernel do sistema. Esses blobs foram então incorporados aos sistemas distribuídos, incluindo aos mais dedicados aos ideais do Software Livre, como é o caso da comunidade Debian\footnote{Voto da comunidade Debian a favor do uso dos \emph{blobs}: dia 15/10/2006, http://www.debian.org/vote/2006/vote\_007.en.html}. 

Nesse contexto, Richard Stallman, fundador da FSF, e Mark Shuttleworth, patrocinador de Ubuntu, sublinharam juntos a importância de manter um esforço para construir e manter uma versão literalmente "livre" do sistema. Um estudioso, Paul O'Malley, trouxe o debate para ao público através do IRC e, junto com Brian Brazil, começaram a desenvolver uma arquitetura de sistema renovada. Os blobs e os repositórios de softwares fechados e/ou restritos foram tirados e formou-se uma comunidade por volta desse sistema experimental que ganhou o nome de gNewSense (gNS). Os usuários-desenvolvedores confrontaram-se a dificuldades técnicas importantes, tornando o uso do sistema difícil para os propósitos comum e raramente compatível com os hardwares. Porém, atualmente, em Junho 2009, o sistema está disponibilizado na versão 2.2 e conseguiu combinar todas as alternativas possíveis aos programas restritos, oferecendo uma alternativa possível, embora seja ainda complicada, aos compromissos feitos pelas distribuições concorrentes.

Cabe ressaltar que, as análises propostas aqui foram realizadas a partir de materiais encontrados principalmente no Fórum, nas mailing-lists da comunidade\footnote{http://wiki.gnewsense.org/index.php?n=ForumMain.ForumMain}, nas interações no seu canal IRC\footnote{\#gnewsense@irc.freenode.net:6667}. Além da observação dos debates ao vivo, e nos arquivos de anais de chat, foram realizadas entrevistas de vários usuários-desenvolvedores, entre os quais, foi um dos fundadores da distribuição, Paul O'Malley.

\subsection{Lógicas de transgressões} \label{3.2.1}

As pragmáticas de programação desenvolvidas pelos desenvolvedores dessa comunidade não são entendidas como a transgressão. O hacking, compreendido no seu aspecto underground, não é um vetor desenvolvido no esforço tecnológico que conduz a construção desse sistema alternativo. De fato, os usuários-desenvolvedores gostam de se apresentar, como os do projeto GNU ou próximo à FSF, como 'gnu hacker'. Por ser um ato profundamente político e cívico, as características das pragmáticas da comunidade gNewSense devem ser procuradas nas outras lógicas que constituam nossa tipologia.

\subsection{Logicas de civismo} \label{3.2.2}

As pragmáticas do civismo aparecem dentro de gNS como um conjunto de escolhas para se conseguir um resultado de propósito geral. De fato, as restrições técnicas que se impõem aos hackers da gNS para desenvolver o sistema em cima de uma plataforma com limitações importantes é um ato positivo no sentido de ignorar as alternativas e os meios-termos realizados pelas outras distribuições do Linux. Há um recuso categórico das perspectivas de desenvolvimento 'reformistas' que encaram a mudança para o 'livre' como um processo progressivo que necessita a utilização de ferramentas com licença restrita e/ou propriet\'rias. Essa posição dos desenvolvedores reafirma-se cotidianamente nas plataformas de interação com os usuários que peçam ajuda para o uso do sistema e sempre enunciam soluções não-livres. Vejamos a seguir o comentário de um usuário, o mesmo foi escrito para convencer uma pessoa que queria deixar de usar gNS e de experimentar outras distribuições:

\begin{quote}
\textit{Instale ubuntu, [\ldots] depois você poderia seguir a evolução do software 
livre e de gNS, mudar para alternativas livres, um bit após o outro\ldots ok, 
ubuntu não é livre, mas é mais livre do que Windows.}\\
Eu entendo o que você tenta fazer, porém, a mensagem que você manda mesmo é: "tudo bem em usar software não livres por enquanto; quando o software livre não tiver mais diferenciais funcionais com o não-livre, então pode voltar". Isso torna artificial a ética do software livre e não leva agente em lugar nenhum.\footnote{“ then you might follow the evolution of Fsoft and gns, change to free alternative, one byte after the other\ldots ok, ubuntu is not free, but it's freer than Windows.” / “I see what you're trying to do, but the message you actually send out is: "it's ok to use non-free software for now; once free software has caught up and there is no functional difference between free and non-free anymore, then you can switch back". This hollows out the ethics of software freedom and doesn't get us anywhere”}
\end{quote}

Nesse exemplo, o desenvolvedor cobra o usuário de propor um software não-livre (uma distribuição mais amigável do Linux) como jeito de conseguir chegar até um uso exclusivo do Livre. Isso está na contra mão do esforço cívico da comunidade em agir radicalmente. Nesse sentido, a comunidade gNS se caracteriza por uma forte postura política e por critérios tradicionais, ou seja, um ideal realizado por uma ação comum contra um inimigo declarado. De fato, os quatro tipos de liberdades e a filosofia original do projeto GNU são as bases desse movimento tecnológico de criação de um sistema operacional e plataforma de desenvolvimento de software livre. Isto se dá para manter uma alternativa contra todas as formas não-livres de códigos. Entretanto, para esse propósito, um "sacrifício" técnico de não se poder usar ferramentas básicas do mundo virtual contemporâneo. Por exemplo, assistir aos vídeos online ou acessar sites web “dinâmicos”, ou ainda o acesso a vários formatos básicos dos aplicativos de escritório são ações dificultadas. Nesse sentido, os integrantes desse projeto pretendem ativar um ato de civismo, através de pragmáticas de liberdade radical ao facilitar o custo técnico do desenvolvimento de uma alternativa inteira que qualquer uma poderá aproveitar.

\subsection{Lógicas de inversão} \label{3.2.3}

O sistema de propriedade intelectual que estrutura o mundo do software e seus conflitos é baseado em regulamentos, leis, convenções nacionais e internacionais (por exemplo o DMCA – Digital Millenium Copyright Act), que as licenças particulares convocam. Uma tecnologia é fechada ou restrita porque sua licença chama o conceito de copyright ou de direito autoral na licença, caso o usuário concorde em instalar o programa. O propósito do projeto GNU e da licença GPL foi de usar da mesma legitimidade jurídica para defender o oposto, ou seja, a não-apropriabilidade do código disponibilizado. Isto é uma lógica de inversão das defesas proprietárias para constituir defesas do "livre". 

Dessa forma, as preocupações legais são no centro da atividade da comunidade gNewSense e por isso, assumem-se os próprios desenvolvedores dos 4freedom checkers, além das categorias tradicionais (codificadores, suporte ao usuário, gerenciamento dos bugs). Esses desenvolvedores monitorão o sistema e todos os programas disponíveis para verificar a ausência de falha legal, permitindo um código de licença restrita a ser instalado. Para isso, um construtor (builder) específico foi programado para testar os códigos instalados e suas compatibilidades com as exigências de liberdade da comunidade. Contudo, a exigência de uma licença compatível com a GPL não é suficiente para esse propósito, e os desenvolvedores procuram estabelecer se os programas não apontam para fontes secundárias restritas. Mais uma vez, as implicações de tal dedicação aparecem particularmente no contato com os usuários; como  uma resposta de um desenvolvedor a um usuário, que queria instalar um emulador Windows (Wine, sob licença GPL) para poder utilizar seus jogos (proprietários):

\begin{quote}
Um ponto central para entender gNewSense e todo o Software Livre, é de entender que é seu propósito legal de ser completamente software livre. Enquanto preciso pesquisar mais para saber se Wine é livre ou não, o software livre não é para dar um jeito de graça para rodar todos seus aplicativos Windows. É um movimento político, social e tecnológico para um software livre.\footnote{“one of the key points about understanding gNewSense and all free software, is to understand that its legal goal is to be completely free software. While I need to do more reasearch into weather or not Wine itself is, Free Software isn't about  giving a free (monetary wise) way to runs as many windows programs as you can. It's a political, social, and technological movement for Free (as is Freedom) Software”}
\end{quote}

A propósito do que constitui a base política e social que evoca o desenvolvedor, apontamos a procura de uma homogeneidade jurídica inédita e absoluta. Para tanto, a maioria dos debates nas mailinglists de desenvolvimento da distribuição estão focados sobre os aspectos legais dos softwares, enquanto os aspectos técnicos, quando não envolvem uma tecnologia alternativa, são deixados à distribuição-mãe do sistema (Ubuntu). Separar o que é ou não livre é a principal lógica da organização do sistema gNewSense. As lógicas de liberdade levam os programadores a escolher quais softwares podem ser instalados e quais não podem. Assim, ilustramos uma inversão do radicalismo proprietário para um radicalismo do 'livre', e ainda uma “ignorância” deliberada de todos os 'meios-termos' que se encontram no meio tecnologico do Open-Source.

\subsection{Lógicas de colaboração} \label{3.2.4}

Um espaço de colaboração sem restrições é o propósito final da GPL; e as possibilidades de colaboração instauradas pela comunidade gNS são ao mesmo tempo absolutas e fechadas. De fato, por um lado, o potencial de dinâmica participativa dentro do sistema é infinito, porque todo seu código é livre, acessível, aberto, modificável nos termos da GPL e as suas licenças são semelhantes. Por outro, o espaço criado n\~ao aproveite de outros projetos cujas condições de apropriabilidade são distintas. Por isto, considerando a extrema variedade das naturezas jurídicas dos códigos no desenvolvimento das tecnologias, o domínio GPL e assimilados, são totalmente hermético a boa parte das inovações. Elas ainda devem ser recodificadas (quando não são patenteadas) num projeto que responda aos critérios da comunidade.

Ademais, a ideologia do projeto a respeito das possibilidades de colaboração pertence a dois prazos temporais. Desse modo, entendemos que à curto prazo, há um sacrifício em nome de uma utopia, uma idéia de liberdade absolutamente não restrita que bloqueia as possibilidades de colaboração e, assim, permite as trocas de códigos somente dentro do domínio da licença GPL. Esse compromisso dedica-se inteiramente ao ideal do projeto GNU, para que à longo prazo seja  realizado sua concepção de colaboração, absoluta e não restrita. Isto representa um posicionamento realmente ativista dentro das comunidades livres, que volta à heterogeneidade de licenciamento. Meios e fins se misturam para um propósito do movimento, refletindo a ideologia proposta e por isso a contradição que aponta a falta de abertura como restrição a colaboração não atinge as representações dos atuantes voltados a um propósito "maior".

\section{As pragmáticas de abertura: A comunidade Samba} \label{3.3}

Umas das principais dificuldades encontradas pelo sistema Linux dentro de instituições foi sua falta de compatibilidade com os protocolos de códigos fechados. De fato, com a multiplicação das estações Windows, todos passaram a usar os protocolos de redes internas próprias a esse sistema para os computadores comunicarem-se dentro de uma mesma instituição. Em tal ambiente foi difícil para os sistemas Linux, servidores ou clientes gerenciar tais protocolos, ou mesmo, simplesmente comunicar-se com eles. Foi com a intenção atenuar essa carência específica que foi criado o projeto Samba, um projeto agora antigo e com muito sucesso no mundo do SL/CA.

Iniciado como projeto de doutorado por um aluno australiano, Andrew Tridgell, em 1991, o software Samba permitiu pouco a pouco aos sistemas baseados no Unix (Linux, BSD, e outros) de dialogar nas redes com os sistemas e servidores Windows. Na prática, permitiu, dentro de outras coisas, aos clientes windows de interagir com servidores baseados na tecnologia Unix, notavelemente, para os serviços de impressão e de compartilhamento de arquivos. Licenciado sob GPL, o projeto se tornou padrão em todos os ramos do SL/CA e para gerenciar os protocolos de redes híbridas. 

Vale à pena dizer que as análises propostas aqui estão realizadas a partir de materiais encontrados principalmente no Fórum\footnote{http://www.nabble.com/Samba-f13150.html} e as mailing-lists da comunidade e nas interações no seu principal canal IRC\footnote{\#samba@irc.freenode.net:6667}.

\subsection{Lógicas de transgressões} \label{3.3.1}

As lógicas de transgressão estão presentes na comunidade Samba pelo simples fato de serem as que fundaram o projeto. Para conseguir se comunicar com os protocolos de rede da Microsoft, os quais são fechados, foi preciso usar os métodos da engenharia reversa. Sendo fechados, os protocolos apresentam-se ao desenvolvedor como uma 'caixa preta', cujas características podem ser descobertas somente pela análise de suas entradas e saídas. Com o uso de packet sniffer, os desenvolvedores usam de ferramentas comuns da exploração de rede para 'sentir' o comportamento dos protocolos fechados e assim determinar qual código permitirá se comunicar com eles. O Samba explora na verdade um dos casos onde tal prática é permitida, isto é, quando é realizada a fins de permitir uma interoperabilidade entre padrões diferentes.

Tal atividade ilustra-se pela exigencia de uma competência técnica avançada, e representa um desafio atraente para seus atuantes: derrubar um segredo proprietário e fazê-lo funcionar com uma tecnologia livre. As referências comuns dentro do hacking underground estão presentes no discurso dos desenvolvedores à medida que não ultrapassam os limites que a lei permite, pois, por ser um projeto aberto e muito usado, as interações dos usuários-desenvolvedores acontecem à vista de todos.

Porém, num contexto mais geral de observação, podemos revelar aqui umas das características das políticas informais de transgressão, isto é, o ato transgressivo é muito dependente do objeto que ele tenta derrubar. Se nós consideramos o ato de permitir os protocolos livres de dialogar com os proprietários como um ato em favor do “livre”, então o ato de transgressão realizado nesse propósito fica necessário enquanto os protocolos são mantidos. Nesse sentido, se as motivações dos usuários-desenvolvedores que participam da comunidade Samba são, por parte, inspiradas pelas lógicas de transgressão que sua codificação necessita, elas estão dependentes da existência desses mesmos protocolos proprietários. Por isso a lógica transgressiva permite tanto aos protocolos proprietários do que é livre de sobreviver num mesmo ecossistema que não é entendido de maneira reivindicatória. A pragmática de abertura na sua lógica transgressiva não sublinha a presença de códigos proprietários na rede, mas procure um 'direito' a se comunicar com eles. O ambiente maior onde estão interagindo esses códigos não é apontado como livre, mas como sendo naturalmente heterogêneo.

\subsection{Lógicas de civismo} \label{3.3.2}

Os métodos de engenharia reversa se encontram em vários projetos livres. Trata-se de 'correr atrás' de formatos de arquivos ou protocolos indispensáveis ao funcionamento básico de um sistema. Como samba procura integrar os protocolos da microsoft ao mundo Unix, open office procure "correr atrás" do formato .doc e Wine atrás das interfaces de programação de aplicativos (API) do sistema Windows para eles serem executados em cima de uma plataforma diferente. Por isso, o resultado que se apresenta ao usuário final não é uma alternativa como é o caso para gNS, mas uma interoperabilidade, um código que permita as tecnologias híbridas de comunicar. Sendo que o propósito mais fundamental das tecnologias da informação é a comunicação, tal propósito tem um impacto muito maior do que a criação de uma alternativa total. Isto é permitir os "meios-termos".

Porém, há uma proposta política forte atrás do uso da engenharia reversa, porque ela é sempre vista por seus atuantes por ser um modelo de desenvolvimento muito fundamental à inteligência humana. Quer dizer que olhar para um objeto desconhecido e entendê-lo a partir do que ele produz e relacioná-lo com o que foi colocado nele. Essa idéia é  como um "direito fundamental" do programador, como do ser humano em geral. Não ter autorização para usar essa forma inteligente é percebido como restringir uma cognição muito natural. Nesse sentido, um desenvolvedor web francês comenta:

\begin{quote}
Bom, eu sou programador web, e trabalho muito em cima da tecnologia Flash que é pouco livre [\ldots] O que eu gosto na filosofia do livre é que ela parece defender meu modo de aprendizagem contra um modelo que parece o proibir […] Eu nunca fui na universidade, aprendi tudo sozinho, e por isso, tenho falta de conceitos gerais ou fundamentais em engenharia da computação. O que eu sei fazer é ver um negocio e entender como funciona, tanto faz que seja livre ou não, a caixa preta é meu quotidiano.. assim... ha um “bixo” que faz coisas e recebe outras, como ele funciona? Como eu faço para ele tratar uma coisa ou produzir outra ? é que nem bêbê que aprende a falar, ou adolescente que aprende a trepar: isso funciona ? Sim, guardo. Isto não? jogo fora. Isto mais ou menos? preciso dar uma olhada, etc... e assim vou em frente, junto soluções, concertos para desenvolver novos saberes, ferramentas, etc...
\end{quote}

Se a interdição de tal prática, um dia, for banalizada ao mundo da computação, há a idéia que a cognição a mais básica do usuário-desenvolvedor nem poderia mais exprimir-se, no sentido de usar seu raciocínio. Por isso, o uso da engenharia reversa é percebido por seus usuários como o fato de 'pensar' um software da maneira mais fundamental. Por enquanto o DMCA considera esta prática como aceitável por fins de observação ou educação. Porém os atores proprietários sublinham o fato que muitas operações de pirataria realizadas com software (cracking, por exemplo) envolvem essas técnicas, e por isso proibi-la ou regulá-la  severamente seria um ato eficiente contra seu uso ilegal.

Assim, as comunidades livres como Samba, que praticam esse modelo de desenvolvimento defendem profundamente esse direito apontando seu valor social e político de defesa como uma modalidade primaria do entendimento humano.

\subsection{Lógicas de inversão} \label{3.3.3}

Por não ser uma alternativa, mas um esforço no sentido de incorporar as realizações do mundo proprietário, a comunidade Samba não opera uma lógica de inversão além do que a licença GPL já realiza. Por isso, as outras lógicas aqui examinadas devem ser consideradas de maior importância que essa, para entender as pragmáticas de abertura.

\subsection{Lógicas de colaboração} \label{3.3.4}

Por ser uma comunidade muito aberta às contribuições, tendo usuários desenvolvedores dos mundos livres e proprietários dentro dela, a comunidade samba encarna muito bem as lógicas extremas de colaboração que se encontram no mundo open-source. O foco das pragmáticas de abertura nas suas lógicas cívicas, a interoperabilidade das tecnologias, as lógicas de colaboração aparecem como o último propósito de uma comunidade como Samba. Além disso, a porta de entrada dada às tecnologias privadas para comunicarem com as abertas é uma realização que pode ser criticada do ponto de vista da filosofia GNU-GPL, mas que é muito incentivado na asa pragmática do movimento SL/CA.

Atrás disso, acha-se uma interpretação diferente dos limites entre o que pode permanecer fechado é o que deve ser aberto. A diferença aparece em um nível técnico, enquanto a filosofia GNU procura um 'tudo' livre, um projeto como Samba procura uma 'possibilidade' de livre, isto é a oportunidade com um código livre de comunicar como qualquer coisa. O domínio do livre é então entendido como um tudo homogêneo que possa dialogar com outros conjuntos de naturezas diferentes, e por isso ver sua pragmática de abertura acrescentada. Um exemplo relatado por um dos fundadores do projeto Samba trata de outra preocupação atual do movimento SL/CA, que é oferecer uma alternativa livre aos protocolos proprietários de VoIP (Skype, Google Talk, entre outros): “Eu espero só que eventualmente os desenvolvedores de software livre conseguem furtar na rede do Skype e inter operar com os protocolos Skype. Finalmente, há bons antecedentes para isso com outros softwares livres\ldots”\footnote{“I just hope that eventually Free Software developers can work out how to hook into the Skype network and inter-operate with the Skype protocols, after all, there are good precedents for that with other Free Software\ldots”} (ALLISON, 2005).

Como podemos observar a referência aponta a fraqueza do projeto Ekiga, alternativa livre ao Skype mal sucedida, o software o mais usado para o VoIP. Enquanto Ekiga oferece um software livre para usar seus próprios protocolos livres nos seus servidores comerciais, Jeremy Allsion propõe como solução o uso de engenharia reversa para permitir um software livre interagir com o protocolo proprietário do Skype. Vemos então que as lógicas de colaboração desenvolvidas pelas pragmáticas de abertura como se acham na comunidade promovem não um meio-termo ou uma concessão do mundo livre ao proprietário, mas uma reinterpretação do domínio do livre a respeito dos protocolos de comunicação.

\section{As pragmáticas de segurança: As comunidades BSD} \label{3.4}

As pragmáticas de segurança ilustram um foco especial dado ao desempenho de uma tecnologia computacional. De fato, dentro das preocupações de um programador, como a velocidade, interoperabilidade ou a leveza de uma tecnologia, existe também a noção de segurança. Um software, como todo 'sistema', é quebrável e tem falhas que são oportunidades para uma pessoa mau-intencionada explorar e modificar um comportamento sem se submeter ao algoritmo central. Com a difusão, em grande escala da Internet a partir dos anos 90, sublinhou-se a importância de tais assuntos de segurança para limitar o espalhamento de vírus e as explorações de redes privadas.

De fato, as tecnologias livres sempre revindicaram ser mais seguras por ser justamente abertas, isto é, que qualquer 'falha' fica mais aparente e por isso é mais rapidamente comunicada e consertada. Independentemente das pretensões do livre, e mais por desafio tecnológico, os sistemas Unix baseado nas tecnologias BSD sempre desenvolveram uma preocupação maior a respeito da segurança e estabilidade, particularmente para operar em servidores.

As comunidades BSD são as que se formaram nos arredores da distribuição de Unix desenvolvida na Universidade de Berkeley (BSD – Berkeley Software Distribution) entre 1977 e 1995. Após essa data, dividiram-se em três famílias principais, FreeBSD, netBSD e OpenBSD. Nos focaremos principalmente à comunidade FreeBSD, sendo a mais comum (77\% dos sistemas BSD em uso em 2005\footnote{Fonte: http://www.bsdcertification.org/downloads/pr\_20051031\_usage\_survey\_en\_en.pdf}) e à do projeto OpenBSD por ter projetos conexos mais produtivos (openSSH, openSSL).

Mais uma vez, reforçamos aqui que as análises propostas são baseadas em materiais encontrados principalmente no Forum\footnote{http://forums.freebsd.org/} e as mailing-lists da comunidade e nas interações no seu principal canal IRC\footnote{\#freebsd@irc.freenode.net:6667}.

\subsection{Lógicas de transgressões} \label{3.4.1}

A estabilidade que as tecnologias BSD disponibiliza aos seus usuários necessita  uma intensa atuação no campo da segurança. Por isso, muitos desenvolvedores estão vinculados aos mesmos circuitos de informação que disponibilizam as falhas de sistemas. A lógica profunda da segurança na informática responde a uma troca entre atacantes e defensores.

De um ponto de vista representativo, os atacantes querem achar falhas para explorá-las, mas querem um sistema seguro para usar e por isso procuram. Os defensores querem também achar as falhas para consertar os sistemas e por isso procuram dialogar com os atacantes. Em resumo, cada um dos partidos procura o que o outro acha, e por isso, a alta exigência de segurança dos sistemas BSD aproxima seus desenvolvedores dos campos virtuais do hacking underground.

Porém, o propósito que motiva as atuações no campo da segurança dos desenvolvedores BSD é a segurança do sistema, por isso a transgressão das tecnologias aparece como um meio indireto de realizar este objetivo.

A inversão operada no nível das lógicas de transgressão é a base do processo de tornar seguro das tecnologias BSD. O ato de transgressão que necessita buscar falhas e reproduzir integralmente pelos desenvolvedores, mas o propósito é inverso por ser destinada a correção falha. Um exemplo ilustrativo é o processo de 'auditoria' (OpenBSD code auditing) realizado pela comunidade OpenBSD em todos os software disponibilizados com a distribuição. Tal processo exige de um desenvolvedor que leia o código, procurando possiveis explorações, o que é exatamente o processo que realiza um cracker mal-intencionado para poder explorar um código. Isto vai além da simples relação de compartilhamento de informação em plataformas onde as falhas estão publicadas, trata-se, contudo, da exploração-segura para escapar da colaboração com os crackers. Nesse sentido, a comunidade openBSD argumenta que, com tal modo de desenvolvimento ela conseguiu desenvolver um sistema que não sofreu falhas de segurança durante 5 anos\footnote{Até junho 2002 o slogan de openBSD era: “Five years without a remote hole in the default install!”. Em 2007 o slogan era: “Only two remote holes in the default install, in a heck of a long time!”. Para ter uma um ponto de comparaçao, tem atualizaçoes de segurança cada mês em sistema como Linux ou Windows.} de uso. Tal preocupação determina por grande parte as lógicas de civismo (contribuições) e de colaboração das pragmáticas de segurança das comunidades BSD.

\subsection{Lógicas de civismo} \label{3.4.2}

A diversificação das atividades na Internet, particularmente as envolvendo troca de dinheiro, como é o caso das compras online, ou as consultas de dados sensíveis (conta bancária) foi permitido paulatinamente pelo uso de tecnologias de criptografia em grande escala. Umas das maiores contribuições na área foi realizada a partir da comunidade OpenBSD, cujos projetos OpenSSL e OpenSSH "liberaram" os primeiros esforços nesse sentido. De fato, o projeto openSSH é adotado em todas as distribuições de Unix, ou seja em quase todos os sistemas que não são Windows, e permite a dois sistemas de dialogar inteiramente com uma conexão protegida e criptográfica. Porém, o projeto com contribuição muito maior foi OpenSSL,  que pretende oferecer a mesma segurança para as conexões via  web. Sem dúvida, muitas das conexões que acontecem na web necessitam um tipo de proteção, como é o caso das compras ou divulgação de senha, para que os dados assim transferidos não apareçam decifrados a quem "lê" o tráfego da rede.

Há uma dimensão política e social a tais atos, próxima aos primeiros passos da criptografia. A segurança e a privacidade estão consideradas pelos atuantes dessa comunidade como uma necessidade social, uma carência. O código assim escrito é um ato político e social no sentido de cobrir essa necessidade. Tal esforço é aproveitado por toda a sociedade conectada. Uma comunicação segura o comércio online, as transações financeiras, a designada "pirataria" de conteúdo intelectual, ou seja, qualquer tipo de comunicação aproveita tais esforços. Por isso, há a defesa direta a um direito a segurança e privacidade, que define seus termos sem limites de "objetos". Isto é que essas funções no mundo político "real" pertencem em geral como função reguladora do estado que tem monopólio da segurança e definem os limites da privacidade. Por exemplo, no caso da criptografia, e das ferramentas liberadas por comunidades como as dos sistemas BSD, as possibilidades de segurança e privacidade são disponibilizadas por atores independentes que ganham sua legitimidade por ser comunidades "livres". Assim, o poder político tradicional somente pode interagir com elas por meio de restrições e nao de disponibilização ou monopólio.

\subsection{Lógicas de inversão} \label{3.3.3}

A licença BSD, por favorecer uma colaboração sem restrições incluindo a apropriação do seu código para projetos proprietários não entra nas lógicas de inversão como podem ser achadas no projeto GNU-GPL.

\subsection{Lógicas de colaboraç\~ao} \label{3.3.4}

As comunidade BSD não estão explicitamente aberta como as do Samba ou de gNS. Quando se entra no site do projeto FreeBSD não tem um link ou um anúncio explícito chamando os usuários a contribuir ou participar. Os imperativos técnicos próprios a essas comunidades exigem um alto grau de competência comum apontada como 'elitista' pelas outras comunidades livres. Os atuantes consideram que a exigência técnica atrás do projeto os proíbe aceitar a ajuda de amadores cujo trabalho aparece como ma-acabado e provavelmente fraco em frente a suas exigências de estabilidade.

Esse 'profissionalismo' atrai mais ainda os investidores privados por propor condições de apropriabilidade muito vantajosa para as empresas desenvolvendo produtos fechados. A licença BSD permite o uso de um código que ela regula sem constringir nenhuma obrigação de retorno para as comunidades. Muitos exemplos de colaboração assim determinada resultaram produtos famosos ao mercado das novas tecnologias, dentro das quais Xbox, console de jogo desenvolvida pela Microsoft a partir de um kernel FreeBSD, PSP, console de jogo portável desenvolvida pela Sony a partir de um kernel NetBSD, MacOSX, sistema operacional da Apple desenvolvida a partir do sistema FreeBSD, entre outros.

Ainda, códigos BSD são usados para projetos livres, e incorporado ao sistema Linux, por exemplo. Isso coloca as comunidades BSD como plataforma de colaboração possível entre as lógicas, isto é, as mais proprietárias como as mais livres, e os usuários-desenvolvedores gostam de sublinhar que tal característica privilegia antes de tudo as características técnicas do sistema e permite um ambiente diversificado se manter. Nesse sentido, um comentário de um desenvolvedor no Forum da comunidade FreeBSD, nos informa: “É a hora de entender que FOSS é versátil. Você não queria que Linux fosse BSD, BSD fosse MacOSX, etc. Assim sendo, podemos ter a escolha e, por definição, algo “bom para tudo” não pode ser bom.”\footnote{“It's time to understand FOSS is versatile. You wouldn't want Linux to be BSD, BSD to be OS X ... etc. This gives us CHOICE. Something that is "good for everything" cannot be good by definition.”.}. Desta forma, ilustramos uma lógica diferente na interpretação do que é o software livre e o que ele permite: A versatilidade de um meio-ambiente cuja diversidade técnica e legal permite a soluções diferentes de adaptarem-se adequadamente a necessidades diversas.

\section{Comparando e analisando as características das pragmáticas} \label{3.5}

\begin{figure}[hbt]
\caption{Quadro recapitulativo} \label{fig3.1}
%\begin{tabular}{|>{\raggedright\arraybackslash}p{3cm}|>{\raggedright\arraybackslash}p{3.5cm}|>{\raggedright\arraybackslash}p{3.5cm}|>{\raggedright\arraybackslash}p{3.5cm}|}
\begin{tabular}{|L{3cm}|L{3.5cm}|L{3.5cm}|L{3.5cm}|}
\hline
 & Pragm\'atica de liberdade (\emph{gNewSense}) & Pragm\'atica de abertura (\emph{Samba}) & Pragm\'atica de seguran\c ca (\emph{BSD}) \\
\hline
L\'ogicas de transgress\~a\c co & 0 & Engenharia reversa & Procura de falhas \\
\hline
L\'ogicas de civismo & Alternativa inteira e aut\^onoma \`as propostas n\~ao-livres & Interoperabilidade de tecnologias h\'i­bridas & Seguran\c ca-estabilidade do dispositivo \\
\hline
L\'ogicas de invers\~a\c co & Espa\c co jur\'idico pr\'oprio protegido pela mesma legitimidade legal que o dom\'i­nio propriet\'ario & 0 & 0 \\
\hline
L\'ogicas de colabora\c c\~ao & Comunidade aberta; Colabora\c c\~ao absoluta em Espa\c co herm\'etico & Comunidade aberta; Colabora\c c\~ao sem limites pr\'oprios ao SL/CA & Versatilidade; Comunidades relativamente fechadas; Explora\c c\~ao comercial unilateral \\
\hline
\end{tabular}
\end{figure}


Ao comparar as pragmáticas da programação expostas aqui (Figura \ref{fig3.1}, \pageref{fig3.1}) podemos observar que essas três comunidades interpretam diferentemente a sua participação ao movimento do software livre e que essas diferenças exprimam-se em primeiro lugar nas escolhas feitas na hora de codificar o software objeto das comunidades.

Essa diversidade técnica sublinha para seus atuantes uma representação diferente do que significa um código livre e aberto, o que motivam eles a atuarem nos projetos e, por fim, qual idéia eles tem do ambiente tecnológico maior no qual o SL/CA interage. Tais características constituem a personalidade social e política de cada comunidade e a diferença das outras. As escolhas realizadas formam um arcabouço político de uma comunidade, de seu código até sua atuação no campo das tecnologias. Que seja por volta de uma preocupação estética ou uma responsabilidade a respeito de um potencial regulador, os mesmos padrões de decisões estão colocados no processo comunitário de criação e promoção do software próprio a cada comunidade.

Além disso, alguns padrões aparecem com a leitura do quadro recapitulativo (Figura \ref{fig3.1}). Primeiramente, as lógicas de transgressões e de inversão parecem se excluir uma a outra. No caso das pragmáticas de liberdade, as lógicas de transgressões estão pouco recomendadas dentro do discurso da comunidade por ser reconhecidas como uma forma de incluir 'meio-termos' a que quer se constituir como uma alternativa comprometida a uma ideologia. Enquanto isso, as pragmáticas de abertura e de segurança desenvolvem bastante as características transgressivas dos seus modelos de desenvolvimento por permitir a inclusão do objeto transgredido no software codificado. Isto é, no caso de Samba, a comunicação com os protocolos proprietários, e no caso de BSD, a defesa contra as falhas de segurança.

Por parte essas características das lógicas de transgressões determinam as das lógicas de civismo. Enquanto a ausência de transgressão na pragmática de liberdade produz a necessidade de construção de uma alternativa entendida como absoluta, a presença de tais lógicas nas pragmáticas de abertura e de segurança sublinham a construção de um desempenho técnico mais pragmático. Isto é, no caso do samba, uma contribuição à interoperabilidade dos sistemas Unix com o mundo proprietário, e no caso de BSD, a distribuição de ferramentas seguras e estáveis para a atividade de cada um na rede.

A respeito das lógicas de inversão, podemos observar que elas caracterizam especificamente o domínio livre GNU-FSF. Isto é que as lógicas de inversão próprias às pragmáticas de liberdade como se acham na comunidade gNewSense, estão o monopólio de um software que se quer exclusivo. Por ter procurado seus meios de defesa contra o proprietário nos mesmos campos que ele, isto é, as licenças autorais, o domínio GNU que o projeto gNS vem sistematizar, procura inverter as proposta de sistemas operacionais tanto proprietárias como híbridas. 

Por fim, as lógicas de colaboração parecem oferecer um bom espelho da esfera política de cada uma das comunidades, por ser o seu resultado concreto a respeito da sua capacidade a interagir com o ambiente tecnológico de software nos seus arredores. Por isso, enquanto as pragmáticas de liberdade promovem uma lógica de colaboração utópica, isto é, absoluta, mas num espaço restrito a realizar qualquer meio-termo, as pragmáticas de abertura e segurança interagem bastante com seus ambientes respectivos. Porém essas interações se diferenciam bastante nas suas naturezas. As pragmáticas de abertura fazem da colaboração um objeto do software que desenvolvem. O objeto da colaboração é atraído ao esforço tecnológico livre para ele se tornar acrescentado dele. Diferentemente, as pragmáticas de segurança promovem uma colaboração muito forte com qualquer tipo de ator enquanto o 'profissionalismo' dos sistemas é mantido.

Observamos então que dentro do próprio movimento SL/CA interagem oposições radicais de visão política e social, em termo de meios desenvolvidos, propósitos atribuídos, contribuições realizadas e colaborações constituídas. Esse conjunto heterogêneo, porém vincula a mesma figura ao interagir com sua esfera politica maior, como movimento unido, ou pelo menos designado como tal. As suas asas reformistas, progressivas, ou revolucionárias são misturadas numa mesma entidade social recolhida pelo senso comum sob o nome genérico de SL/CA. Por isso, umas das hipóteses que possam ser emitida aqui é que o que é convencionalmente designado como movimento deveria talvez ser estudado sob as propriedades de um público. Nesse sentido, vale mencionar o esforço, talvez contraditório, das comunidades SL/CA concorrentes e opostas sobre vários assuntos, como é o caso dos domínios designados como 'livre' ou open-source, para no trato do discurso manter uma identidade comum como público diversificado, cujas interações estruturam o panorama tecnológico dos softwares. Assim testemunha o fundador do projeto gNewSense, Paul O'Malley, numa entrevista pessoal:

\begin{quote}
Os desenvolvedores de software em vários campos, isto é desenvolvedores do Linux, como de BSD, usam de licenças que respeitam as quatro liberdades. Ha muitos outros campos, mas esses dois estão suficientes para os propósitos dessa analogia. Cada um desses campos permite facilmente a cada um participar, e além de participar, de interpolar, num sentido próprio as ciências sociais, justificando às pessoas que se juntam a um campo particular que estão no campo certo graças as filosofias internas consistentes. Eles são muito similares, e somente em casos periféricos se diferenciam. Ambos declaram ter a validade absoluta, porem em função de suas diferenças, são entidades muito distintas.

Eles são como gêmeos partilhando muito ADN, portanto que são entidades do software livre legítimos embora separados. Uma prova positiva disto é quando você segue o caminho de cada um, uma comunidade tem seu trabalho colocado em MacOSX e a outra no projeto Debian, e talvez em gNS\footnote{“Software developers in the various camps, that is the developers of GNU/Linux systems, and developers in the various BSD camps use licences which respect the four freedoms. There are many other camps but these two will suffice for the purposes of this analogy. Both camps make it very easy for one to join and upon joining make it easy for one to interpolate in a social science way, justifying to the person joining their particular camp, and that most certainly they are in the right place by having internally consistent philosophies.  They are very similar, and the edge cases they differ. Both claim to have the ultimate validity, however they are by virtue of their differences very distinct entities. / They are like twins sharing a lot of DNA, in so far as they are legitimate free software entities albeit separate. Proof positive is when you get down streams of both, one community had their work put into OS X and the other into Debian project, the latter of which eventually turns into gNewSense.”}
\end{quote}

Assim, apresenta-se ao entendimento do observador não um movimento político homogêneo, mas um público diversificado cuja dinâmica constitua-se a partir de um conjunto de movimentos concorrentes, e às vezes, até opostos.  As filosofias desenvolvidas por cada campo do SL/CA são construções respectivas das projeções e representação que as várias comunidades produzem a partir das escolhas que eles realizam ao codificar seu software. Porém, um conjunto comum (o "ADN") contribuiu a promover um modo de desenvolvimento mais abstrato que é partilhado.

Todavia, qual seja híbrido ou a pretensão absoluta, uma noção das liberdades as quais devem responder uma licença são entendidas como colocadas ("as quatro liberdades"). Embora a comunidade seja "elitista" ou "educativa" em frente aos usuários-desenvolvedores, ha uma mesma idéia desses papéis, como contributivo a um esforço de desenvolvimento livre e aberto. Por fim, portanto que as aplicações feitas desses softwares possam ser radicalmente diferentes, permanecendo livre ou tornando-se proprietárias, o conjunto primário de software fica acessível nos mesmos termos para qualquer um. Esses pontos comuns constituem um público, cujas modalidades comunitárias produzem os movimentos que fazem a dinâmica política do SL/CA.

