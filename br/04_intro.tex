\chapter*{Introduç\~ao}
	

O movimento do software livre e de código aberto (SL/CA) é uma reação à privatização do código que aconteceu nos anos 1960. Desde então estruturou-se em vários ramos complexos e chegou a oferecer uma alternativa tecnológica concreta em várias áreas da sociedade de informação, principalmente no uso e a construção da internet.

Assim, várias mudanças esclarecem o sucesso do software livre, ou pelo menos sua atualidade nas políticas públicas, nos debates políticos e na literatura universitária. Primeiro, desde mais de uma década, o número de dispositivos informativos que acompanham o cotidiano dos indivíduos de classe média cresceu de maneira inédita. O celular e o computador pessoal são dois objetos comunicativos, cujas fronteiras afastam-se com a redução do consumo de energia e de tamanho junto com um melhor desempenho e estabilidade. Os recursos oferecidos tornam-se ferramentas básicas de muitas atividades que até então não aproveitavam dos efeitos de redes. Nesse contexto, o software livre encontra um sentido no senso comum como desafio técnico, permitindo a participação dos usuários além do que permite as plataformas proprietárias.

Ainda, uma das idéias que circula nos debates comuns sobre os programas de código aberto é a idéia de uma comunidade sem alfândega, cujo acesso é definido pela simples participação de quem a deseja. A imagem do usuário-desenvolvedor aproxima-se então da idéia de um cidadão que participa de um movimento coletivo: o movimento tecnológico. O “\emph{You}” (Você), pessoa do ano em 2006 segundo o jornal TIME, representa bem essa corrente contemporânea que coloca o indivíduo como atuante decisivo da dinâmica social no mesmo tempo que o funde inteiramente a comunidade a qual ele participe. Nesse sentido, o movimento do software livre é um exemplo que dá muito sentido a essa “modernidade”, mostrando realização como a Internet, possibilidades como a participação política direta, e perguntas como o modelo econômico de apropriação dos saberes.

Enfim, por qualquer direção que for o movimento tecnológico atual, ele é influenciado por seus atuantes, e uma maioria de pessoa no mundo não tem acesso à rede e as ferramentas tecnológicas básicas. É nos países periféricos que a fratura digital é a mais aparente, as classes médias têm acesso às tecnologias de informação enquanto as classes baixas acumulam atraso na educação digital e não participam nem aproveitam dessas evoluções, embora elas pareçam estruturar profundamente a sociedade contemporânea. Nesse contexto, o software livre aparece como uma solução ótima para as políticas públicas de inclusão digital, sendo que permite oferecer tecnologias com custos reduzidos e potenciais de adaptabilidade maior.

Nesse trabalho, a análise foca-se sobre os aspectos culturais do movimento e principalmente seus determinantes políticos. Além disso, as comunidades “livres” apresentam-se sem discurso ou agenda político precisa, desenvolvendo então uma cultura independente como ramo da cultura “hacker”. Para tanto, argumenta-se que embora esse “agnosticismo político” é característico do movimento SL/CA, ele é a face de uma prática social intensa, informal e normativa no contexto informacional, que constrói a aura política do movimento com um conjunto. Assim, a hipótese explorada nessa pesquisa é que a prática da programação entendida como pragmática, é o vetor constitutivo e determinante das práticas políticas informais e de suas conseqüências sociais. Assim, a seguir explora-se como esse estudo está sistematizado.

No primeiro capítulo, busca-se apresentar uma história breve da emergência, constituição e consolidação do movimento SL/CA no mundo. Voltada às origens do objeto software na história do início da computação, mostra-se ainda que as distinções “naturais” feitas entre software e hardware podiam ser interpretadas diferentemente antes que companhias, como a Microsoft ou a AT\&T, venham isolar um objeto “software” e fechar seu código para disponibilizar um produto finito com opções de configuração e de apropriação limitada. 
Nesse contexto, o movimento SL/CA aparece como uma reação, ou seja, um esforço de manter uma tradição de livre interação no processo de inovação e de construir uma alternativa ao modelo proprietário que se fortalece com a produção em grande escala de computadores pessoais. Por dentro, o movimento tem estruturas comunitárias, jurídicas e institucionais complexas que tornam difícil delimitar uma linha de ação comum, além de querer promover a necessidade de um código fonte aberto para os programas. Contudo, a divisão institucional entre software “livre” e software “aberto”, que acontece publicamente em 1998, permite entender melhor as lógicas econômicas e ideológicas atrás de diversos ramos do movimento global, e assim delimitar seus desafios contemporâneos.

Mais adiante, já no segundo capítulo, tenta reconstruir-se uma parte do esforço teórico feito pela literatura universitária em cima dos aspectos culturais do movimento SL/CA e da cultura hacker. Assim, a noção de ética hacker, entendida como construção social, permite isolar várias características dos atuantes do movimento do software livre. Porém, seguindo as reflexões da antropóloga Gabriella Coleman (2008) que afirma que o conceito de ética hacker fica ainda preso ao binário moral midiático, que se constituiu por volta da figura do hacker (Pirata ou Herói da era digital). Assim, há um intento para isolar o discurso político aparente como um agnosticismo político, cuja função é de permitir à paixão pela informação dos hackers de se exprimir livremente. Essa prática estrutura suas políticas informais, tecnológicas e práticas que tentaremos reduzir a uma tipologia de tipos ideais.

Ainda no segundo capítulo, apresenta-se o paradigma teórico de pragmática da programação; construído para interagir com a pesquisa técnica e bibliográfica. Aqui alertamos que o ato de escrever código acontece num contexto de padrões e expectativas que determinam sua estética. Isso acontece, comparando o fato de programar como o de realizar arte, lei, ou arquitetura física a partir de referências universitárias. Desse modo, mostra-se o código como um discurso normativo e político sobre a informação e seu tratamento pela máquina e a rede. De forma geral, a idéia é de mostrar que a pragmática da programação é o vetor do “agnosticismo político” do movimento pelo fato de ser o vetor das suas políticas informais, e que as lógicas do código que afastam as referências políticas tradicionais são as que articulam o discurso normativo do movimento sobre a informação. 

No terceiro e último capítulo, propõe-se, então, uma tipologia de três pragmáticas emblemáticas de vários projetos open-source, que tem como propósito interagir com a tipologia das políticas informais do movimento SL/CA. Assim, as lógicas de transgressão, de tecnologia cívica, de inversão e de colaboração são reinterpretadas pelo vetor das pragmáticas de liberdade, abertura e segurança. Com análise de estudos de casos, referências técnicas e conceitos teóricos encontrados no esforço tecnológico do movimento do software livre, estabelece-se padrões de interpretações dos contextos político-tecnológicos pelos próprios atuantes e pelas comunidades.
