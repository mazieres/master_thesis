\chapter*{Résumé}

Ce mémoire cherche à présenter les significations politiques et culturelles d’un Mouvement du Logiciel Libre considéré comme un ensemble hétérogène de communautés et de projets. De plus, à partir d’un historique de l’objet « Logiciel » depuis son origine, nous montrons qu’il a été différencié du matériel (\emph{Hardware}) et, ainsi, fermé comme un produit fini par les entreprises de softwares naissantes. Dans ce contexte, le Mouvement du Logiciel Libre se présente autant comme une réaction au phénomène de « blackboxing », que comme une continuation de la tradition de libre-échange d’informations au sein de l'ingénierie informatique. Pour cela, il se structure à travers plusieurs aspects de l'éthique Hacker et de son agnosticisme politique pour construire une alternative technologique concrète. Ainsi, nous pouvons affirmer que les caractéristiques socio-politiques des communautés du logiciel libre doivent être recherchées dans l’acte même de la programmation, dans sa pragmatique, en tant qu’art ou régulation. De cette manière, nous étudions les cas spécifiques de plusieurs communautés (gNewSense, Samba, BSD) pour tenter de systématiser leurs positionnements techniques et socio-politiques envers le mouvement technologique contemporain.

\vspace{2cm}

\begin{tabular}{ll}
\textbf{Mots-clés} & Logiciel Libre;\\
& Hacker;\\
& Programmation (ordinateur) -- Aspects politiques;\\
& Programmation (ordinateur) -- Pragmatique.\\
\end{tabular}

