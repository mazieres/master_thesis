\chapter{Aspects sociaux du mouvement du logiciel libre} \label{2}

Ce chapitre a pour but de présenter les éléments politiques et sociaux du mouvement du Libre présent dans la littérature universitaire développée à son propos et d'en proposer une typologie. Après avoir présenté les termes utilisés pour désigner les actants du mouvement au sein des communautés et dans les sens commun et universitaire (\ref{2.1.1}), nous aborderons le mouvement à travers le concept d'éthique hacker (\ref{2.1.3}), entendu dans un contexte théorique de construction sociale de l'éthique (\ref{2.1.2}). Une critique constructive de ce concept permet de réinterpréter les significations culturelles du hacking (\ref{2.2}) en usant de la notion "d'agnosticisme politique" (\ref{2.2.1}). De cette manière, l'intéret est porté sur la passion pour l'information comme matrice minimale des interactions de nature politique des utilisateurs-développeurs, permettant alors de construire une typologie de celles-ci à partir des travaux de Gabriella Coleman (\ref{2.2.2}).

Ainsi, se construit un paradigme d'interprétation de l'activité de l'utilisateur-développeur de logicel libre à travers l'idée de pragmatiques de la programmation (\ref{2.3}). Après avoir présenté l'acte de programmer dans sa dimension artistique et régulatrice (\ref{2.3.1}) nous pourrons utiliser des pragmatiques de la programmation comme une interprétation sociopolitique d'un concept linguistique, permettant de comprendre la relation des utilisateurs-développeurs avec leurs politiques informelles et l'information (\ref{2.3.2}).

\section{L'\'Ethique Hacker} \label{2.1}

\subsection{Hackers, utilisateurs-développeurs, geeks et nerds} \label{2.1.1}

De nombreux termes désignent les personnes qu'on trouve autour du logiciel libre. Cependant, des significations supplémentaires de chacun d'entre eux pointent des catégories plus spécifiques ou générales d'individus. Pour cette raison, nous présenterons, dans le détail, quatre des appellations qui ont été rencontrées fréquemment au long de notre recherche bibliographique et de terrain.

\subsubsection{Hacker} \label{2.1.1.a}

"Hacker" est probablement le terme le plus romancé à propos de la culture informatique. L'industrie cinématographique hollywoodiene a, d'ailleurs, déjà profité des recours cognitifs qui se trouvre autour de ce domaine : un "ego illimité" exprimé dans un monde de 0 et 1, qui affronte un réalité manipulée, tronquée, décevante, limitée\ldots La trilogie des frères Wachowski\footnote{\emph{The Matrix} (1999), \emph{The Matrix Reloaded} (2003), \emph{The Matrix Revolutions} (2003), de Larry and Andy Wachowsky.}, \emph{Cypher}\footnote{Vicenzo Natali, \emph{Cypher} (2002)}, \emph{Pi}\footnote{Darren Aronofsky, \emph{Pi} (1998)}, développent cette même idée, sans compter les innombrables productions de second plan, dont la liste ne peut prétendre à l'exhaustivité\footnote{\emph{Wargames}, de John Badham (1983); \emph{Hackers}, de Iain Softley (1995); \emph{Pay Check}, de John Woo (2003); \emph{Live Free or die hard}, de Len Wiseman (2007), entre autres.}.

Ce terme, Hacker, est presque toujours présenté comme dérivé du terme anglais \emph{Hijacker}, pirate. Moins fréquemment, on souligne sa relation étymologique avec le verbe "couper" des langues germaniques et aussi anglo-saxonnes. Alors que l'étimologie anglaise fait référence au sens très médiatique d'un bandit exploitant les systèmes d'informations et les réseaux (\emph{Cracker}), l'étimologie germanique ouvre des horizons plus amples de significations. De fait, il s'agit de "couper" les chemins déja connus pour trouver des astuces, des solutions, des hypothèses pour un problème, qui soient originales, non-évidentes et élégantes. C'est dans ce sens que le terme \emph{hack} a été utilisé la première fois, au sein des laboratoires du club du \emph{Tech Model Railroad} (TMRC) du MIT, dans les années 1950. Ainsi, on nommait "\emph{hack}" les modifications intelligentes qui se faisaient aux relais électriques. De même, dans les années 1960, on faisait référence à l'acte d'hacker (\emph{hacking}) au sein du laboratoire d'intelligence artificielle de la même université, où a été créé le projet GNU. On désignait alors le fait d'aller chercher le code source pour améliorer et adapter un logiciel à ses propres besoins. Dans ce sens, "hacker" est particulièrement lié avec le mouvement du logiciel libre. Les entreprises brésiliennes d'informatique comme Google Brasil, Microsoft Brasil, IBM Brasil, ont toujours fait référence au "déchiffreur" (\emph{Decifrador}), en respect pour la langue portugaise.

\subsubsection{Geek et Nerd} \label{2.1.1.b}

Geek est une expression anglaise dérivée du mot \emph{geck}, idiot (\emph{fool}) et étrange/marginal (\emph{freak}). De même, le terme de geek a des origines avec le mot \emph{geck} du germanique ancien, signifiant fou et avec les dialectes français, \emph{gicque}, signifiant "fou de carnaval"\footnote{Source : Oxford American Dictionary}. Il s'agit d'un personne étrange, ou au moins étrangère, particulièrement une étant perçue par les autres comme étant obsédée par une plusieurs choses, parmi lesquelles, l'intellectualité, l'electronique"\footnote{Dictionary.com}. Nous pouvons qualifier la culture \emph{geek} comme un mouvement culturel populaire typique des sociétés contemporaines de l'information. De plus, les geeks sont qualifiés comme ayant des intérêts exhaustifs et créatifs sur des sujets liés aux nouvelles technologies [Konzack, 2006]. Pour cela, le logiciel libre peut être considéré comme un domaine de la culture geek et les individus y participant, comme des geeks. Ainsi, l'acte de programmer s'associe parfaitement aux exigences d'exhaustivité, de créativité et de changement présentes dans la culture geek.

L'anthropologue Christopher Kelty, dans son étude des significations culturelles du \emph{free software}, fait constamment référence aux \emph{geeks} comme étant les actants principaux du mouvement \citep{Kelty2008}. Pour l'auteur, il s'agit d'une convergence philosophique du côté du transhumanisme. Le transhumanisme est une modalité de croyance en la ligne du temps du progrès technique, laquelle permettrait de dépasser le corps humain. Pour les transhumanistes, l'intervention technique a un rôle spécifique qui la différencie de l'intervention politique, culturelle ou sociale. Dans ce contexte, il n'y a pas de rhétorique, mais une hypothétique "vérité pure" du "ça fonctionne!". \'A l'extrême et pour utiliser une métaphore propre au monde de la programmation : "les mauvaises idées ne se compileront pas"\footnote{Un programme est "compilé" quand une série de commande et d'instructions deviennent une application exécutable.}.

Bien qu'on trouve une connotation de "bizarre" autour du terme \emph{geek}, il semble que la connotation péjorative s'attache au mot \emph{nerd}. De fait, \emph{nerd} désigne une personne avec de hautes capacités intelectuelles et ayant des intérêts hors des normes sociales communes. Ainsi, on se concentre davantage sur les signifiants de deviance, d'étrange, de hors-normes. Dans le sens commun, il semble que la figure du \emph{geek} se construise dans le contexte d'un spécialiste autodidacte, alors que celle du \emph{nerd} désignerait un adolescent boutonneux. Néanmoins, ces différences semblent apparaître quand on cherche à créer des catégories, mais le conflit geek-nerd n'a pas de sens pour ses acteurs, et chacun peut nommer l'autre de l'une des deux nominations sans avoir une identité très différente. Ce terme peut encore être utilisé, comme l'est geek, pour désigner les acteurs du mouvement du logiciel libre. Par exemple, le \emph{nerd} et l'idéal-type du hacker utilisé par Paul Graham dans \emph{Hackers and Painters} \citep{Graham2004}.

\subsubsection[Utilisateur-développeurs} \label{2.1.1.c}

Enfin, le terme d'utilisateur-développeur est le seul revu ici qui est spécifique aux communautés du logiciel libre. Il désigne des individus qui, autour de ces communautés, utilisent des logiciels libres et par cette utilisation en arrivent à aider au développement du propre software. Cette aide peut être minimale ou presque symbolique, en acceptant de renvoyer des "\emph{crash report}"\footnote{Rapports de disfontionnement} ou de répondre à des questionnaires. Il peut aussi s'agir de travaux plus majeurs, comme la réalisation de documentations, la programmation de modules additionnels ou du développement du propre software.

La figure des utilisateur-développeurs est très présente dans les projets libres. Ils constituent la "communauté" qui défini ce modèle de développement comme collaboratif. C'est probablement par la reproduction de cette figure à d'autres domaines de production que se perçoit le mieux l'influence que peut avoir le mouvement du logiciel libre hors de ses frontières. Il s'agit, par exemple, des rôles distribués de "consommateur-producteur" de l'économie collaborative (Wikinomia) et de "lecteur-rédacteur" dans le production de contenu dans le Web 2.0, qui sont intimmement liés à l'imaginaire du logiciel libre et à la figure de l'utilisateur-développeur \citep{Tapscott+Williams2008}.

\subsection{La construction sociale de l'éthique} \label{2.1.2}

Une des analyses assez développées à propos des aspects culturels du mouvement du logiciel libre et de la culture hacker est celle de l'éthique. Ainsi, l'\emph{éthique hacker} est un paradigme sufisamment flexible pour intégrer la multitude de cas qui forment les communautés autour de chaque logiciel. Une démonstration de ceci est que si les hackers et les utilisateurs-développeurs ne s'alignent pas exactement sur les mêmes réalités, la culture hacker est un des points communs à toutes les communautés d'utilisateurs-développeurs.

Pour donner à cette notion la flexibilité désirée, on doit écarter la conception kantienne de l'éthique selon laquelle la loi morale ne vient en aucun cas de l'expérience empirique, mais de ses impératifs catégoriques, pré-établis : "Agis de telle manière que la maxime de ton action puisse devenir le principe d'une législation universelle". Il s'agit pour nous de prendre en compte la \emph{vie sociale de l'éthique} dans le sens des théoriciens sociaux comme Bakhtin \citep{Bakhtin1984, Bakhtin1993}. Ainsi, les règles de vie et de connivence se construisent de manière interactive et dynamique, à partir des pratiques sociales et des expériences vécues dans le cadre communautaire. De cette manière, l'éthique peut se maintenir seulement si on peut observer une relative stabilité des pratiques sociales et une communauté de sens autour des expériences vécues. L'analyse se concentre, alors, sur les évènements qui font les transformations et ainsi, on souligne l'importance d'une auto-modélation communautaire, c'est à dire, du rôle décisif du contexte social, historique et économique.

Dans cette perspective, l'idée reste simple et déjà développée par de nombreux théoriciens \citep{Galison1997, Good1994, Guterson1996}, car il s'agit d'exclure les excès de particularismes comme ceux d'universalismes et de rester concentrés à l'observation de faits concrets. Ainsi, les individus adoptent des valeurs et réalisent des choix moraux à travers des actions qui sont influencées par des institutions ou des technologies. C'est cela que nous pouvons apprendre des mots de Coleman et Hill : "Les expériences sociales qui viennent en faveur ou contre une pratique sociale, modèlent la nature des relations individuelles et déterminent les orientations éthiques d'une communauté" [Coleman et Hill, 2005]. En d'autres termes, l'utilisation et l'organisation de cette éthique interagissent avec la communauté ou le groupe à des échelles économiques, politiques et sociales. Au niveau économique, il peut s'agir de production d'un bien, d'un modèle de développement ou d'organisation. Au niveau politique, cela peut être un agenda politique et des manifestations publiques. Enfin, au niveau social, un réseau et des noeuds d'individus et groupes qui s'étendent et traduisent leurs normes au contact de nouveaux domaines et groupes. Pour nous permettre de faire le lien entre modèle de développement participatif, quelques crimes électroniques, les conférences internationales sur l'open-source, Internet, etc, l'\emph{éthique hacker} est une notion heuristique à la compréhension du phénomène du logiciel libre.

\subsection{La notion d'éthique hacker} \label{2.1.3}

On considère, dans la littérature universitaire, que l'expression de \emph{éthique hacker} est utilisée la première fois par le journaliste Steven Levy, dans une des premières oeuvres reconnues sur les cyber-communautés : \emph{Hackers, heroes of the computer revolution} \citep{Levy1984}. \'A la différence des définitions utilisées rapidemment après, Levy n'hésite pas a donner de l'emphase à son contenu. Selon lui, l'éthique hacker est un type de prédecesseur moral au logiciel libre et aux communautés open-source. En accord avec cette définition, l'accès à l'information par le hack est totale et tout accès fait découvrir une partie du monde et pour cela se doit d'être absolu, sans limites. \`A cette fin, l'information et sa circulation doivent être libérées des contraintes institutionnelles en défiant les autorités établies et en restant méfiant par rapport aux légitimités externes à la communauté. Dans ce sens, il est nécessaire de privilégier les formes décentralisées d'organisation et, dans ce contexte, les hackers doivent être appréciés en fonction de leurs qualités techniques et créatives et non pour n'importe quel critère social (race, âge, sexe). Enfin, il y a l'idée que le renforcement technique et pratique des technologies informatiques contribuera à un monde meilleur, une meilleure organisation de la société, plus répartie, juste, démocratique.

\`A propos des "vrais hackers" que Steven Levy voulait identifier, nous trouvons : John MacCarthy, Bill Cosper, Richard Greenblatt, Richard Stallman. Ensuite, l'auteur décrit une "deuxième génération" de hackers, les \emph{hardware hackers}. Il s'agit des grands noms de l'informatique personnelle, dont une partie participe aux dèbuts de la Silicon Valley : Steve Wozniack, Steve Jobs, Bill Gates, Bob Marsh, Fred Moore. Enfin, une troisième génération apparaît à l'époque des premiers jeux interactifs, les \emph{game hackers} (John Harris, Ken Williams, entre autres).

Néanmoins, en se concentrant sur les externalités gagnantes des communautés informatiques (logiciel, matériel, jeu), l'auteur de \emph{Hacker} semble passer à côté du phénomène social des groupements de hackers comme un tout. De fait, en 1984 -- date de la publication du livre -- les pratiques hackers dépassaient largement les cadres industriels et contribuaient à la publication d'une immense quantité d'informations qui, à leurs niveaux respectifs, allaient permettre de participer/profiter de ce progrès technique, de l'ère de l'information et des réseaux mondiaux. Dans le cas du logiciel libre, attribuer à Richard Stallman ou à Linus Torvalds la paternité du système GNU/Linux n'est pas seulement une erreur politique, d'oubli des masses d'ingénieurs qui ont contribué aux projets, mais écarte aussi l'observateur ou le lecteur des intérêts, liens et valeurs qui ont motivé des milliers de hackers du monde \emph{online} à travailler ensemble. Ceci nécessite d'accepter comme matériel d'analyse et de témoignage de cette époque les listes d'emails, les forums, les wikis, les manifestes, les débats, les flux d'inforamtions qui ont fait la culture hacker et structuré son éthique.

De cette manière, le philosophe Pekka Himanem a développé une notion d'éthique hacker sensiblement différente. En faisant référence aux travaux de Weber sur l'éthique protestante et le capitalisme moderne, il oppose les valeurs hackers aux valeurs protestantes.
Selon lui, l'éthique hacker se rapproche plus des définitions de l'\emph{activité} comme elles se trouvent dans les oeuvres de Platon ou Aristote : un travail d'excellence autour d'une recherche de passion, de bonheur et de créativité \citep{Himanem2001}.

Les analyses de Levy, comme celles d'Himanem, cependant, semblent se cristalliser autour d'un contre-poid au développement péjoratif à l'image du pirate informatique et, dans ce sens, on trouve des travaux encore plus explicites, essayant de "sauver" une réputation ternie \citep{Best2003a, Hannemyr1999}. De manière générale, les conceptions du hacker et du hacking auxquelles ces travaux tentent de s'opposer sont, d'un côté, celles d'adolescents obsédés par Internet et par l'obtention de savoirs dangereux et/ou interdits \citep{Borsook2001, Slatalla+Quittner1996}, et d'un autre côté, des tournois audacieux d'intrusions dans des systèmes privés \citep{Schwartau2000}. De fait, il semble que les études jusqu'alors citées s'en tiennent à interagir avec les présupposés existant à propos de la figure du hacker et de son rôle dans la société d'information, sans jamais se demander ce qui fait la particularité de ce groupe en soi. Dans ce sens, l'anthorpologue Gabriella Coleman, dont les travaux font référence tout au long de notre étude, affirme : "La littérature sur les hackers, donc, a tendance à enfermer l'acte de hacker dans un binaire moral au sein duquel les hackers sont soit encesés, soit rabaissés. Cette tendance menace d'isoler plus que d'éclaircir la signification culturelle de hacking informatique"\footnote{“The literature  on hackers, thus, tens to collapse hacking into a moral binary in which hackers are either lauded or denounced. This tendency threatens to obscure more than it reveals about cultural significance of computer hacking.”} \citep[p.256]{Coleman+Golub2008}.

\section{Les significations culturelles du hacking} \label{2.2}

\begin{quote}
Ramenée à sa plus simple expression - et abstraction faite de son contenu – l'éthique hacker a son équivalent dans la formule "l'art pour l'art". L'important içi est de saisir que, contrairement à l'activisme politique, l'objet de l'activité hacker, la connaissance et l'exercice de la curiosité, est intérieure à son sujet \citep{Reimens2002}.
\end{quote}

\subsection{L'agnosticisme politique du mouvement du logiciel libre et de la figure du hacker} \label{2.2.1}

Il est évident que de nombreux projets ont été inspirés par la philosophie du Libre. Dans le domaine du journalisme, ont été mis à disposition des archives entières (ex : BBC) et s'est développée la figure d'un lecteur participatif -- le lecteur-rédacteur -- qui construit et discute les flux informatifs, particulièrement dans les tendances du web 2.0. Dans le domaine de la loi, on trouve de nouvelles formes contractuelles pour réguler les productions intelectuelles et artistiques inspirées par les licences libres (\emph{Creative Commons}). Dans l'éducation, des cours sont mis en ligne comme c'est le cas, au MIT, de l'\emph{Open Course Ware}, et ceci sur des plate-formes technologiques libres. Ces projets sont tous liés à un contexte de réseau et d'interconnexion caractéristique de l'Internet et de la société d'information. Cependant, comme l'affirme Christopher Kelty, les significations culturelles du logiciel libre vont au-delà du simple diagnostic de la société d'information et prouvent une réorientation plus spécifique, pour autant qu'elle a à voir avec des pratiques techniques et légales détaillées et spécifiques. De même, il s'agit d'une réorientation plus générale car culturelle et pas seulement économique ou légale. Une preuve de ceci est qu'avec Internet, la gouvernance et le contrôle de la création et la dissémination du savoir a changé considérablement, opérant ainsi une "réorientation du pouvoir et du savoir" qui questionnent profondément la pertinence et la légitimité du système de propriété intellectuelle \citep{Kelty2008}.

Dans ce contexte, bien qu'il soit clair que la vie politique du code libre et ouvert soit déjà avancée du fait  de ses activités et influences, la figure du hacker et des actants du mouvement refuse n'importe quelle affiliation politique. C'est ceci que note Gabriella Coleman en disant : "Alors qu'il est parfaitement acceptable et conseillé d'avoir un groupe sur le logiciel libre à une conférence anti-globalisation, les développeurs  de logiciel libre recommandent qu'il soit inacceptable de revendiquer que le libre a pour but l'anti-globalisation, ou à ce propos, n'importe quel programme politique."\footnote{“While it is perfectly acceptable and encouraged to have a panel on free software at an anti-globalization conference, FOSS developers would suggest that it is unacceptable to claim that FOSS has as one of its goals anti-globalization, or for that matter any political program.”} \citep[p.507]{Coleman2004a}.

Il semble adéquat d'observer ici que cet \emph{agnosticisme politique} peut s'exprimer à travers diverses défenses contre les tentatives de donner un sens politique à un acte, une production ou une situation. Dans le domaine de la sécurité informatique, le critère principal est la perfection du système de protection. Pour cela les considérations d'ordre idéologique sont écartées, tout comme le fait pour une technologie d'être libre ou non, au profit de critères comme : l'ouverture code, des mises à jour rapides et une certaine stabilité de la technologie. Dans ce sens, on peut citer le commentaire d'un participant du canal \emph{\#hack} du serveur IRC Freenode, lors d'un débat sur le logiciel libre : "ce n'est tout à fait "peu importe"\ldots Si les outils sont ouverts, tant mieux ! on travaille tous sur des systèmes ouverts en général [FreeBSD, Linux]; ce qui est important c'est d'avoir la technologie la plus sûre, la plus configurable et contrôlable. C'est ça qui fait qu'une technologie est bonne pour un boulot de sécurité, pas le fait qu'elle soit libre ou pas."\footnote{\#hack@irc.freenode.net, 10 novembro 2008}. Ainsi le site \emph{insecure.org}, qui tient à jour une liste des 10 meilleurs outils de sécurité informatique\footnote{http://sectools.org/}, commente à peine le type de licence qui accompagne les logiciels. Les critères privilégiés sont le prix (gratuit ou non), la compatibilité avec les différents systèmes (Linux, windows, OSX, BSDs) et l'accès au code source. Le logiciel qui arrive en tête de liste depuis des années, Nessus, un outil pour identifier des failles de système, est payant et de code fermé.

Dans diverses activités informatiques, parmi lesquelles l'administration réseau, l'analyse de système d'information, la programmation Web, le refus de l'idéologie se manifeste par exemple à l'encontre du chef de projet, dont les capacités techniques sont fréquemment réduites et influencées par les effets de mode. Tout au long de notre travail, nous avons rencontré de nombreux témoignages dans ce sens, particulièrement dans le domaine du développement web, où les programmeurs cohabitent avec des supérieurs tournés vers la communication et le marketing publicitaire. L'idéologie apparaît, alors, comme un moyen pour l'ignorant de la technique, de se familiariser avec les limitations du concepteur. Ainsi, il construit un discours permettant de justifier d'options technologiques, de budgets, de délais, entre autres. De cette manière, on arrive à une situation où le responsable de projets, souvent dans une situation hiérarchique ascendante, réalise des choix à partir d'informations tronquées que les concepteurs vont devoir suivre sans pouvoir se demander si ce sont les plus aptes à garantir le meilleur service. Nous pouvons identifier un tel phénomène dans le témoignage de Marcelo, administrateur réseau dans une grande université publique brésilienne :

\begin{quote}
On devait programmer un truc pour la SOFTEX\footnote{Association pour la promotion de l'excellence du logiciel brésilien (\emph{Associação para Promoção da Excelência do Software Brasileiro} -- SOFTEX}\ldots Là, réunion dans le bureau du coordonnateur du projet acompagné de son étudiant qui "comprend l'informatique". L'idée est claire, simple, il n'y a pas de problème jusqu'au moment où le mec nous dit qu'on doit programmer le truc en Java, parce-que Java c'est le super-langage qui va gagner de tous les autres, parce que il y a la machine virtuelle super-mega-adaptable, blahblahblah\ldots Tous croient aux effets de mode, ça leur donne des airs de spécialistes\ldots alors\ldots Je réponds que Java n'est pas la meilleure solution pour ce projet\ldots alors là commence la machine à propagande TI [Technologie de l'Information], et pour lui donner un appui, l'étudiant-qui-comprend-l'informatique dis oui à tout : Java est la meilleure solution dans 100\% des cas ! \ldots alors je défie le petit étudiant de me dire si c'est possible de programmer telle chose en Java\ldots ils pensent, tentent de changer de sujet, etc, etc, et alors, conclusion brillante du gros gars : Si c'est pas 100\%, c'est 99\% ! \ldots\footnote{Devíamos programar o troço para a SOFTEX\ldots Ai reunião no escritório com o coordenador do projeto acompanhado de seu estudante que “manja de informática”. A idéia é clara, simples, não tem problema até o cara falar para gente que deve programar o negócio em Java porque Java é a super linguagem que vai ganhar de todos os outros, porque tem a 'máquina virtual' super mega adaptável, blábláblá\ldots Todos acreditam na moda, dá a eles ares de experto\ldots Então\ldots Eu respondo que Java não é a melhor solução para esse projeto\ldots E ai começa a máquina a propaganda TI, e para dar um suporte, o estudantezinho “que manja de informática” apoia todas, Java funciona em 100\% dos casos\ldots Ai, desafio o estudante de me dizer se é possível de programar tal coisa em Java\ldots Eles pensam, tentam mudar de assunto, etc, etc e ai conclusão brilhante do gordo: então se não for 100\%, é 99\%\ldots}
\end{quote}

En racontant cet évènement, l'informateur donne un exemple de situation où l'idéologie ("la propagande TI") vient faire préjudice à son travail, et l'oblige de réaliser une tache avec des options technologiques non-optimales, cela pour des raisons qui paraissent "absurdes"; c'est à dire, des "croyances" de quelqu'un avec une position hiérarchique supérieure et une connaissance technique incomplète et influencée par les effets de mode. Ici, le "mec de SOFTEX" doit probablement baser son choix du langage de programmation Java sur les tendances internes de son institution de promotion du logiciel brésilien ("Java est vu comme le meilleur langage, alors un programme écrit en Java est meilleur") se désassociant ainsi du raisonnement logique tourné vers l'efficacité de la technologie.

Bien que cet exemple illustre une polémique technique qui ne puisse pas être directement qualifiée de "politique", il semble que c'est la même logique qui détermine la conscience politique de ces actants. Ainsi la figure du responsable apparaît comme liée aux logiques de ses propres intérêts, ou du moins non à ceux des concepteurs. De l'autre côté, l'exigence d'excellence technique crée une rationalité qui protège les concepteurs du politique. Dans ce sens, le mouvement du logiciel libre ne prétend pas gagner les faveurs d'une politique publique ou rendre favorable un décision, mais continuer à évoluer en toute autonomie, pour se dédier au meilleur développement technologique possible.

Ces observations permettent de mettre en relief le fait que les littératures anarchiste et libérale font référence au mouvement du logiciel libre comme exemplaire de leurs modes respectifs de production et d'organisation. D'un côté, les analyses libérales trouvent ici un exemple concret de "main invisible", d'absence d'intervention de l'\'Etat ou de n'importe quelle autre autorité externe aux processus de décisions internes. Les auteurs développent néanmoins une conscience propre de leur environnement, par un processus libre, et parviennent à prendre des décisions, pour faire des choix tournés vers leurs intérêts qui, joints, créent et maintiennent la communauté du logiciel libre. D'un autre côté, on trouve des auteurs qui voient ici la réalisation concrète de formes anarchistes d'organisation \citep{Moglen1999, Gross2007}. On souligne alors la haute productivité réelle de l'autogestion des travailleurs, et les racines logiques du défi à l'autorité. De cette manière, en créant autant une alternative possible au néolibéralisme et à l'anarchisme, qu'une reformulation de ceux-ci, l'agnosticisme politique du mouvement du logiciel libre et de la figure du hacker protège son exigence d'excellence technique de directions politiques prédéterminées.

\subsection{L'information comme passion et la typologie des politiques informelles} \label{2.2.2}

En essayant de trouver l'essence du hacking, c'est à dire, un point fixe d'analyse raisonnablement indépendant des pré-supposés du sens commun, l'anthropologue Gabriella Coleman va se concentrer sur la relation que maintiennent les hackers avec l'information. Ainsi, un point commun entre tout les utilisateurs-développeurs, hackers et nerds, est l'amour donné à la liberté de l'information et à sa libre circulation \citep{Coleman2003}. Il s'agit d'un sentiment très fort, maniaque et esthétique, proche de \emph{l'amour fou} décrit dans les oeuvres de l'anarchisme ontologique, comme un moyen de se réaliser entièrement comme individu et de défier les structures de la société \citep{Bey1985, Bey1991}. Selon les mots de Bruce Sterling, les hackers sont "possédés non seulement par la curiosité, mais par une véritable luxure du savoir" \footnote{Cité dans \citep{Coleman2003}}. Dans une recherche de l'origine d'une telle passion, Gabriella Coleman partage son expérience :

\begin{quote}
Mon expérience avec le logiciel libre défend ce principe. L'esprit d'exploration qui forme les bases du hacking doit commencer par démonter la mixture familiale, au grand dam de la mère: ensuite amène à apprendre à programmer à 5 ans, pour la joie de l'union parentale, ensuite se transforme en s'enfermer dans sa chambre pour lire tout les manuels d'informatique, lesquels sont confondus par les parents avec de l'inquiétude pré-adolescente ; ensuite, il s'agit d'apprendre toutes les entrées et sorties de ce système d'exploitation appelé Unix, en découvrant tout les traits typographiques et temporels de l'Internet, au grand étonnement de l'anthropologue ; et, enfin, passer son temps à contribuer, en écrivant du code pour des projets ouverts, souvent encore, en dépit de la consternation des parents\footnote{“My experience with free software support this fundamental tenet. The spirit of exploration that forms the basis of hacking might start by taking apart a household blender, much to a mother’s horror: then lead to learning how to programme at the age of 5, much to the delight of the parental unit; then transform into locking oneself in the bedroom to read every computer manual, wich parents duly confuse with pré-teen angst ; then learning every last topographical and temporal feature of the Net much to the amazment of the anthropologist ; and finally to volunteering their time to code on fre software projects, often to the dismay, again, of their parents.”}\citep[p.298]{Coleman2003}.
\end{quote}

Après cette discussion, il semble opportun de faire référence à notre expérience au sein d'un département d'administration de réseaux et systèmes d'une université publique brésilienne, comme à celle avec de nombreux professionels et utilisateurs-développeurs de divers domaines de l'informatique. Cela permet d'affirmer que l'élément de passion pour l'information est déterminant dans la différenciation entre "l'armée de réserve" des professionnels des TI, intéressés par l'acquisition d'un savoir et de sa valeur directe sur le marché du travail, et ceux qui, en plus de ces préoccupations matérielles, développent un intérêt constant pour l'appropriation, la production et la discussion de l'information. Au sein de ce département d'informatique, la division des tâches semblait plus déterminée par la capacité de chacun à "s'amuser" face à un problème, que par la formation ou le rang. Le terme "s'amuser", utilisé par les propres acteurs, se réfère principalement aux aptitudes des employés à se responsabiliser, c'est à dire, à percevoir les ramifications d'un problème, se confronter à d'autres erreurs possibles, de tenter les résoudre de la manière la plus systématique et de s'étonner, c'est à dire, s'amuser avec les paradigmes utilisés dans ce processus et les croiser avec des considérations techniques et théoriques afin d'améliorer leur conception générale du système. De cette manière, la curiosité et l'intérêt personnel pour le fonctionnement des technologies utilisées paraissent être le critère déterminant pour juger de la qualité d'un employé. Un informateur illustre :

\begin{quote}
chaque fois qu'un nouveau stagiaire entre [dans le département d'informatique] je lui dis : l'Université ne va pas t'apprendre un travail mais te donner une formation générale... Tu va apprendre les choses en mettant les mains dans la pâte, avec des problèmes techniques, concrets, en cherchant des solutions, des petits trucs simples et efficaces. Plus tu cherches des raisons derrière un problème, genre : les structures du système, comment marche le réseau, le protocole, comment dialoguent deux ordinateurs, plus tu vas créer des réparations expertes, intelligentes, qui vont rester et par-dessus lesquelles on va pouvoir travailler et développer de nouvelles choses. \c Ca, il y'en a pas beaucoup qui comprennent, hein ? lors u dernier recrutement de stagiaires, il y en avait plus de 15 dans mon bureau et aucun d'entre eux n'a su me dire ce qu'est une base de données... tu le crois ?
\end{quote}

Comprendre cette caractéristique du hacker, qu'il soit technicien de laboratoire ou utilisateur-développeur dans un projet libre, permet d'observer que ce qui défini le plus cette figure c'est sa recherche "absurde" pour une information libre, abondante et de précision et que c'est à partir de cela que se construit l'interaction de l'actant avec son contexte social. L'utilisateur-développeur veut une information libre, capable de circuler et d'être modifiée, mais cela se réalise dans un environnement de micro-restrictions qui limitent son activité. Il s'agit de restrictions légales, culturelles, politiques, économiques, qui perturbent une activité subjective d'auto-réalisation. \'A propos du logiciel libre, les restrictions les plus évidentes sont légales, c'est à dire, au sein du code, celui que l'on peut ouvrir et modifier, et l'autre qu'on ne peut qu'exécuter. Parmi d'autres options, la retro-ingénierie permet d'obtenir une partie d'un code fermé, mais cela enfreint les droits d'auteurs. De même certaines innovations libres sont utilisées et fermées par des technologies propriétaires. Cela illustre en partie le fait que les frontières entre logiciel libre et propriétaire ne sont pas si claires que ses deux acteurs le laissent entendre. ne plus du fait qu'il y ait de nombreux outils et de protocoles de piratage sous licence libre (ex : Bittorrent), une des origines du mouvement du logiciel libre est d'une certaine manière le \emph{hacking} du système Unix de AT\&T (cf. \ref{1.2.2}). Cette confusion peut être explorée par les propres compagnies de logiciel propriétaires comme l'observe T\'ulio Vianna  à propos des rapports états-uniens sur le piratage. Par exemple, la \emph{United State Trade Representative} (USTR), associe les parts de marché du logiciel libre à celles  du piratage pour dénoncer les crimes contre les droits d'auteurs \citep{Vianna2006}. 
Cependant, le contexte de restrictions auquel se confronte le \emph{geek} dans ses explorations va bien au-delà des considérations légales qui sont les plus médiatiques. Ainsi on trouve divers génériques d'interaction avec la société et ses structures et dans ce sens, Gabriella Coleman a identifié trois idéaux-types de logiques propres à des éthiques différentes de l'information :

\subsubsection{Les politiques de transgression : le \emph{undergound hacking}} \label{2.2.2.a}

La partie transgressive  du monde du \emph{hacking} est son visage le plus fantasmé et médiatique, mais aussi le plus déprécié. Dans tous les cas, elle reste quantitativement, en terme de production de code, de participants et d'institutions, minoritaire. Le \emph{hacking undergound} s'est illustré notamment dans la formation des notions d'ingenierie sociale et de \emph{human data} (données humaines) où la systématisation des comportements humains et de leurs failles, s'exprime à travers un cynisme explicite \citep{Mitnick+Simon2002}. On trouve alors une forte critique du libéralisme associée à la notion nietzschéenne de pouvoir et de plaisir. Cependant, "comme la tentative de Nietzsche d'élever le pouvoir créatif de l'individu ne s'est jamais vraiment libérée des notions libérales des Lumières, la pratique du \emph{hacking underground} représente plus une radicalisation des fondements du libéralisme, qu'une véritable rupture"\footnote{"[...]just as Nietzsche’s attempt to elevate the creative powers of the individual never fully succeeded in definitively escaping the orbit of the Enlightenment’s liberal notions, so too, the practice of the hacker under-ground represents merely a radicalization, rather than a complete break from, the moral claims of liberalism."} \citep[p.263]{Coleman+Golub2008}. Cette éthique de la transgression est une critique politique, qui se manifeste notamment par le culte du "plaisir d'être surveillé" et de "l'interface entre la surveillance et son échappatoire" \citep{Hebdige1997}. L'information est "bonne", "agréable", si elle est interdite et si son aquisition nécessite une transgression.

\subsubsection{Les politiques de technologies : la cryptoliberté} \label{2.2.2.b}

La cryptographie est l'encodage de données par un algorithme réversible au moyen d'une clé de données secondaires (décodage). Cette technologie est utilisée à des fins de confidentialité, d'authenticité et de contrôle d'intégrité. Dans le domaine de l'informatique, la première clé publique a été publiée en 1975 par Whitefiels Diffie et Martin Hellman (MIT). L'usage de la cryptographie s'étend alors aux institutions et aux entreprises, mais il n'existe pas encore de solutions pour les ordinateurs personnels.

En 1991, alors que le Sénat américain s'apprêtait à voter une loi pour interdire l'utilisation privée de la cryptographie, Phil Zimmerman publie la première clé publique utilisable par des particuliers (PGP - \emph{Pretty Good Privacy}), se rendant ainsi coupable de désobéissance civile et encourant une accusation de trahison. \'A cette occasion, l'auteur du programme développe tout un discours pour se défendre sur la scène médiatique :

\begin{quote}
Si la privacité est hors-la-loi, seuls les criminels auront le droit à la privacité : [...] PGP permet aux gens de se charger de leur privacité. Il y a une nécessité sociale pour cela et c'est pourquoi je l'ai écrit. 
\begin{flushright}
Phil Zimmerman
\end{flushright}
\end{quote}

Ceci est le début de la formation de l'éthique crytpo-libertaire, renforcée par la création en 1992 des Cypherpunks, association de hackers militant pour les droits civils. Ils travaillent sur des technologies de privacité individuelle et militent contre les lois la limitant. Sur le plan politique, ce sont des valeurs libérales qui soutiennent ce mouvement, notamment celles de l'autonomie de l'individu et de sa liberté face au gouvernement. Se crée ainsi l'espoir que les technologies puissent résoudre les problèmes sociaux, car elles reformulent dans le nouveau langage technologique la répulsion libérale à l'intervention de l'état. Concrètement, ce mouvement rassemble des anarcho-capitalistes radicaux, des démocrates, des républicains. Pour certains d'entre eux, il n'y a rien de nouveau dans ce mouvement, il s'agit simplement de la continuation, dans la société de l'information, de la lutte pour les droits constitutionnels.

De plus, le domaine de la cryptographie est un bon exemple de l'influence des communautés libres sur les structures d'une société, en rendant difficile l'interdiction de l'usage d'une technologie, en la mettant libremment à disposition. Ainsi, dans un pays comme la France, où l'usage de la cryptographie était interdit, les nécessités provoquées par les réalisations concrêtes des communautés (libre utilisation, carence sociale) ont largement contribué à l'autorisation de son usage en 1999.

\subsubsection{Les politiques d'inversion : le mouvement du logiciel libre} \label{2.2.2.c}

Parallèlement au mouvement cryptolibertaire, d'autres hackers vont développer une autre vision de la sécurité. Pour Richard Stallman, fondateur de la FSF, la connaissance ne doit pas faire l'objet d'orientation, car le bénéfice tiré d'une information est toujours fait au détriment de la communauté. Stallman militait au sein des bureaux du MIT en laissant sa machine sans aucun mot de passe pour que ses fichiers soient mis à disposition de tous avec un message d'accueil expliquant sa philosophie de l'information. Lorsque la FSF a été créée en 1984, s'est développée une pédagogie qui a ouvert le monde des hackers à l'extérieur. Leurs actions respectaient la loi et ils se servaient d'elle pour se protéger. La création de la Licence Publique Générale (GPL) délimite une zone légale de sécurité, de publicité, où les codes restent ouverts. C'est une notion de liberté positive qui est mise en avant, de liberté par l'ouverture, et non une liberté par la négative, par la fermeture ou le secret, comme cela peut être le cas avec la cryptographie.

Cette idée d'inversion se trouve aussi dans l'article \emph{Beating them at their own game} (les battre à leur propre jeu) \citep{Best2003a} où Kirsty Best démontre que le Libre représente pour ses utilisateurs-développeurs, plus un investissement qu'un rejet des structures sociales existantes autour des nouvelles technologies. Il ne s 'agit pas de promouvoir un changement radical, mais "d'entrer dans le jeu" des structures capitalistes pour redéfinir ses termes à propos des technologies de l'information et des nouveaux moyens de communication.

\subsubsection{Les politiques de collaboration : le mouvement \emph{open-source}}
\label{2.2.2.d}

Dans notre étude, ce type-idéal a été ajouté à la typologie de Gabriella Coleman, dans le but de marquer la différence entre l'éthique \emph{open-source} et celle du Libre, nous permettant ainsi de souligner la convergence de plusieurs éthique dans celle étudiée ici.

Associés par "coincidence" \citep{Torvalds+Diamonds2001} au projet GNU, Linux et son projet technologique vont favoriser la variation "ouverte" du mouvement du Libre, alors appelé à se radicaliser. Le logiciel ouvert, met en avant un modèle de développement non seulement bon, mais surtout efficace. La liberté de l'information libère un \emph{entertainment}, un mérite, qui sont des motivations considérées beaucoup plus efficaces qu'un simple salaire pour inciter à la participation à un espace collaboratif productif. Cette éthique hacker est très développée dans l'oeuvre de Himanem, préfacée par le créateur de Linux, où est soulignée la flexibilisation d'un travail entendu comme hobby et passion, qui peut ne pas recevoir une rémunération directe systématique \citep{Himanem2001}. L'idéologie ainsi transmise est plus économique que politique. La liberté de l'information est revendiquée parce que nécessaire et stratégique, néanmoins il faut aussi pouvoir en tirer un bénéfice et favoriser ainsi la création de valeurs.

Dans cette nouvelle économie collaborative, la "wikinomia"\footnote{Wikinomia est le nom donné à l'économie collaborative à partir du nom de la plate-forme collaborative Wiki qui, par exemple, sert de base à l'encyclopédie en ligne Wikpédia.}, "la capacité de rassembler les talents individuels et les entreprises dispersées est en train de devenir "la" compétence du dirigeant et de l'entreprise" \citep{Tapscott+Williams2008} et, ainsi, l'objet du propre \emph{hack}. Dans le domaine du logiciel, ce groupe s'est illustré dans la construction du Web 2.0, agglomérats d'outils participatifs où l'utilisateur donne du contenu à la plate-forme, comme c'est le cas de l'encyclopédie Wikipédia basée sur la technologie Wiki, logiciel libre développé par Ward Cunningham en 1995. De plus, l'éthique à dominante collaborative s'illustre dans la sphère technologique en créant des algorithmes performants visant à créer des statistiques dédiées à l'analyse des comportements des utilisateurs d'Internet. Il s'agit alors de "programmer l'intelligence collective"\footnote{Titre d'un livre récent ayant fait référence dans le domaine de la programmation du web collaboratif : SEGERAN, Toby, \emph{Programming Collective Intelligence:} Bulding Smart Web 2.0 applications}, O'Reilly, 2007.} et de traiter les informations accumulées par l'observation des utilisateurs, pour, par exemple, suggérer des "produits assimilés", "un partenaire idéal", etc. Si la programmation de la collaboration développe sa propre éthique de l'information, ses outils technologiques s'inspirent beaucoup des mouvements vus précédemment : elle associe l'ingénierie sociale du \emph{hacking underground} et est viscéralement liée au libre accès aux informations, afin de pouvoir les traiter, et à leur stricte confidentialité, pour garantir la privacité des utilisateurs et empêcher les contestations judiciaires d'atteinte à la vie privée.

\begin{center}*\end{center}

En observant cette typologie, on note une \emph{heteroglossia} \citep{Bakhtin1984, Bakhtin1993}, c'est à dire une diversité au sein du même code linguistique, dans lequel on trouve une discussion incessante sur la liberté. Ce qui constitue le discours moral des hackers et ce qui différencie leurs éthiques, c'est l'élaboration d'un sens autour de ce qu'est la liberté et de ce que signifie être libre. Cette diversité donne un dynamisme aux communautés hackers, qui passent alors d'une variété de discours à une autre et changent ainsi de répertoires de références sans beaucoup se préoccuper des contradictions de contenu, du style ou des effets politiques. De plus, si la trajectoire discursive qui réalise la collaboration dans les communautés est en constante négociation, au fil du temps, des points fixes apparaissent. La défense de la liberté d'information cohabite avec une tradition libérale qui trouve alors une nouvelle visibilité et un nouveau discours hétéroclite en harmonie avec l'ère digitale. 

Néanmoins, on peut affirmer que ce qui unit ces nouveaux discours c'est une pratique partagée de la programmation. En effet, quel que soit le domaine technologique ou politique de ces idéaux-types, ils désignent des communautés qui ont pour ressemblance, pour activité commune, le fait d'écrire du code. Dans ce sens, Coleman affirme que "la liberté du logiciel libre, bien qu'influencée par des sensibilités libérales majeures, est fondamentalement modelée par les pragmatiques de la programmation et le contexte social de l'utilisation de l'Internet"\footnote{“The freedom of free software, while influenced by wider liberal sensibilities, is fundamentally shaped by pragmatics of programming and the social context of internet use.”} \citep[p.509]{Coleman2004a}.

\section{Les pragmatiques de la programmation comme outil pour aborder et suivre l'interaction des utilisateurs-développeurs avec l'information} \label{2.3}

Dans un effort de définition de l'acte de programmer, Andrew Goffey utilise la définition de "dire" faite par Michel Foucault dans \emph{L'archéologie du savoir} : "\st{Dire} [programmer], c'est faire quelque chose -- quelque chose de différent que d'exprimer ce que quelqu'un pense, traduire ce que quelqu'un sait, et quelque chose de différent que de s'amuser avec la structure du langage"\footnote{“To \st{speak} [program] is to do something – something other than to express what one thinks, to translate what one knows, and something other than to play with the stucture of language.”} \citep[p.14]{Goffey2008}. En observant les utilisateurs développeurs comme programmeurs, une dualité semble diriger leur tâche : d'un côté, il s'agit d'un acte très précis, d'un traitement rationnel d'une situation rationnelle dans un contexte rationnel ; cependant, d'un autre côté, le discours entre les programmeurs sur leur objet, l'ontologie que les auteurs développent sur leur oeuvre, appellent de nombreuses références subjectives, intersubjectives, et proches de l'expression d'une nature artistique et esthétique du code (\ref{2.3.1}). En  observant et analysant ces natures artistiques et normatives de l'acte de programmer, nous pourrons, alors, exposer notre paradigme d'analyse, c'est à dire, les pragmatiques de la programmation comme vecteur de l'agnosticisme du mouvement du Libre et de sa relation à l'information.

\subsection{La programmation comme art et régulation} \label{2.3.1}

\subsubsection{La programmation comme art et l'esthétique du code} \label{2.3.1.a}

{\footnotesize \begin{verse} \begin{flushright}
"Poète à poète. Je t'imagine\\
en marge du langage, au début de l'été\\
à Wolfeboro, New Hampshire, écrivant du code.\\
Tu n'as pas la notion du temps. Ni celle des minutes.\\
Elles ne peuvent pas atteindre ton monde,\\
ton ordinateur gris\\
avec quand déjà maintenant jamais et une fois.\\
Tu avais perdu les septs autres.\\
Celui-ci est le huitième jour de la création.\\
\ldots\\
\newpage
J'écris sur un écran bleu\\
comme n'importe quelle montagne, comme n'importe quel lac, composant ceci\\
pour te montrer comment le monde recommence:\\
Un mot à la fois.\\
D'une femme à une autre."\footnote{Hommage de la poètesse Eavan Boland à Mme Grace Murray Hopper (1906 -- 1988), programmeuse d'un compilateur pour le language COBOL, pendant les années 1940.\\ “Poet to poet. I imagine you / at the edge of language, at the start of summer / in Wolfeboro, New Hampshire, Writing code. / You have no sense of time. No sense of minutes, even. / They cannot reach inside your world, / your grey workstation / with when yet now never and once. / You have missed the other seven. / This is the eighth day of creation.[…] I am writing at a screen as blue /as any hill, as any lake, composing this / to show you how the world begins again: / One word at a time. / One woman to another.”}\\
\emph{Code}, Boland, 2001.
\end{flushright} \end{verse} }

\emph{L'art de la programmation} est le titre d'un ensemble de quatre volumes reconnu comme une des plus importantes oeuvres didactiques du domaine de l'informatique, et nommé comme l'une des douze majeures monographies scientifiques par le journal \emph{American Scientist}. La rédaction de ce livre a commencé en 1962 lorsque la maison d'édition Addison Wesley proposa à Donald E. Knuth, alors docteur en mathématiques de 24 ans et candidat à un poste de professeur, d'écrire un livre sur les compilateurs. Jusqu'à aujourd'hui, le livre est actualisé fréquemment par l'auteur et un cinquième volume est en préparation pour 2015. Selon Maurice J. Black, cette oeuvre suit un courant d'enseignement et de commentaires des programmes et de la programmation comparable à un effort de critique littéraire. Avant \emph{L'art de la programmation}, les \emph{Commmentaires de Lions sur le sixième version de Unix} de John Lions accompagnaient déjà les programmeurs dans leur étude du code source du système de AT\&T, alors mis à disposition. Plus qu'une simple étude, il s'agissait en fait de la lecture d'une entité esthétique dont les commentaires révélaient les sens, particularités et logiques \citep{Black2002}.

Au-delà du fait que les écrits sur le code puissent être considérés comme une critique littéraire, le code lui-même apparaît comme une oeuvre en soi. Ce point de vue est parfaitement incarné par les théories de la programmation littéraire (\emph{literate programming}) développées par Donald Knuth :

\begin{quote}
Un adepte de la programmation littéraire peut être vu comme un essayiste, dont la principale préoccupation est d'exposer avec excellence de style. Un tel auteur, son thesaurus en main, choisit les noms de ses variables avec attention et explique ce que chacune d'elles signifie. Lui ou elle s'efforce de faire un programme qui soit compréhensible, car ses concepts ont étés introduits dans un ordre qui est le plus adéquat à l'entendement humain, en utilisant un mélange de méthodes formelles et informelles qui se renforcent les unes les autres.\footnote{“The practitioner of literate programming can be regarded as an essayist, whose main concern is with exposition and excellence of style. Such an author, with thesaurus in hand, chooses the names of variables carefully and explains what each variable means. He or she strives for a program that is comprehensible because its concepts have been introduced in an order that is best for human understanding, using a mixture of formal and informal methods that nicely reinforce each other”} \citep[p.1]{Knuth1983}.
\end{quote}

De cette manière, l'analogie faite entre la programmation et la littérature reformule une méthodologie de l'acte d'écrire du code. "Selon Knuth, le meilleur code n'est pas celui écrit depuis la perspective pragmatique d'un ingénieur, mais celui écrit depuis la perspective artistique d'un auteur. L'économie du style, la clarté de l'expression et l'élégance formelle sont aussi essentielles à la bonne programmation qu'elles le sont à la bonne écriture.\footnote{“For Knuth, the best code is written not from the pragmatic perspective of an engineer, but from the artistic perspective of an author. Economy of style, clarity of expression, and formal elegance are as essential to good programming as they are to good writing”} \citep{Black2002}. Plus radical encore dans le sens de considérer l'acte de codifier comme une pratique poétique, le mouvement de la Poésie Perl (\emph{Perl Poetry}) utilise le langage de programmation Perl, caractérisé par l'usage de fonctions désignées en termes anglais "naturels", pour écrire et traduire des poèmes compilables (Figure \ref{fig2.1}, p.\pageref{fig2.1}).

\begin{figure}[htb]
\caption{The Coming of Wisdom with Time (1910, 2000)} \label{fig2.1}
\begin{multicols}{2}

\begin{verse}
Though leaves are many, the root is one;\\
Through all the lying days of my youth\\
I swayed my leaves and flowers in the sun;\\
Now I may wither into the truth\\
\end{verse}
\begin{flushright}
William Butler Yeats, 1910.
\end{flushright}

\columnbreak
{\scriptsize
\begin{verbatim}
while ($leaves > 1) {
        $root = 1;
}
foreach($lyingdays{'myyouth'}) {
        sway($leaves, $flowers);
}
while ($i > $truth) {
        $i--;
}
sub sway {
        my ($leaves, $flowers) = @_;
        die unless $^O =~ /sun/i;
}
\end{verbatim}
}\\
\begin{flushright}
Wayne Meyers, 2000.\\
\end{flushright}
{\scriptsize
\begin{verbatim}
perl The\ Coming\ of\ Wisdom\ with\ Time 
Died at The Coming of Wisdom with Time line 12.
\end{verbatim}
}
\end{multicols}
\end{figure}

La métaphore de l'art n'est pas étrangère aux faits de langages communs au sein du milieu informatique. Comme le note Samir Chopra et Scott Dexter, la "beauté" du code est un sujet confortable pour les informaticiens et le langage pour la décrire a de nombreux adjectifs émotifs. Une suite d'instructions est belle ou laide, propre ou sale, légère ou pesante, fantastique, impressionnante, horrible, etc \citep{Chopra+Dexter2007}. Dans ce sens, un extrait du code source du kernel Linux de 1990 l'illustre (Figure \ref{fig2.2}, p.\pageref{fig2.2}) :

\begin{figure}[h]
\caption{Extrait du Kernel Linux 0.11 (1990)} \label{fig2.2}
{\footnotesize
\begin{verbatim}
/*
* Yeah, yeah, it's ugly, but I cannot find how to do this correctly
* and this seems to work. If anybody has more info on the real-time
* clock I'd be interested. Most of this was trial and error, and some
* bios-listing reading. Urghh
*/
#define CMOS_READ(addr) ({ \
outb_p(0x80|addr,0x70); \
inb_p(0x71); \
})
(Linux Kernel 0.11 main.c )
[...]
#define outb(value,port) \
__asm__(“outb %%al,%%dx”::”a” (value),”d” (port))
#define inb(port) ({ \
unsigned char_v; \
__asm__ volatile (“inb %%dx,%%al”:” = a” (_v):”d” (pot)); \
_v; \
}]
(Linux Kernel 0.11 /include/asm/io.h )
\end{verbatim}
}
\end{figure}

Bien qu'il fonctionne correctemment, ce code est décrit comme "laid" (\emph{ugly}) par son auteur, pour ne pas être la meilleure solution au problème rencontré, dans ce cas, synchroniser l'horloge du système opérationnel avec l'horloge physique du hardware. On pourrait croire que si un code fait fonctionner un outil et que son résultat est celui que l'utilisateur attend, le code est valide, correct. Néanmoins, la relation du programmeur avec son code, avec les autres programmeurs et avec l'objet programmé -- ici le système opérationnel -- exige une élégance logique qui doit faire sens pour ses actants. Ce code doit réguler l'interaction du logiciel avec le matériel, de l'utilisateur avec ces deux derniers et des développeurs à venir avec l'auteur. Ces interactions forment un objet abstrait, sujet à beaucoup d'interprétations et de constructions différentes et, comme l'artiste extrait une oeuvre unique de son corps ou d'une représentation, on peut considérer "le code comme une tentative de capturer la beauté d'un objet abstrait, l'algorithme" \citep[p.77]{Chopra+Dexter2007}. En comparant ceci avec la citation à suivre d'un critique littéraire tentant définir la poésie : “Donc! La poésie serait le moment où le mot exact, l'agencement exact, est obtenu pour qu'enfin le réel qui a pu être cerné, soit restitué grâce au matériau !”\footnote{Fabrice Luchini en dédicace au Virgin Megastore à Paris, 17/11/2008 : \url{http://www.dailymotion.com/relevance/search/luchini/video/x7yrj5_fabrice-luchini-en-rencontre-dedica_creation}}. Un algorithme, pour être un montage unique qui délimite les interactions et pour être restitué grâce à un programme, est, alors, comme la poésie, un art.

L'objectif de comprendre le code comme un art est de pouvoir délimiter les caractéristiques du logiciel libre propre à sa nature artistique. Puisque l'esthétique nous dit quelque chose sur l'objet logiciel et sa création, sur la relation entre le programmeur et son artéfact, parler de logiciel libre, c'est délimiter les spécificités d'un code ouvert dans ces interactions. Le principal argument des communautés FOSS dans ce sens est le fait que, comme l'artiste est influencé par son contexte de production, par les oeuvres avec lesquelles il interagit et desquelles il s'inspire, le modèle de développement collaboratif permet à une esthétique très forte de se structurer. Comme un auteur doit lire de bons écrits, un bon programmeur doit s'inspirer de bons codes pour réaliser les exigences communautaires d'esthétique. Dans ce sens, et non sans ironie, Bill Gates affirme :

\begin{quote}
Le meilleur moyen pour se préparer [à être un bon programmeur] est d'écrire des programmes, et d'étudier les excellents programmes que d'autres personnes ont écrits\ldots Vous devez vouloir lire les codes des autres, puis écrire le vôtre, puis le faire réviser par d'autres encore. Vous devez vouloir être au sein de ce retour incroyable où vous trouvez des gens au niveau international pour vous dire ce que vous faites de mal\ldots Si jamais vous parlez à un grand programmeur, vous verrez qu'il connaît ses outils comme un artiste connaît ses pinceaux? C'est impressionnant de voir combien les bons programmeurs ont en commun dans la manière qu'ils ont de développer\ldots Quand vous amenez ces gens à observer un très bon morceau de code, vous obtenez une réaction très, très semblable\footnote{“The best way to prepare [to be a good programmer] is to write programs, and to study great programs that other people have written\ldots You've got to be willing to read other people's code, then write your own, then have other people review your code.  You've got to want to be in this incredible feedback loop where you get the world-class people to tell you what you're doing wrong\ldots If you ever talk to a great programmer you will find he knows his tools like an artist knows his paintbrushes. It's amazing to see how much great programmers have in common in they way they develepoed\ldots When you get those people to look at great piece of code, you get a very, very common reaction.”} Bill Gates, in cite[p.83]{Lammers1989}.
\end{quote}

Ainsi le logiciel libre facilite une interaction sans restriction de codes divers. L'inspiration par d'autres codes, en plus d'être possible, est encouragée par une tradition forte de critiques et commentaires au sein des communautés FOSS. Il y a une créativité de groupe qui caractérise le produit final comme étant le "meilleur" code, car écrit par les meilleurs programmeurs, car formés par ce même meilleur code, formé petit à petit au sein de communautés qui ont de très fortes exigences esthétiques. De plus, au sien des communauntés FOSS, il y a l'idée que l'exigence traditionnelle d'ésthétique et d'élégance technique fait "survivre" ce que la littérature classique a perdu. Dans ce sens, Maurice Black commente :

\begin{quote}
[La critique littéraire] démontre ses engagements politiques en abandonnant les thèmes de la littérature et de l'esthétique [\ldots] et, plus encore, en démontrant son manque de foi en l'esthétique par sa réification de la laideur au niveau de style ou de choix thématique. Dans l'objectif de transgresser et de déstabiliser les canons littéraires à fins politiques, les théoriciens culturels sont maintenant avec une alacrité toute prête pour traiter des sujets comme la folie, la torture, la douleur, la monstruosité, la pornographie et la maladie. La culture informatique, de son côté, a adopté un modèle traditionnel d'esthétique littéraire comme moyen de changement effectif, en trouvant utilité politique et valeur sociale dans un produit bien fini qui est, à la fois, entièrement utilisable et beau à contempler\footnote{“The first has demonstrated its political commitments by all but abandoning literature and aesthetics as its subject matter [\ldots] and even by demonstrating its loss of faith in aesthetics through its reification of ugliness at the level of critical style and choice of subject matter – with the goal of transgressing and destabilizing the literary canon for political effect, cultural theorists now turn with ready alacrity to subjects such as madness, torture, pain, monstrosity, pornography, and disease. Computing culture, on the other hand, has adopted a traditionel model of literary aesthetics as a means of effecting change, finding political utility and social value in the well-crafted product that is at once entirely usable and wholly beautiful to contemplate.”} \citep[p.20]{Black2002}.
\end{quote}

\subsubsection{La programmation comme architecture et la régulation par le code} \label{2.3.1.b}

Quand dans un établissement privé de restauration on revient de la salle de bains vers la salle principale, comme la place d'alimentation d'un centre commerciale, on trouve presque systématiquement une porte donnant accès aux cuisines, au cellier, ou bien à un lieu de repos pour les employés. Sur la porte, on trouve un message en interdisant l'accès : "accès restreint", "entrée interdite", "réservé aux employés"\ldots Ainsi on se dirige en fonction de son propre statut (client, employé, agent de sécurité) par le chemin qui nous est autorisé. Quand quelqu'un se connecte à son compte Gmail, son ordinateur lit les données d'un serveur d'un centre Google. Le navigateur lit donc des données qui sont physiquement voisines de nombreux autres comptes Gmail. Cependant, une règle l'empêche de diriger la tête de lecture du disque dur vers un autre compte, comme il l'avait fait pour son propre compte. C'est une règle codifiée dans un language-machine qui permet cette opération. Il y a sur ce serveur un système opérationnel qui administre des comptes pourvus d'espaces et de privilèges distincts (utilisateurs différenciés, administrateur de système, etc). Alors, comme dans le centre commercial où une architecture physique vient délimiter des salles et des couloirs dont les accès sont restreints par des normes exprimées par des symboles, le logiciel d'un serveur email vient réguler l'activité des utilisateurs par une architecture et des lois.

\begin{figure}[htb]
\caption{Comparaison entre un plan d'architecture et un code délimitant les espaces disques d'utilisateurs d'un système} \label{fig2.3}

\begin{multicols}{2}

Un plan d'architecture 
\\
\includegraphics[width=60mm]{plan.png}

\columnbreak

Un code Linux standard délimitant les espaces de divers utilisateurs

{\scriptsize
\begin{verbatim}
root:x:0:0:root:/root:/bin/bash
fulano:x:1000:1000:fulano,,,:/home/fulano:/bin/bash
daemon:x:1:1:daemon:/usr/sbin:/bin/sh
nobody:x:65534:65534:nobody:/nonexistent:/bin/sh
sys:x:3:3:sys:/dev:/bin/sh
sync:x:4:65534:sync:/bin:/bin/sync
games:x:5:60:games:/usr/games:/bin/sh
man:x:6:12:man:/var/cache/man:/bin/sh
lp:x:7:7:lp:/var/spool/lpd:/bin/sh
mail:x:8:8:mail:/var/mail:/bin/sh
news:x:9:9:news:/var/spool/news:/bin/sh
uucp:x:10:10:uucp:/var/spool/uucp:/bin/sh
proxy:x:13:13:proxy:/bin:/bin/sh
\end{verbatim}
}
\end{multicols}
\end{figure}

Depuis la publication du livre de Lawrence Lessig, \emph{Code and other laws of cyberspace} \citep{Lessig1999}, affirmer que le code est une loi qui régule les comportements par coercition est devenu une idée commune au sein des analyses de la société d'information. Grâce à des métaphores détaillées de la loi, de l'architecture, des normes sociales et des marchés financiers, l'auteur montre que le logiciel régule les comportements, que "code est loi". \' A partir de cette observation, la réflexion de Lessig va s'articuler selon un syllogisme simple : si les institutions peuvent réguler le logiciel et que le logiciel peut réguler les individus, alors les institutions peuvent réguler les individus par le logiciel. Ainsi, on souligne l'importance de comprendre les implications politiques et sociales des décisions techniques, car elles incarnent une nouvelle forme de loi. De manière générale aussi, se discutent les changements que les dirigeants doivent promouvoir pour l'Internet.

James Grimmelmann [Grimmelmann, 2005] fait une critique du travail de Lessig et des débats qui naquirent des analogies entre logiciel, loi et architecture. Bien que ces métaphores soient considérées comme heurisitques à la compréhension des rôles sociaux et politiques du code, Grimmelmann souligne le manque de discussion des différences qualitatives entre logiciel et loi et entre logiciel et architecture. Ainsi, si ces analogies donnent beaucoup de sens pour comprendre l'objet logiciel dans son contexte social et politique, plusieurs différences de nature et de mode d'opération différencient ces objets. Pour l'auteur, la comparaison avec l'architecture permet de souligner la nature automatisée et immédiate du logiciel, alors que celle de la loi permet de souligner sa nature plastique. Le logiciel est \emph{automatisé} parce que une fois programmé, il agit sans nécessiter aucune intervention humaine. Au dela de la propre norme, qui une fois établie se reproduit de façon autonome, des algorithmes performants, communément désignés comme "intelligents", peuvent s'adapter aux évolutions des comportements qu'ils régulent. Le logiciel est \emph{immédiat} car, au lieu de s'appuyer sur les bases de la sanction, il empêche simplement l'action de se produire. La coercition n'a pas de degré différent selon son appréhension par les individus, c'est à dire par la peur de la sanction, car l'acte interdit ne peut tout simplement pas etre réalisé. Enfin, le logiciel est \emph{plastique} car il est concevable n'importe quel système qu'un programmeur puisse imaginer et décrire avec précision. Ce dernier point souligne aussi la nature fragile et cassable du code, c'est à dire, que n'importe quel faille qui peut être imaginée, peut être explorée. Dans ce contexte, une caractéristique supplémentaire différencie le code ouvert du code fermé. Un logiciel dont le code est ouvert est transparent, ses modes de régulations sont accessibles à la compréhension des sujets dont il détermine le champ des possibles. D'un autre côté, un logiciel fermé maintient un algorithme caché, comme fait unique et autonome, indépendant de la compréhension que ses sujets en ont.

\subsection{Les pragmatiques de la programmation comme paradigme d'analyse} \label{2.3.2}

\subsubsection{Les pragmatiques de la programmation : une interprétation socio-politique d'une notion linguistique} \label{2.3.2.a}

Une des définitions les plus anciennes de la pragmatique est faite par Charles W. Morris : "La pragmatique est cette partie de la sémiotique qui traite des relations entre les signes et les usagers des signes" [Morris, 1938]. D'un point de vue strictement linguistique, la pragmatique traite avant tout du sens, comme la sémantique, mais pour elle, c'est l'usage qui est fait des formes linguistiques qui détermine le sens qu'elles peuvent avoir. Dans sa branche la plus pragmatique, notamment dans l'oeuvre de Yehosua Bar-Hillel \citep{Barhillel1970}, on utilise des paradigmes comme l'étude des symboles lexicaux, sens littéral et sens communiqué, actes de langage.

De manière générale, la pragmatique permet de questionner divers faits de langage : quand nous parlons, que faisons-nous ? que disons-nous exactement ? qui communique avec qui ? qui suis-je pour mon interlocuteur ? quel sens est nécessaire à la cohérence de mes propos ? un mot a-t-il un sens litéral ? Quels sont les usages du langage ? \citep{Armengaud2007}. Ainsi, parler des "pragmatiques de la programmation" serait questionner de la même manière l'acte d'écrire du code : Que faisons-nous lorsque nous programmons ? Que programmons-nous ? Comment l'utilisateur perçoit-il l'ordinateur ? comment l'ordinateur reçoit-il les instructions des utilisateurs ?  Comment se forment les erreurs d'interprétation et de syntaxe ? Et, principalement, dans quelle mesure la réalité du programmeur et sa capacité à programmer interagissent, se déterminent-elles l'une et l'autre ?

Dans un abord plus philosophique de la pragmatique et des interprétations plus politiques et sociales qu'elle permet, la pragmatique a réorienté le regard de la science du langage sur les interlocuteurs. Leurs complexités, mises à nu par son analyse, amènent l'observateur à questionner les concepts de sujet et d'individu, notamment en se concentrant sur l'interlocution. Ainsi, s'il y a une Raison, elle ne peut exister que par la validation intersubjective du savoir. Appliqué à l'esthétique, l'angle d'analyse offert par la pragmatique fait des oeuvres artistiques un ensemble de formes symboliques dont les sens et références dépendent des ses conditions de vie et d'action, de son contexte. L'art est alors une expérience qui a pour conséquence de communiquer avec un contexte qui vient lui donner son sens : "Je ne dis pas que la communication est l'objectif d'un artiste, mais qu'il s'agit d'une conséquence de son travail"\footnote{“I do not say that communication is the intent of an artist. But it is a consequence of his work.”} \citep{Dewey1934}.

De cette manière, l'analogie entre la programmation et l'art par l'utilisation de l'idée d'esthétique du code \citep{Chopra+Dexter2007} démontre que plusieurs parallèle sont acceptés entre écrire du code et rédiger de la littérature, entre commenter du code et faire de la critique littéraire, entre écrire des scripts comme s'écrit de la poésie \citep{Black2002}. Pour cela, en acceptant l'idée que la programmation puisse être considérée comme un des "beaux-arts" \citep{Levy1992}, le concept de pragmatique aide à écarter les notions de sujet de l'analyse du mouvement du logiciel libre et à se concentrer sur la constitution d'un corpus de code qu'il incarne. Les différentes communautés et leurs interactions autour des projets logiciel constituent une \emph{praxis} qui se révèle à l'observateur par l'analyse des relations développées avec l'information.

\subsubsection{La pragmatique de la programmation comme relation avec l'information et ses politiques informelles} \label{2.3.2.b}

L'hypothèse l'usage des pragmatiques de la programmation comme outil d'analyse du mouvement du logiciel libre est que la programmation est un discours sur l'information. Un programme traite, gère, crée et modèle des informations. Les individus qui code ces logiciels créent donc un discours sur ces informations en choisissant la manière par laquelle la machine va les traiter. Chaque protocole de réseau traite les flux d'informations d'une manière différente, et l'effort de configuration fait par l'administrateur d'un réseau déterminé réalise des choix spécifiques qui vont aussi structurer l'information, sa diffusion et sa réception. De ces choix et régulations qui constituent la relation à l'information, les programmeurs construisent une ontologie que le programme vient automatiser. Dans ce sens, la notion de pragmatique permet d'isoler cette tache du programmeur et de développer une vision ample des interactions qui se manifestent autour de l'acte de programmer. Au delà des interactions entre les propres utilisateurs-développeurs, ont peut considérer aussi leurs interactions avec l'objet ordinateur, ses logiques, ses limites. Elles maintienent une relation privilégiée, pour être très normative et décisive, avec une machine qui traite l'information, alors entendu comme \emph{dispositif}, c'est à dire, "un ensemble hétérogène qui résulte du croisement des relations de pouvoir et de savoir" \citep{Agamben2007}. Cette notion de "dispositif" est caractéristique de l'effort des communautés FOSS au sein du mouvement technologique. La machine isolée, particulière, se définit, dans ses fonctions et possibles, par les logiciels qu'elle utilise pour traiter les informations que l'utilisateur lui donne. Les normes et structures de tels logiciels résultent des relations de pouvoir et de savoir présentes dans le processus de codification d'un programme. Alors, la pragmatique du programmeur est l'étape la plus fondamentale (après la création du propre langage de programmation) où ces relations se réalisent.

De cette manière, l'information est structurée par l'objet logiciel, et les communautés FOSS produisent un discours sur ces flux d'information qui génèrent un ensemble automatisé et plastique qui donne son contenu au Libre comme mouvement politique et régulateur de l'espace digital. Paradoxalement, comme nous l'avons montré, la pragmatique de la programmation est aussi le vecteur de l'agnosticisme politique du mouvement du Libre du fait que la logique de la programmation détermine et permet l'absence de discours politique apparent, ou, au moins, référençable de manière commune. Néanmoins, c'est un discours spécifique que forment ces pragmatiques, sur la société d'information, qui construit un répertoire de références propre à ses actants. Ainsi, l'utilisateur-développeur crée un discours qui est un acte. Cette idée nous permet de comprendre que le code est à la fois une théorie, une abstraction et une pragmatique. Il est au même moment l'écriture d'un texte, une preuve, et la réalisation d'une expérience concrete. Pour cela, les idées politiques, comme la liberté d'expression, sont implantées par les communautés dans le processus de production et dans l'objet produit. Un exemple est la technologie Wiki, qui, construite au sein de communautés libres avec des dynamiques participatives, permet la création d'un contenu collectif ouvert. La créativité du groupe crée un processus dont les conditions se reproduisent dans le produit \citep{Sawyer2003}

En somme, l'effort de l'observateur du mouvement doit être d'émettre des hypothèses au niveau le plus fondamental de ce discours -- le code -- pour comprendre les constructions des entités politiques du mouvement du logiciel libre au sein de ses politiques informelles (transgression, civisme, inversion, collaboration).
