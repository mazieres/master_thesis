\chapter*{Introduction}
\addcontentsline{toc}{chapter}{Introduction}

Le mouvement du logiciel libre est une réaction à la privatisation du code qui s'est opéré dans les années 1960. Depuis lors il s'est structuré en différentes branches complexes et est parvenu à proposer une alternative technologique concrète dans divers domaines de la société de l'information, principalement dans l'utilisation et la construction de l'Internet.

De cette manière, on peut souligner plusieurs changements qui mettent en lumière le succès du logiciel libre, ou, du moins, son actualité dans les politiques publiques, les débats politiques et la littérature universitaire. En premier lieu, depuis plus d'une dizaine d'années, le nombre de dispositifs informatiques qui accompagnent le quotidien des individus des classes moyennes a crû de manière inédite. Le téléphone portable et l'ordinateur personnel sont deux objets communicacionnels dont les frontières s'estompent avec la réduction de la consommation d'énergie et de taille, tout en développant de meilleures performances et une stabilité accrue. Les moyens ainsi offerts deviennent des outils élémentaires de nombreuses activités qui, jusqu'alors, ne profitaient pas des effets de réseaux. Dans ce contexte, le logiciel libre apparait dans le sens commun comme un défi technique permettant la participation des utilisateurs au-delà de ce que permettent les plate-formes propriétaires.

De plus, parmi les idées qui circulent à propos du logiciel libre, on trouve celle d'une communauté sans frontières, dont l'accès est défini par la simple participation de ceux qui la souhaite. L'image de l'utilisateur-développeur se rapproche alors de celle d'un citoyen qui participe à un mouvement collectif: le mouvement technologique. Le "\emph{You}" (Toi/Vous), élu personne de l'année en 2006 par le journal TIME, représente bien ce courant contemporain qui, d'une part, place l'individu comme actant décisif de la dynamique sociale, et, d'autre part, le fond entièrement au sein de la communauté à laquelle il participe. Dans ce sens, le mouvement du logiciel libre est un exemple qui donne beacoup de sens à cette "modernité", montrant des réalisations comme l'Internet, des possibilités, comme la particiaption politique directe et des problématiques, comme celle du modèle économique de l'appropriation des savoirs.

Enfin, dans n'importe quelle direction qu'aille le mouvement technologique actuel, il est influencé par ses actants et une majorité de la population mondiale n'a pas accès aux réseaux et outils technologiques fondamentaux. C'est dans les pays périphériques que la fracture digitale est la plus apparente. Les classes moyennes ont accès aux technologies de l'information alors que les classes populaires accumulent un retard dans leur éducation digitale et, ainsi, ni ne participent, ni ne profitent de ces évolutions, bien que celles-ci paraissent structurer profondemment les sociétés contemporaines. Dans ce contexte, le logiciel libre se présente comme une solution adaptée aux politiques publiques d'inclusion digitale, étant donné qu'il permet d'offrir des technologies à coûts réduits et avec un potentiel d'adaptabilité majeur.

Dans cette étude, l'analyse se concentre sur les aspects culturels du mouvement et principalement sur ses déterminants politiques. Cependant, les communautés "libres" se présentent sans discours ou agendas politiques précis, développant ainsi une culture indépendante, comme branche de la culture "hacker". Néanmoins, bien que cet "agnosticisme politique" soit caractéristique du mouvement du libre, il est, dans le contexte informationnel, le visage d'une pratique sociale intense, informelle et normative qui construit l'aura politique du mouvement, entendu comme un ensemble. Ainsi, une hypothèse explorée dans cette recherche est que la pratique de la programmation, entendue comme pragmatique, est le vecteur constitutif et déterminant des pratiques politiques informelles et de ses conséquences sociales. De cette manière, nous présentons comment cette étude est systématisée.

Dans le premier chapitre, nous présentons une brêve histoire de l'émergence, de la constitution et de la consolidation du mouvement du logiciel libre dans le monde. En revenant sur les origines de l'objet Logiciel dans l'histoire des débuts de l'informatique, il se démontre que les distinctions "naturelles" faites entre le logiciel (\emph{software}) et le matériel (\emph{hardware}) pouvaient être interprétées différemment avant que des compagnies, comme AT\&T ou Microsoft, viennent isoler un objet "logiciel" et fermer son code-source pour disponibiliser un produit fini avec des options de configuration et d'appropriation limitées. Dans ce contexte, le mouvement du logiciel libre apparaît comme une réaction, c'est à dire, un effort de maintenir un tradition de libre interaction dans le processus d'innovation, et de construire une alternative au modèle propriétaire qui se fortifie avec la production à grande échelle des ordinateurs personnels. En son sein, le mouvement a des structures communautaires, juridiques et institutionnelles complexes qui rendent difficile la délimitation d'une ligne d'action commune, au-delà de vouloir promouvoir la nécessité d'un code-source ouvert pour les logiciels. Néanmoins, la division institutionnelle entre logiciel "libre" et "ouvert" qui se manifeste en 1998, permet de mieux comprendre les logiques économiques et idéologiques des diverses branches du mouvement global, et, ainsi, mettre en lumière ses défis contemporains.

Plus loin, le second chapitre tente de reconstruire une partie de l'effort théorique fait par la littérature universitaire traitant des aspects culturels du mouvement du logiciel libre et de la culture Hacker. Ainsi, la notion d'éthique hacker, entendue comme construction sociale, permet d'isoler plusieurs caractéristiques des actants du mouvement du libre. Cependant, comme l'affirment les travaux de l'anthropologue Gabriella Coleman \citep*{Coleman+Golub2008}, le concept d'éthique hacker reste encore prisonnier du binaire moral mediatique qui s'est formé autour de l'image du hacker (Pirate ou Héros de l'aire digitale). De cette manière, il y a une tentative d'isoler le discours politique apparent comme un agnosticisme politique, dont la fonction est de permettre à la passion pour l'information qu'ont les hackers, de s'exprimer librement. Cette pratique structure ses politiques informelles, technologiques et pratiques que nous tenterons de réduire à une suite d'idéaux-types.

Toujours dans le second chapitre, nous présentons le paradigme théorique des pragmatiques de la programmation, construit pour interagir avec la recherche technique et bibliographique. Nous soulignons alors que l'acte d'écrire du code se réalise dans un contexte de normes et d'expectatives qui déterminent son esthétique. Ceci se réalise en comparant le fait de programmer avec celui de réaliser une oeuvre d'art, une loi ou une architecture physique, à partir de références universitaires. Ainsi, nous présentons le code comme un discours normatif et politique sur l'information et son traitement par la machine et le réseau. De manière générale, l'idée est de montrer que la pragmatique de la programmation est le vecteur de l'"agnosticisme politique" du mouvement, du fait d'être le vecteur de ses politiques informelles, et que les logiques du code qui écartent les références politiques traditionnelles sont celles qui articulent le discours normatif du mouvement sur l'information.

Dans le troisième et dernier chapitre, nous proposons une typologie des trois pragmatiques emblématiques de divers projets \emph{open-source}, qui aura pour but d'interagir avec les typologies des politiques informelles du mouvement du logiciel libre. Ainsi, les logiques de transgression, de civisme technologique, d'inversion et de collaboration sont réinterprétées par le vecteur des pragmatiques de liberté, d'ouverture et de sécurité. Par l'analyse des études de cas, des références techniques et des concepts théoriques qui se trouvent dans l'effort technologique du mouvement, il s'établit des génériques d'interprétations des contextes politico-technologiques par les propres actants et communautés.
