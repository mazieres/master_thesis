\chapter*{Resumo}	

Esta dissertação procura apresentar as significações políticas e culturais de um movimento de Software Livre e de Código Aberto (SL/CA) entendido como conjunto muito heterogêneo de comunidades e projetos. Ademais, a partir de um histórico do objeto “software” desde a sua origem, mostramos como ele foi diferenciado do hardware e depois encerrado como um objeto fechado pela companhias de software nascentes. Nesse contexto, o movimento SL/CA aparece tanto uma reação ao fenômeno de blackboxing, como uma continuação da tradição de compartilhamento de informações dentro da engenharia da computação. Por isso, estrutura-se ao redor de vários ramos da ética hacker e de seu agnosticismo político para constituir uma alternativa tecnológica concreta. Isto nos permite afirmar que as características sociopolíticas das comunidades do Software Livre devem ser procuradas no próprio ato de programar, na pragmática, como arte ou regulação. Dessa forma, estudamos os casos específicos de varias comunidades (gNewSense, Samba, BSD) para tentar sistematizar os seus posicionamentos tecnológicos e sociopolíticos a respeito do movimento tecnológico contemporâneo.

\vspace{2cm}

\begin{tabular}{ll}
\textbf{Palavras-Chaves} & Software Livre;\\
& Hacker;\\
& Programação (Computadores) – Aspectos Políticos;\\
& Programação (Computadores) – Pragmáticas.\\
\end{tabular}

