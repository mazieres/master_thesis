\selectlanguage{english}
\chapter*{Abstract}	

This dissertation presents some political and cultural significations of a Free Software Movement, understood as a heterogeneous aggregation of projects and communities. Then, the historical analysis of the “software object” shows how it become, in the first place, differentiated from the hardware and, then, secondly, closed as an end-product by the rising software companies. In this context, the Free Software Movement presents itself as a reaction to \emph{blackboxing} phenomena, as well as a continuation of the computater engineering tradition of sharing knowledge freely. Therefore, FS Movement has become structured through diverse blends of Hacker Ethic and its own political agnosticism, in order to build a concrete technological alternative. This leads to the argument  that sociopolitical characteristics of Free Software communities should be found in the very act of programming, and in its pragmatics as an art or a regulation. Finally, specific cases of several communities (gNewSense, Samba, BSD) are examined in an attempt to systematize their sociopolitical and technological positions of the contemporary technological movement.

\vspace{2cm}

\begin{tabular}{ll}
\textbf{Tags} & Free Software;\\
& Hacker;\\
& Computer programming management – Political aspects;\\
& Computer programming management – Pragmatics.\\
\end{tabular}


