\chapter{Les pragmatiques de la programmation et ses logiques politiques informelles} \label{3}

Nous avons observé dans le chapitre précédent les significations culturelles du mouvement du logiciel libre par le moyen de : a) la notion d'éthique hacker, qui nous a permis de souligner 4 idéaux-types de logiques politiques informelles ; b) la construction de l'idée des pragmatiques de la programmation comme vecteur de l'interaction des utilisateurs-développeurs FOSS avec l'information et ainsi, nous a permis d'émettre des hypotèses au niveau le plus fondamental de l'activité du programmeur -- c'est à dire le code -- pour comprendre le construction de l'entité politique du mouvement du logiciel libre.

\' A cette fin, nous proposons ici une typologie de trois pragmatiques qui se retrouvent dans le discours des communautés libres. En premier lieu, il y a la pragmatique de la \emph{liberté} entendue comme l'idéologie qu'on trouve derrière le mouvement GNU et les premières traces de la culture du logiciel libre. Par exemple, on peut choisir un outil pour le fait qu'il soit sous licence GPL, et ceci comme impératif supérieur à tout autre, afin de créer une solution homogène "libre". De plus, la pragmatique de la liberté est le discours et les interactions du mouvement du logiciel libre qui privilégient les technologies "100\% libre", dans le sens où l'entendent les actants.

Au delà, nous trouvons l'idée d'\emph{ouverture} dans les discussions à propos de la comptatibilité des technologies, principalement les protocoles et les formats d'archives. La pragmatique d'ouverture serait, alors, le discours des actants du Libre sur ces problématiques, qui privilégie ces caractéristiques par rapport aux autres.

Enfin, la préocuppation de \emph{sécurité} se rencontre particulièrement dans les problématiques de stabilité des systèmes d'information et de ses défenses contre d'éventuelles attaques sur le réseau.  Ainsi, la pragmatique de sécurité serait reflétée dans les efforts des communautés pour rendre leurs programmes avant tout sûrs, même si cela peut créer des lacunes de performances ou de compatibilité, ou bien même si des licences hybrides protègent alors une partie de leur code.

L'objectif dans ce chapitre est de faire dialoguer la typologie des logiques politiques informelles propres à l'éthique hacker, avec les pragmatiques de la programmation, afin que se révèlent les propriétés politiques d'un acte que nous considérons alors comme fondamental, la programmation. Dans ce sens, la pragmatique de la liberté sera examinée à partir de l'étude de la communauté gNewSense (\ref{3.1}), celle d'ouverture en observant le projet SAMBA (\ref{3.2}); et celle de sécurité par l'analyse des systèmes BSD (\ref{3.3}). Enfin, nous présenterons un tableau de comparaison des interactions conceptuelles désirées (\ref{3.4}).

\section{Les pragmatiques de liberté : la communauté gNewSense} \label{3.1}

\begin{quote}
Le logiciel non-libre n'est jamais une solution, alors, s'il vous plait, ne pas rationaliser, justifier ou minimiser les conséquences de proposer un logicel non-libre comme solution.\footnote{“Non-Free Software is never a solution so please do not rationalize, justify, and minimize the consequences of proposing non-free software as a solution”.} (5\up{e} principes de la communauté gNewSense).
\end{quote}

La pragmatique de la liberté est la première qui apparaît au contact des communautés du Libre. Pour cela, les responsables de ces communautés veulent défendre par la technologie le droit à participer aux évolutions librement, car un code ouvert est la condition d'une telle idée de liberté. Ces prétentions ont été embrassés de manière emblématiques par la communauté gNewSense qui tente de construire une version 100\% libre de Linux, car celui-ci n'est pas "libre" dans le sens que tente de promouvoir la FSF et ses communautés proches. La plupart des distributions Linux ont toujours permis l'installation de logiciels restreints, qui, ainsi, constituaient un système sous l'empire de réglementations hétérogênes. Cette banalisation de codes restreints dans l'environement de Linux a fait un pas supplémentaire avec l'adoption en 2006 de morceaux (\emph{blobs}) de codes fermés dans le propre kernel du système. Ces blobs sont alors été incorporés aux systèmes distribués, parmi lesquels un des plus dédiés aux idéaux du logiciel libre\footnote{Vote de la communauté Debian en faveur de l'adoption des blobs binaires : 15/10/2006, \url{http://www.debian.org/vote/2006/vote_007.en.html}}.

Dans ce contexte, Richard Stallman, fondateur de la FSF et Mark Suttleworth, investisseur de Ubuntu, ont souligné d'une même voix l'importance de maintenir un effort pour construire une version strictement libre du système. Un des participants de la conférence, Paul O'Malley, amène le débat au public par IRC et, avec Brian Brazil, commence à développer une architecture rénovée du système. Les blobs et les répertoires de logiciels fermés et/ou restreints sont retirés et se constitue une communauté autour de ce nouveau système qui reçoit alors le nom de gNewSense (gNS). Les utilisateurs-développeurs se confrontèrent à des difficultés techniques importantes, rendant le système difficile à utiliser pour les nécessités communes et avec des problèmes de compatibilité de matériel. Cependant, actuellement, en Juin 2009, le système est mis à disposition dans sa version 2.2 et réussit à associer toutes les alternatives possibles aux programmes restreints, offrant ainsi une possibilité tierce aux compromis faits par les distributions concurrentes.

Les analyses proposées ici ont été réalisées à partir de matériaux récupérés principalement sur le Forum et les listes de mails de la communauté\footnote{\url{http://wiki.gnewsense.org/index.php?n=ForumMain.ForumMain}} et des interactions sur le canal IRC\footnote{\#gnewsense@irc.freenode.net:6667}. Au-delà des observations des débats sur le vif, et dans les archives de chat, ont été réalisés des entretiens de plusieurs utilisateurs-développeurs, parmi lesquels, Paul O'Malley, un des fondateurs de cette distribution.

\subsection{Logiques de transgression} \label{3.1.1}

Les pragmatiques de la programmation développées par les utilisateurs-développeurs de cette communauté ne sont pas entendues comme transgressive. Le hacking, dans ses aspects \emph{underground}, n'est pas un vecteur de l'effort technologique qui conduit la construction de ce système alternatif. De fait, les utilisateurs-développeurs aiment à se présenter, comme ceux du projet GNU ou proches de la FSF, comme "\emph{gnu hacker}". Pour être un acte profondément politique et civique, les caractéristiques des pragmatiques de la communauté gNewSense doivent être recherchées dans les autres logiques qui constituent notre typologie.

\subsection{Logiques de civisme}\label{3.1.2}

Les pragmatiques de civisme apparaissent au sein de gNS comme un ensemble de choix tournés vers un résultat d'intérêt général. De fait, les restrictions techniques qui s'imposent aux "gnu hackers" pour développer le système sur une plate-forme avec des limitations techniques importante, sont un acte positif dans le sens où elle ignorent les alternatives et les compromis réalisés par les autres distributions de Linux. Il y a un refus catégorique des perspectives de développement "réformiste" qui entendent le changement du Libre comme un processus progressif qui nécessite l'utilisation d'outils sous licences restrictives et/ou propriétaires. Cette position des développeurs s'affirme quotidienement sur les plates-formes d'interactions avec les utilisateurs qui demandent de l'aide pour pouvoir utiliser le système et souvent énoncent des solutions non-libres. Ci-dessous nous pouvons observer le commentaire d'un utilisateur, rédigé pour convaincre une personne qui voulait arréter d'utiliser gNS :

\begin{quote}
\textit{Installe Ubuntu,[\ldots] ensuite tu pourras suivre l'évolution du logiciel libre et migrer petit à petit vers des alternatives libres, un bit après l'autre\ldots ok, Ubuntu n'est pas libre, mais c'est plus libre que Windows.}\\
Je comprends ce que tu essaies de faire, cependant le message que tu fais vraiment passer est : "pas de problème si tu utilises des logiciels non-libres pour le moment ; quand le logiciel libre n'aura plus de différences fonctionnelles avec le non-libre, alors reviens". Cela rend artificielle l'éthique du logiciel libre et ne nous mène nulle part.\footnote{“ then you might follow the evolution of Fsoft and gns, change to free alternative, one byte after the other\ldots ok, ubuntu is not free, but it's freer than Windows.” / “I see what you're trying to do, but the message you actually send out is: "it's ok to use non-free software for now; once free software has caught up and there is no functional difference between free and non-free anymore, then you can switch back". This hollows out the ethics of software freedom and doesn't get us anywhere”}
\end{quote}

Dans cet exemple, le développeur reproche à l'utilisateur de proposer un logiciel non-libre (une distribution plus intuitive de Linux) comme un moyen pour arriver à une utilisation exclusive du Libre. Ceci est à contresens de l'effort civique radical de la communauté. Dans ce sens, la communauté gNS se caractérise par une forte posture politique et par des critères traditionnels, c'est à dire, un idéal réalisé par une action commune, contre un ennemi commun déclaré. De fait, les quatre libertés et la philosophie orginelle du projet GNU sont les bases de ce mouvement technologique de création d'un système opérationnel et d'une plate-forme de développement de logiciel libre. Ceci voulant se maintenir comme une alternative contre toutes les formes non-libres de code. Cependant, à cette fin, il y a un "sacrifice" technique à ne pas pouvoir utiliser certains outils basiques du monde virtuel contemporain. Par exemple, assister à des vidéos en ligne ou accéder à certains sites internet "dynamiques", ou encore accéder à plusieurs formats basiques d'applications de bureau. Dans ce sens, les participants à ce projet prétendent activer un civisme, à travers des pragmatiques de la liberté radicales en essayant d'améliorer le coût technique du développement d'une alternative entière dont quiconque peut profiter.

\subsection{Logiques d'inversion}\label{3.1.3}

Le système de propriété intellectuelle qui structure le monde du logiciel et ses conflits est basé sur des réglements, des lois, des conventions nationales et internationales (par exemple le DMCA -- \emph{Digital Millenium Copyright Act}), que des licences particulières invoquent. Une technologie est fermée ou restreinte quand elle se réfère à des concepts de copyright ou de droits d'auteurs que l'utilisateur accepte en installant le programme. L'objectif du projet GNU et de la licence GPL a été d'utiliser cette même légitimité juridique pour défendre le contraire, c'est à dire, la non-appropriabilité du code mis à disposition. C'est une logique d'inversion des défenses proprétaires pour construire les défenses du Libre.

De cette manière, les préoccupations légales sont au centre de l'activité de la commuauté gNewSense et pour cela il existe des "contrôleurs des 4 libertés" (\emph{4Freedom Checkers}) en plus des catégories traditionnelles de développeurs (codeur, aide à l'utilisateur, gestion des bugs, etc). Ces développeurs surveillent le système et tous les programmes disponibles pour vérifier l'absence de failles légales permettant à un code restreint d'être installé sur la machine. Cependant, l'exigence d'une licence compatible avec le GPL n'est pas suffisante pour cet objectif, et les développeurs cherchent aussi à déterminer si les programmes ne sont pas liés à des sources secondaires restreintes. Encore une fois, les implications d'un tel engagement apparaissent particulièrement au contact des utilisateurs. En témoigne la réponse d'un développeur à un utilisateur souhaitant installer un émulateur Windows (Wine, sous licence GPL) pour pouvoir utiliser ses jeux (propriétaires) :

\begin{quote}
Un des points clés pour comprendre gNewSense et l'ensemble du logiciel libre est de comprendre que leur objectif légal est d'être complétement libre. Alors que j'ai besoin de faire des recherches pour savoir si Wine est libre ou non, le logiciel libre n'est pas fait pour donner un moyen gratuit pour excécuter toutes tes applications Windows. C'est un mouvement politique, social et technologique pour un logiciel libre.\footnote{“one of the key points about understanding gNewSense and all free software, is to understand that its legal goal is to be completely free software. While I need to do more reasearch into weather or not Wine itself is, Free Software isn't about  giving a free (monetary wise) way to runs as many windows programs as you can. It's a political, social, and technological movement for Free (as is Freedom) Software”}
\end{quote}

En soulignant ce qui constitue la base politique et sociale qu'évoque le développeur, nous mettons en lumière la recherche pour une homogénéité juridique inédite et absolue. Ainsi une bonne partie des débats au sein des listes de mails de développement de la distribution est concentrée sur les aspects légaux des logiciels, alors que les aspects techniques, quand ils ne traitent pas d'une technologie alternative, sont laissés aux bons soins de la distribution-mère  du système (Ubuntu). Séparer ce qui est libre de ce qui ne l'est pas est la principale logique d'organisation du système gNewSense. Les logiques de liberté amènent les programmeurs à  choisir quels logiciels peuvent être installés et lesquels ne peuvent pas l'être. Ainsi, nous illustrons une inversion du radicalisme propriétaire pour un radicalisme du Libre, et, plus encore, une ignorance délibérée de tous les "compromis" qui se rencontre dans le milieu technologique de l'Open-source.

\subsection{Logiques de collaboration}\label{3.1.4}

Un espace de collaboration sans restrictions est l'objectif final de la GPL ; et les possibilités de collaboration instaurées par la communauté gNS sont en même temps absolues et fermées. De fait, d'un côté, le potentiel de dynamique participative au sein du système est infini car tout le code utilisé est libre, accessible, ouvert, modifiable dans les termes de la GPL et de ses licences semblables. D'un autre côté, l'espace ainsi créé ne profite pas à d'autres projets dont les conditions d'appropriabilité sont distinctes. Pour cela, en considérant l'extrême variété des natures juridiques des codes dans le développement des technologies, le domaine de la GPL est totalement hermétique à un grand nombre d'innovations. Celles-ci doivent toujours être recodifiées (quand elles ne sont pas brévetées) au sein d'un projet qui corresponde aux critères de la communauté.

De plus, l'idéologie du projet à propos de ses possibilités de collaboration appartient à deux délais temporels. Ainsi, nous comprenons qu'à court terme, il y a  sacrifice au nom d'une utopie, une idée de liberté absolument non restreinte qui empêche certaines possibilités de collaboration et, ainsi, permet des échanges de codes seulement au sein du domaine de la GPL. Ce compromis est entièrement dédié à l'idéal du projet GNU, afin qu'à long terme soit réalisée sa conception de collaboraiton, absolue et non-restreinte. Ceci représente un positionnement  réellement activiste au sein des communautés du Libre. Moyens et fins se mélangent pour donner l'objectif de ce mouvement, reflétant ainsi l'idéologie proposée. De cette manière, la contradiction qui souligne le manque d'ouverture comme une restriction à la collaboration n'atteint pas ses actants concentrés sur un objectif majeur.

\section{Les pragmatiques d'ouverture : la communauté Samba}\label{3.2}

Une des principales difficultés affrontées par les systèmes Linux au sein des institutions, a été son manque de compatibilité avec les protocoles de codes fermés. De fait, avec la multiplication des stations Windows, toutes les institutions sont passées à des protocoles au sein de réseaux internes qui permettent aux ordinateurs de communiquer. Un tel environnement a été difficile pour les systèmes Linux, serveurs comme clients, pour gérer de tels protocoles, ou même, simplement communiquer avec eux. C'est dans l'intention d'atténuer cette carence spécifique qu'à été créé le projet Samba, un projet maintenant ancien et réussi au sein du Libre.

Initié comme un projet de doctorat par un étudiant australien, Andrew Tridgell, en 1991, le logiciel Samba a permis peu à peu aux systèmes basés sur Unix (Linux, BSD, et autres) de dialoguer sur les réseaux avec les systèmes et serveurs Windows. En pratique, cela a permis, entre autres choses, aux clients Windows d'interagir avec les serveurs Unix, notamment pour les services d'impression et de partage de fichiers. Licencié sous GPL, le projet est devenu un standard dans toutes les branches du Libre pour gérer les protocoles de réseaux hybrides.

Les analyses proposées ici sont réalisées à partir de matériaux extraits principalement du Forum\footnote{\url{http://www.nabble.com/Samba-f13150.html}}, des listes de mails et du principal canal IRC\footnote{\#samba@irc.freenode.net:6667} de la communauté.

\subsection{Logiques de transgression}\label{3.2.1}

Les logiques de transgression sont présentes au sein de la communauté Samba  du simple fait que sont elles qui ont fondé ce projet. Afin de réussir à communiquer avec les protocoles de Microsoft, qui sont fermés, il a été nécessaire d'utiliser les méthodes de la rétro-ingénierie. \'Etant fermés, les protocoles se présentent aux développeurs comme une "boîte noire" dont les caractéristiques peuvent être découvertes uniquement par l'analyse de ses entrées et sorties. En utilisant des \emph{packet sniffers}, les développeurs utilisent des outils communs à l'exploitation de réseaux pour "sentir" le comportement des protocoles fermés et ainsi déterminer quel code permettra de communiquer avec eux. Samba explore en fait un des cas où cette pratique est permise, c'est à dire, quand elle est réalisée afin de permettre une interopérabilité entre des normes divergentes.

Une telle activité s'illustre par l'exigence d'une compétence technique avancée, et représente un défit attrayant pour ses actants : déconstruire un secret propriétaire et le faire fonctionner avec une technologie libre. Les réferences communes au sein du \emph{hacking underground} sont présentes dans les discours des développeurs dans la mesure où elles ne dépassent pas les limites de ce que la loi permet, car, étant un projet ouvert et très utilisé, les interactions des utilisateur-développeurs se déroulent au vu de tous.

Cependant, dans un contexte plus général d'observation, nous pouvons souligner ici une des caractéristiques des politiques informelles de transgression, c'est à dire, le fait que l'acte transgressif soit très dépendant de l'objet qu'il tente de déconstruire. Si on considère que l'acte de permettre à des protocoles libres de dialoguer avec d'autres propriétaires comme un acte en faveur du Libre, alors l'acte de transgression réalisé reste nécessaire tant que ces protocoles sont maintenus. Dans ce sens, si les motivations des utilisateurs-développeurs qui participent de la communauté Samba sont, en partie, inspirées par les logiques de transgression que son développement nécessite, elles sont alors dépendantes de l'existence de ces protocoles propriétaires. Pour cela, la logique transgressive permet autant aux protocoles propriétaires qu'aux libres de survivre dans un écosystème qui n'est pas entendu de manière revendicative. La pragmatique d'ouverture, dans sa logique transgressive, ne souligne pas la présence de code propriétaire sur le réseau, mais cherche un "droit" à pouvoir communiquer avec eux. L'environnement majeur où interagissent ces codes n'est pas présenté comme étant libre, mais plutôt comme étant naturellement hétérogène.

\subsection{Logiques de civisme}\label{3.2.1}

Les méthodes de rétro-ingénierie se trouvent dans divers projets libres. Il s'agit de "courir derrière" des formats de fichiers ou des protocoles indispensables au fonctionnement basique d'un système. Tout comme Samba cherche à intégrer les protocoles de Microsoft au monde Unix, Open Office "cours derrière" le format ".doc" et Wine derrière les Interfaces de Programmation d'Applications (API) du système Windows. Le résultat ainsi proposé à l'utilisateur final n'est plus une alternative, comme c'est le cas avec gNewSense, mais une interopérabilité, un code qui permet à des technologies hybrides de communiquer. \'Etant donné que le but le plus fondamental des technologies de l'information est la communication, un tel objectif d'interopérabilité a beacoup plus d'impact que celui de création d'un alternative autonome. Il s'agit de permettre les compromis.

Cependant, il y a une proposition politique forte derrière l'usage de la rétro-ingénierie, car elle est toujours défendue par ses utilisateurs comme un modèle de développement très fondamental à l'intelligence humaine. Il s'agit de regarder un objet inconnu et de le comprendre à partir de la relation entre ce qu'il produit et ce qu'il a traité. Cette opération est considdérée comme un "droit fondamental" du programmeur, comme de l'être vivant en général. Ne pas avoir l'autorisation d'utiliser cette forme d'intelligence est perçu comme une restriction à une condition très naturelle. Dans ce sens, un développeur web commente :

\begin{quote}
Bon, je suis programmeur web, je travaille beaucoup sur des technologies Flash, qui n'est pas libre [\ldots] ce que j'aime dans la philosophie du libre est qu'elle semble défendre mon mode d'apprentissage contre un autre modèle qui semble l'interdire [\ldots] je n'ai jamais été à l'Université, j'ai tout appris par moi même, et pour cela, je manque de concepts généraux... Ce que je sais c'est voir un truc et comprendre comment ça marche, peu importe que ce soit libre ou non... la boîte noire c'est mon quotidien... bon... il y a le "bidule" qui fait des trucs et en reçoit d'autres, comment marche-t-il ? Comment je fais pour qu'il traite une chose ou en produise une autre ? c'est comme un bébé qui apprend à parler ou un ado qui apprend à baiser : ça marche ? oui, je garde. ça ne marche pas ? je dégage... ça marche plus ou moins ? j'ai besoin de jeter un coup d'oeil, etc... et comme ça j'avance, j'assemble des solutions, des trucs pour assembler de nouveaux savoirs, des outils, etc ...
\end{quote}

Si l'interdiction d'une telle pratique était un jour banalisée au monde de l'informatique, il y a l'idée qu'une cognition très fondamentale aux programmeurs ne pourrait plus s'exprimer, qu'une rationalité n'aurait plus le droit de se déclencher. Pour l'instant le DMCA considère cette pratique comme acceptable à des fins d'observation ou d'éducation. Cependant les acteurs propriétaires soulignent le fait que beaucoup d'opérations de piratage sont réalisées par ces techniques, et, pour cela, souhaitent les voir interdites ou bien fortement réglementées.

Ainsi, les communautés libres comme Samba, qui pratiquent ce modèle technique, défendent profondément ce droit en soulignant sa valeure sociale et politique comme défense d'une modalité primaire de l'entendement humain.

\subsection{Logiques d'inversion} \label{3.2.3}

Pour ne pas être une alternative mais un effort dans le sens d'incorporer les réalisations du monde propriétaire, la communauté Samba n'opère pas une logique d'inversion au-delà de ce que sa licence (GPL) réalise déjà. Pour cela, ce sont les autres logiques examinées ici qui doivent être considérées avec le plus grand soin pour comprendre la pragmatique d'ouverture.

\subsection{Logiques de collaboration} \label{3.2.4}

\'Etant une communauté très ouverte aux contributions et ayant des utilisateurs-développeurs des mondes libres et propriétaires en son sein, la communauté Samba incarne bien les logiques de collaboration qui se rencontrent dans le monde open-source. L'interopérabilité des technologies et les logiques de collaboration apparaissent comme l'ultime objectif de la communauté Samba. De plus, bien que l'entrée permise aux technologies restreintes afin qu'elles puissent communiquer avec les Libres soit une réalisation qui puisse être critiquée du point de vue de la philosophie GNU-GPL la plus radicale, elle est très encouragée par l'aile pragmatique du mouvement du Libre.

On trouve derrière une telle idée une interprétation différente des limites entre ce qui peut rester fermé et ce qui doit être ouvert. La différence apparaît à un niveau technique où, alors que GNU cherche une solution homogène libre, une projet comme Samba cherche à créer un "possible" du Libre, c'est à dire l'opportunité qu'un code libre puisse communiquer avec n'importe quel autre. Le libre est alors entendu comme un ensemble homogène certes, mais capable de dialoguer avec d'autres ensembles de natures différentes, et pour cela, voir sa pragmatique d'ouverture renforçée. Un exemple évoqué par un des fondateurs de Samba explore cette perpective à propos d'une des préoccupation actuelle du mouvement du logiciel libre, qui est d'offrir une alternative aux protocoles propriétaires de VoIP (Skype, Google Talk, entre autres) : "J'espère seulement, qu'éventuellement, les développeurs de logiciels libres réussissent à s'accrocher au réseau Skype et à dialoguer avec ses protocoles. Après tout, il y a d'excellents précédents de cela avec d'autres logiciels libres\ldots"\footnote{“I just hope that eventually Free Software developers can work out how to hook into the Skype network and inter-operate with the Skype protocols, after all, there are good precedents for that with other Free Software\ldots”} \citep{Allison2005}.

Comme nous pouvons l'observer la référence souligne la faiblesse du projet Ekiga, alternative peu utilisée chez Skype, le logiciel phare du domaine de VoIP. Alors que Ekiga offre un logiciel libre pour utiliser ses propres protocoles libres, Jeremy Allison propose comme solution l'utilisation de la rétro-ingénierie pour permettre à un logiciel libre d'interagir avec les protocoles propriétaires de Skype. Nous voyons ainsi que les logiques de collaboration développées par les pragmatiques d'ouverture promeuvent non pas un compromis du Libre envers le propriétaire, mais une réinterprétation du domaine du libre par rapport aux protocoles de communication.

\section{Les pragmatiques de sécurité : les communautés BSD}\label{3.3}

Les pragmatiques de sécurité illustrent l'importance toute particulière donnée aux performances d'une technologie informatique. De fait, parmi les préoccupations d'un programmeur, comme la rapidité, l'interopérabilité ou la légèreté d'une technologie, il existe aussi la notion de sécurité. Un logiciel, comme tout "système", est cassable et possède des failles qui sont autant d'opportunités pour une tierce personne d'exploiter et de modifier ainsi son comportement sans se soumettre aux règles de l'agorithme principal. Avec la diffusion à grande échelle de l'Internet à partir des années 1990, on a souligné l'importance des problématiques de sécurité afin de limiter la diffusion de virus et l'exploitation des réseaux privés.

De fait, les technologies libres ont toujours prétendu être plus sûres car ouvertes, c'est à dire, que chaque faille restant plus apparente, elle est plus rapidemment communiquée et réparée. Indépendemment des ambitions du Libre, et plus par défi technologique, les systèmes Unix basé sur les noyaux BSD, particulièrement destniés aux serveurs, ont toujours développé la sécurité et la stabilité.

Les communautés BSD sont celles qui se sont formées aux alentours de la distribution d'Unix développée par l'Université de Berkeley (BSD -- Berkeley Software Distribution), entre 1977 et 1995. Après cette date, elles se sont divisées en trois familles principales : FreeBSD, OpenBSD et netBSD. Nous nous concentrerons principalement sur la communauté FreeBSD, qui est plus large (77\% des systèmes BSD en usage en 2005\footnote{Fonte: \url{http://www.bsdcertification.org/downloads/pr_20051031_usage_survey_en_en.pdf}} et celle de OpenBSD qui ont des projets connexes très productifs (OpenSSH, OpenSSL).

Encore une fois, les analyses proposées ici sont basées sur des matériaux principalement extraits du Forum\footnote{\url{http://forums.freebsd.org/}} , des listes de mails et des canaux IRC\footnote{\#freebsd@irc.freenode.net:6667} de la communauté. 

\subsection{Logiques de transgression}\label{3.3.1}

La stabilité que les technologies BSD mettent à disposition nécessite de la part des utilisateurs-développeurs beaucoup de participation dans le domaine de la sécurité. Pour cela, beaucoup de développeurs sont connectés aux mêmes circuits d'information qui diffusent les failles de systèmes. La logique profonde de la sécurité en informatique répond à un échange entre attaquants et défenseurs.

D'un point de vue représentatif, les attaquants veulent trouver des failles pour les explorer, mais souhaitent un système sûr afin de l'utiliser. Les défenseurs veulent aussi trouver les failles afin de les réparer et pour cela souhaitent dialoguer avec les attaquants. En résumé, chacune des parties cherche ce que l'autre trouve, et pour cela, les hautes exigences de sécurité des systèmes BSD rapporche ses développeurs des champs virutels du \emph{hacking underground}. Cependant, l'objectif qui motive les développeurs BSD dans le domaine de la sécurité est la stabilité du système, pour cela, la transgression des technologies est uniquement un moyen indirect d'y parvenir.

L'inversion opérée au niveau des logiques de transgression est la base du processus qui permet de rendre sûres les technologies BSD. L'acte transgressif de recherche de failles est reproduit intégralement par les développeurs, mais son objectif est inversé du fait d'être destiné à la correction de failles. Un exemple est la procédure d'"audit" (\emph{OpenBSD code auditing}) réalisée par la communauté OpenBSD sur tous les logiciels mis à disposition avec la distribution. Un telle procédure consiste à lire, ligne par ligne, l'ensemble des codes, en cherchant les failles potentielles de ceux-ci, ce qui est exactement ce que ferait un \emph{cracker} mal intentionné afin de pouvoir les exploiter.. Le \emph{code auditing} va donc au-delà du simple échange d'information, mais permet une exploration sûre car libérée de la collaboration avec les \emph{crackers}. Dans ce sens, la communauté BSD met en avant l'argument qu'avec un tel mode de développement elle a réussi à mettre à disposition un système qui n'a pas souffert de failles de sécurité pendant 5 ans\footnote{Jusqu'en juin 2002 le slogan de OpenBSD était : "Cinq années sans failles dans l'installation par défaut!". En 2007 le slogan était : "Seulement 2 failles dans l'installation par défaut depuis un sacré bout de temps!". Pour avoir un point de comparaison, il y a des actualisations de sécuritié chaque mois pour des sytèmes comme Linux ou Windows.}. Une telle préoccupation détermine en grande partie les logiques de civisme (contributions) et de collaboration des pragmatiques de sécurité des communautés BSD.

\subsection{Logiques de civisme}\label{3.3.2}

La diversification des activités sur l'Internet, particulièrement celles nécessitant l'échange d'argent, comme c'est le cas des achats en ligne, ou celles de consultation de données sensibles (compte bancaire, par exemple) fut permis petit à petit par l'utilisation de techniques de cryptographie à grande échelle. Une des plus grandes contributions dans ce domaine a été mise à disposition par la communauté OpenBSD, dont les projets OpenSSL et OpenSSH ont "libéré" les premiers efforts dans ce sens. De fait, le projet OpenSSH est adopté dans toutes les distributions Unix, c'est à dire sur à peu près tous les systèmes qui ne sont pas Windows et permet à deux systèmes de dialoguer par une connexion protégée et cryptée. Cependant, le projet le plus répandu est OpenSSL, qui prétend offrir la même sécurité pour les connexions Web. Sans doute, beaucoup de connexions qui se font sur le web nécessitent un type de protection, comme c'est le cas pour des achats ou pour la divulgation de mots de passe, afin que les données ainsi transférées n'apparaissent pas "en clair" à la "lecture" du réseau.

Il y a une dimension politique et sociale à de tels actes, proche de celle des premiers pas de la cryptographie. La sécurité et la privacité sont considérés par les actants de ces communautés comme une nécessité sociale, une carence. Le code ainsi produit est un acte politique et social dans le sens de recouvrir ces nécessités. Un tel effort est profitable à toute la société connectée. Une communication sécurisée protège le commerce et des transactions financières, de la dite "piraterie" de contenu intellectuel et les réseaux gouvernementaux, etc... Il y à la défense d'un droit à la privacité qui définit ses termes sans limites d'objets à protéger. Alors que ces fonctions dans le monde politique traditionnel peuvent apparaître comme relevant du domaine régalien qui a obtenu le monopole des mesures de sécurité, au moins en terme de légitimité, et définissent les limites du droit à la privacité. Dans le cas de la criptographie, et des outils libres comme ceux des communautés BSD, les possibilités de sécurité et de privacité sont mises à disposition d'acteurs indépendants qui gagnent leur légitimité du fait d'être de commmunautés "libres". Ainsi, le pouvoir politique traditionnel peut uniquement interagir avec elles par le moyen de restrictions autoritaires et non de mises à disposition ou de monopoles des outils.

\subsection{Logiques d'inversion}\label{3.2.3}

La licence BSD peut favoriser une collaboration sans restrictions permettant l'appropriation de son code par des projets propriétaires et fermés; pour cela les commuautés BSD ne peuvent pas être identifiées aux logiques d'inversion comme le peuvent celles proches des domaines GNU-GPL.

\subsection{Logiques de collaboration}\label{3.2.4}

Les communautés BSD ne sont pas explicitement ouvertes comme celles de Samba ou de gNewSense. Quand on entre sur le site internet du projet FreeBSD il n'y a pas un lien ou une annonce explicite invitant les utilisateurs à contribuer ou à participer. Les impératifs techniques propres à ces communautés exigent un haut degré de compétence souvent pointé du doigt comme "élitiste" par les autres communautés libres. Les actants considèrent que l'exigence technique qui fonde le projet leur interdit d'accepter les amateurs dont le travail peut être perçu comme mal fini et probablement de piètre qualité eu égard à leurs exigences de stabilité.

Ce "professionnalisme" attire tout particulièrement les investisseurs privés car il propose des conditions d'appropriation très avantageuses pour les entreprises qui développent des produits fermés. La licence BSD permet l'utilisation d'un code sous son empire sans aucun retour pour les communautés. Beaucoup d'exemples de collaborations ainsi déterminées ont donné des produits reconnus dans le marché des nouvelles technologies, parmi lesquels : les consoles de jeux Xbox (FreeBSD -- Microsoft) et PSP (NetBSD -- Sony), MacOSX (FreeBSD -- Apple), entre autres.

Les codes BSD sont aussi réutilisés pour des projets libres, et incoporés au système Linux par exemple. Cela positionne les communautés BSD comme plate-formes possibles de collaborations entre les logiques propriétaires et libres, et leurs utilisateurs-développeurs aiment souligner qu'une telle caractéristique favorise avant tout l'excellence technique du système et permet à un environnement diversifié de se maintenir. Dans ce sens, un développeur du Forum FreeBSD illustre : "C'est le moment de comprendre que FOSS est avant tout versatil. Tu ne voudrais pas que Linux soit BSD, que BSD soit MacOSX, etc... Comme ça on a le CHOIX. Quelque chose qui est "bon à tout" ne peut pas, par définition, être bon."\footnote{“It's time to understand FOSS is versatile. You wouldn't want Linux to be BSD, BSD to be OS X ... etc. This gives us CHOICE. Something that is "good for everything" cannot be good by definition.”.}. De cette manière, nous illustrons une logique différente dans l'interprétation de ce qu'est le logiciel libre : la versatilité d'un environement dont la diversité technique et légale permet à des solutions différentes de s'adapter à des nécessités hétérogènes.

\section{Comparaison et analyse des caractéristiques des pragmatiques}\label{3.4}

\begin{figure}[hbt]
\caption{Tableau récapitulatif} \label{fig3.1}
\begin{tabular}{|L{3cm}|L{3.5cm}|L{3.5cm}|L{3.5cm}|}
\hline
 & Pragmatique de \textbf{liberté} (\emph{gNewSense}) & Pragmatique d'\textbf{ouverture} (\emph{Samba}) & pragmatique de \textbf{sécurité} (\emph{BSD}) \\
\hline
Logiques de \textbf{transgression} & 0 & Retro-ingenierie & Recherche de failles \\
\hline
Logiques de \textbf{civisme} & Alternative entière et autonome aux outils non-libres & Interopérabilité des technologies & Sécurité et stabilité du dispositif \\
\hline
Logiques d'\textbf{inversion} & Espace juridique propre, protégé par la même légitimité légale que le domaine propriétaire & 0 & 0 \\
\hline
Logiques de \textbf{collaboration} & Communauté ouverte; Collaboration absolue en espace hermétique & Communauté ouverte ; Collaboration non-limitée par le domaine du Libre & Versatilité ; Communauté relativement fermée ; Potentielle exploitation commerciale unilatérale \\
\hline
\end{tabular}
\end{figure}

En comparant les pragmatiques de la programmation exposées dans la figure \ref{fig3.1} (p.\pageref{fig3.1}) nous pouvons observer que ces trois communautés interprètent différemment leur participation au mouvement du logiciel libre et que ces différences s'expriment en premier lieu dans les choix réalisés par rapport au logiciel objet de la communauté.

Cette diversité technique souligne pour ses actants des représentations différentes de ce que signifie être libre et ouvert, ce qui les motive à agir dans ces projets et, enfin, avec quel environnement technologique majeur le Libre interagit-il. De telles caractéristiques constituent la personnalité sociale et politique de chaque communauté et les différencient les unes des autres. Les choix réalisés forment le squelette politique d'une communauté, de son code et de son action dans le champ technologique. Que ce soit autour d'une préoccupation esthétique ou d'une responsabilité par rapport à un potentiel régulateur, les mêmes génériques décisionnaires sont mis en place dans le processus communautaire de création et de promotion du logiciel.

Au-delà, quelques éléments apparaissent à la lecture du tableau récapitulatif (Figure \ref{fig3.1}). Premièrement, les logiques de transgression et d'inversion semblent s'exclure l'une et l'autre. Dans le cas des pragmatiques de liberté, les logiques de transgression sont peu recommandées au sein du discours de la communauté GNU-GPL du fait qu'elles peuvent-être une porte  d'entrée à certains compromis par rapport à son idéologie. Pour leur part, les pragmatiques d'ouverture et de sécurité développent notablement les caractéristiques transgressives de leurs modèles de développement afin de permettre l'entrée de l'objet transgressé dans le code du logiciel. Il s'agit, dans le cas de Samba, des protocoles de communication propriétaires ; et dans le cas de BSD, la défense contre les failles de sécurité.

En partie, ces caractéristiques des logiques de transgression déterminent celles du civisme. Alors que l'absence de transgression dans la pragmatique de liberté produit la nécessité de construire une alternative entendue comme absolue, la présence de telles logiques dans les pragmatiques d'ouverture et de sécurité soulignent l'effort dans le sens d'une performance technique plus pragmatique. Il s'agit, dans le cas de Samba, d'une contribution à l'interopérabilité des systèmes Unix avec le monde propriétaire ; et dans le cas de BSD, de la mise à disposition d'outils sécurisés et stables pour servir l'activité de tout un chacun sur le réseau.

Nous pouvons donc observer que les logiques d'inversion caractérisent très spécifiquement le domaine GNU-FSF. C'est à dire que les logiques d'inversion propres aux pragmatiques de liberté comme on les observe dans la communauté gNewSense sont le monopole d'un software qui se veut exclusif. En ayant cherché ses défenses contre le monde propriétaire dans les mêmes outils légaux, c'est à dire les licences, le domaine GNU que le projet gNS vient systématiser, cherche à inverser les propositions des systèmes opérationnels propriétaires comme hybrides.

Enfin, les logiques de collaboration semblent offrir un bon reflet de la sphère politique de chacune des communautés, comme un résultat concret de sa capacité à interagir avec son environnement technologique. Pour cela, alors que les pragmatiques de liberté promeuvent une logique de collaboration du domaine de l'utopie, c'est à dire, absolue, mais dans un espace restreint à ne faire aucun compromis, les pragmatiques d'ouverture et de sécurité, elles, interagissent de manière conséquente avec leurs environnements respectifs. Pourtant ces interactions se différencient de manière significative quant à leur nature. La pragmatique d'ouverture fait de la collaboration un objet du software qu'elle développe, l'objet de la collaboration est attiré dans l'effort technologique du Libre pour que celui-ci s'en voit accru. Différemment, la pragmatique de sécurité met en avant une collaboration très forte avec n'importe quel type d'acteur tant que le "professionnalisme" du système est maintenu.

Nous pouvons observer alors qu'au sein d'un même mouvement du logiciel libre, interagissent des oppositions radicales de visions politiques et sociales, en termes de moyens de développement, d'objectifs établis, de contributions réalisées et de collaborations constituées. Cet ensemble hétérogène véhicule cependant  une même figure lorsqu'il se relationne avec sa sphère politique majeure, comme mouvement uni, ou du moins désigné comme tel. Ces branches réformistes et progressistes, ou révolutionnaires sont mélangées dans une même entité sociale recueillie par le sens commun sous le nom générique de Mouvement du logiciel libre et open-source (FOSS). Pour cela, une des hypothèses qui peut être émise ici est que l'objet traditionellement entendu comme "mouvement" pourrait en fait être perçu comme un public. Dans ce sens, il est intéressant de mentionner l'effort, peut être contradictoire, des communautés du Libre pour exprimer un discours sur des sujets concurrents ou opposés, par exemple Libre et/ou Ouvert, pour que soit maintenue une même identité commune, celle d'un public diversifié, dont les interactions structurent un panorama technologique des logiciels. C'est ainsi que témoigne le fondateur du projet gNewSense, Paul O'Malley, lors d'un entretien :

\begin{quote}
Les développeurs de logiciels de divers domaines, c'est à dire autant de Linux que de BSD, utilisent des licences qui respectent les quatre libertés. Il y a beaucoup d'autres domaines, mais ces deux là suffisent aux objectifs de cette analogie. Chacun de ces champs permet facilement à chacun de participer, et au delà, de créer de l'interpolation, au sens où l'entendent les sciences sociales, justifiant auprès des personnes qui les rejoignent qu'il sont dans le bon camp, grâce à des philosophies internes consistantes. Celles-ci sont très similaires, et ne se différencie que dans certains cas marginaux. Chacune proclame être la plus valide, néanmoins leurs différences font qu'elles sont fortement distinctes.

Elles sont comme des jumeaux qui partagent beaucoup du même ADN, néanmoins, bien que séparées, ce sont des entités légitimes du logiciel libre. Un preuve positive de ceci est que si vous suivez chacune d'entre elles, une communauté a son travail mis dans MacOSX et l'autre dans le projet Debian, et qui peut éventuellement s'orienter vers gNS\footnote{“Software developers in the various camps, that is the developers of GNU/Linux systems, and developers in the various BSD camps use licences which respect the four freedoms. There are many other camps but these two will suffice for the purposes of this analogy. Both camps make it very easy for one to join and upon joining make it easy for one to interpolate in a social science way, justifying to the person joining their particular camp, and that most certainly they are in the right place by having internally consistent philosophies.  They are very similar, and the edge cases they differ. Both claim to have the ultimate validity, however they are by virtue of their differences very distinct entities. / They are like twins sharing a lot of DNA, in so far as they are legitimate free software entities albeit separate. Proof positive is when you get down streams of both, one community had their work put into OS X and the other into Debian project, the latter of which eventually turns into gNewSense.”}.
\end{quote}

Ainsi, se présente à la compréhension de l'observateur, non pas un mouvement politique homogène, mais un public diversifié dont la dynamique se constitue à partir d'un ensemble de mouvements concurrents, et parfois, opposés. Les philosophies développées par chacun des camps du Libre sont construites en fonction des projections et représentations que les différentes communautés produisent à partir des choix qu'elles réalisent en codifiant leur logiciel. Cependant, un ensemble commun (l'"ADN") contribue à promouvoir un modèle de développement plus abstrait et plus partagé.

De toute manière, qu'elle soit hybride ou absolue, une meme notion de liberté à laquelle doivent répondre les licences ("les quatres libertés") est entendue comme acquise. Peu importe qu'une communauté soit "élitiste" ou bien "éducative" face aux utilisateurs-développeurs, il y a une même idée de leur rôle, comme quelque chose de contributif à un effort de développement libre et ouvert. Enfin, pour autant que les applications faites de ces logiciels puissent être radicalement différentes, restant libres ou devenant propriétaires, l'ensemble primaire de logiciels reste accessible dans les mêmes termes par tout un chacun. Ces points communs constituent un public, dont les modalités communautaires produisent des mouvements qui font la dynamique politique du logiciel Libre.
