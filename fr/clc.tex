\chapter*{Conclusion}
\addcontentsline{toc}{chapter}{Conclusion}

\begin{quote}
\textit{On m'a raconté une plaisanterie que Richard Stallman à l'habitude de faire durant ses nombreuses apparitions publiques en congrès ou séminaires, afin de promouvoir le logiciel libre. Il va jusqu'au tableau ou prend une feuille de papier et commence à dessiner :
\begin{center}
\includegraphics[width=60mm]{gnu.png}
\end{center}
"Le mouvement du logiciel libre est tellement malin que même son nom est récursif !" s'exclame-t-il au milieu des rires des ingénieurs, sympathisants et amateurs du Libre.}
\end{quote}

Il y a une pensée contemporaine qui cherche à concevoir la technique comme un simple outil. Ainsi, le comptable, le mécanicien, de manière générale le \emph{technicien} possèdent une pratique hermétique qui en sert une autre, celle de la \emph{pensée} politique, philosophique, sociale, économique. On retrouve les origines de telles idées dans la philosophie grecque antique et sa conception classique du savoir comme un \emph{dedans} qui doit rester indépendant de son \emph{dehors}, la technê. La technique, elle, ne participe pas de la recherche du bonheur, de la beauté, du juste ou du vrai, elle la sert.

Surgit alors l'image platonicienne du sophiste, qui utilise la pensée comme une technique pour rendre n'importe quelle question indiscutable, souvent à des fins jugées immorales. De nombreuses historiographies de cette époque, cependant, ont montré que le courant sophiste avait été discrédité des piliers de l'histoire des idées et qu'il s'agissait en fait d'une philosophie à part entière, pragmatique et rationnelle, qui privilégie l'analyse des situations, des lieux, des évènements et des langages de manière concrète et non comme une fin en soi. 

Si ce n'est pas une décision politique qui conduit le mouvement technico-scientifique actuel et que nous refusons l'apriori d'immoralité ou d'inutilité d'une technê autonome, qu'est-ce qui conduit le mouvement des réseaux et des matériaux, qui construit notre monde déjà dépendant de lui ? On trouve des éléments de réponses dans l'oeuvre de Christopher Kelty et son anthropologie du Geek à travers l'étude des significations culturelles du mouvement du Libre. Selon lui, la participation collaborative et ouverte de nombreux individus a permis à un public \emph{récursif} légitime de se constituer. Tout comme la fonction récursive factorielle, fonction mathématique (figure \ref{fig3.2}) qui multiplie un nombre par les entiers qui le précèdent et ainsi produit un nouveau nombre plus grand, le public identifié par Kelty réalise son présent en invoquant ses éléments déjà existants.

%\begin{center}
\setcounter{equation}{1}
\begin{figure}[h]
\caption{La fonction factorielle} 
\begin{equation}
n!=\prod_{i=1}^n i= 1 \times 2 \times 3 \times \cdots \times (n - 1) \times n
\end{equation} \label{fig3.2}
\end{figure}
%\end{center}

Le logiciel est le concept que les communautés libres viennent invoquer récursivement. Le logiciel invoque le logiciel, qui invoque le logiciel, qui invoque le logiciel. Et ainsi produit des programmes, des formats, des protocoles qui participent au mouvement technologique contemporain.

Nous pourrions décrire un tel mouvement comme l'expérience vécue de la pensée pragmatique, selon laquelle, "l'effort n'est pas de pratiquer l'intelligence, mais d'intellectualiser la pratique"\citep{Dewey1929}. Il s'agit d'un refus de la recherche de la certitude par le moyen de catégories pré-établies par rapport au contexte d'analyse présent. \'A un niveau plus épistémologique, il s'agit d'une position anti-réificative, pour laquelle le concept et la théorie ne sont pas deux objets indépendants, mais des abstractions, dont le produit revient à l'expérience.

Dans ce sens, les premières lignes du \emph{Hacker Manifesto} énoncent : "toutes les classes sont effrayées par cette abstraction implacable du monde, dont les fortunes dépendent encore". Néanmoins, une "classe", la "classe hacker", se positionne différemment : "Nous sommes les hackers de l'abstraction. Nous produisons de nouveaux concepts, de nouvelles perceptions et sensations, hackés à partir de données brutes". Ici, dans une métaphore marxiste d'un monde divisé en classe, la "classe hacker" se différencie par une relation inversée à l'abstraction, comme si elle ne subissait pas sa cognition, mais la produisait à partir d'un traitement conscient des choses en leur état brut.

Comme nous l'avons observé dans le premier chapitre de cette étude, les alternatives offertes par le logiciel libre se sont constituées par le traitement récursif d'un matière première, Unix d'abord, les projets GNU, BSD, Mozilla, ensuite, afin de constituer un ensemble hétérogène en termes de technologie, de règlement et de sphère politique et sociale.

L'importance donnée à la figure de l'utilisateur-développeur dans le second chapitre nous a permis de trouver les mêmes termes au sein de l'éthique et des significations culturelles du hacking. Dans ce sens, nous trouvons des exemples dans les travaux de Gabriella Coleman, qui nous montre comment "l'agnosticisme politique" des communautés du Libre a permis aux individus de réinterpréter leur liberté technique et éthique dans un contexte digital. L'évolution, l'acte, le mouvement ne s'opèrent pas selon l'idée, le présupposé, des objectifs politiques préétablis, mais par la pratique d'un présent. Ce présent se réalise dans un acte, celui de programmer, dont les pragmatiques appartiennent tout autant à une responsabilité légale, de réglementation de l'espace digital, qu'à une responsabilité artistique, de maintenir et promouvoir une esthétique forte et élégante.

La relation à l'information qui nait de cette pragmatique de la programmation se structure  dans les spécificités des communautés que nous avons tenté de souligner chapitre trois. Ainsi, les pragmatiques de l'acte de programmation et les logiques culturelles du programmeur interagissent pour souligner les points caractéristiques des communautés gNewSensense, Samba et BSD, par les priorités qu'elles donnent dans leurs processus de traitement de l'information. Qu'il s'agisse de préoccupations de liberté, d'ouverture ou de sécurité, l'acte reste récursif, du fait de toujours traiter les problématiques communautaires à partir du développement des technologies et informations présentes et passées.

Ainsi se constitu un public dont les modalités donnent leurs contenus politiques à ses mouvements qui se différencient et parfois, s'opposent. Ce qui reste commun à ce public, c'est son traitement récursif des technologies pour produire de nouvelles technologies enrichies et modifiées. C'est ce mouvement technologique qui constitue la base de l'imaginaire politique de ses actants. La technique est entendue comme une infrastructure, un ensemble de moyens socialement identifiés et différenciés pour construire des systèmes opérationnels, des protocoles ou des formats de fichiers qui maintiennent un environnement technologique, alors dépendant de plusieurs éthiques dont les actants se sentent responsables.

Néanmoins, pour autant diversifiée qu'elle soit, l'éthique du Libre reste une alternative à un autre modèle, propriétaire, du fait d'exposer les liens entre la morale et la technique, entre l'infra-structure et la superstructure, le système opérationnel et le système social, comme tant d'objets qui doivent construire un public qui gagne sa légitimité en étant ouvert et libre.
